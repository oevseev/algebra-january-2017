\section{Теорема о полярном разложении оператора}

\begin{thm*}
    $L$ --- ЕП или УП, $a$ --- оператор. $a$ обратим $\Leftrightarrow$
    \begin{enumerate}
        \item $a$ можно представить в виде $a = r \circ u$, где $r$ --- положительно определенный самосопряженный оператор, $u$ --- изометрический оператор.
        \item Такое представление единственно.
    \end{enumerate}
\end{thm*}

\begin{proof}
    \begin{proofpart}
        Так как $a$ обратим, то $a \hat{a}$ --- положительно определенный самосопряженный оператор. По теореме об извлечении корня $\exists!$ положительно определенный самосопряженный оператор $r$, т.ч. $r^2 = a \hat{a}$. Т.к. $r$ положительно определен, то он обратим. Положим $u \coloneqq r^{-1} a$. Тогда $\hat{u} = \widehat{r^{-1} a} = \hat{a} \widehat{r^{-1}} = \hat{a} r^{-1}$, откуда $u\hat{u} = r^{-1} a\hat{a} r^{-1} = r^{-1} r^2 r^{-1} = \id_L$ и $u$ изометричен.
    \end{proofpart}

    \begin{proofpart}
        Предположим, что $a$ допускает еще одно полярное разложение в виде $a = r'u'$. Тогда:
        \begin{align*}
            \hat{a} &= \hat{u}\hat{r} = u^{-1} r \leadsto a\hat{a} = ru u^{-1} r = r^2 \\
            \hat{a} &= \hat{u'}\hat{r'} = (u')^{-1} r \leadsto a\hat{a} = r'u' (u')^{-1} r' = (r')^2
        \end{align*}
        Таким образом, и $r$, и $r'$ --- квадратные корни оператора $a\hat{a}$. Тогда они совпадают, откуда $ru = ru' \stackrel{r^{-1} \cdot}{\leadsto} u = u'$.
    \end{proofpart}
\end{proof}