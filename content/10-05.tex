\section{Теорема об изоморфизме сопряженности}

\begin{defn}
    Пусть $k$ --- коммутативное кольцо с единицей, $U, V, W$ --- $k$ модули. Множество всех билинейных отображений $U \times V \to W$ будем обозначать $\Bi(U, V; W)$. На этом множестве введем структуру $k$-модуля через поточечное сложение и умножение на скаляр.
\end{defn}

\begin{thm*}
    Пусть $k$ --- коммутативное кольцо с единицей, $U, V, W$ --- $k$-модули. Тогда существуют изоморфизмы $k$-модулей:
    $$\lsub{k}{\hom (U \otimes_k V, W)} \stackrel{\lambda}{\to} \lsub{k}{\Bi(U, V, W)} \stackrel{\mu}{\to} \lsub{k}{\hom(U, \lsub{k}{\hom(V, W)})}$$
\end{thm*}

\begin{proof}
    \begin{proofpart}
        Построим $\lambda$. Пусть $g \in \hom(U \otimes V, W)$, $t \colon U \times V \to U \otimes V$ --- каноническое билинейное отображение. Рассмотрим следующую диаграмму:
        \begin{diagram}
            U \times V & & \rTo^t & & U \otimes V \\
            & \rdDashto_{g \circ t} & & \ldTo_{g} & \\
            & & W & &
        \end{diagram}
        Так как $t$ билинейно, а $g$ --- гомоморфизм, то $g \circ t$ билинейно. Положим $\lambda(g) \coloneqq g \circ t$. $\lambda$ обратимо по универсальному свойству тензорного произведения. Ясно, что $\lambda$ является $k$-гомоморфизмом, следовательно $\lambda$ --- изоморфизм.
    \end{proofpart}

    \begin{proofpart}
        Построим $\mu$. Пусть $f \in \Bi(U, W; W)$. Положим $(\mu(f)(u))(v) \coloneqq f(u, v)$. $\mu(f)(u)$ и $\mu(f)$ являются $k$-гомоморфизмами вследствие билинейности $f$. Также ясно, что $\mu$ --- $k$-гомоморфизм. Рассмотрим $\nu \colon \hom(U, \hom(V, W)) \to \Bi(U, V; W)$, т.ч. $(\nu(h))(u, v) \coloneqq (h(u))(v)$. Ясно, что $\nu(h)$ билинейно, а $\nu$ является $k$-гомоморфизмом. Тогда:
        \begin{align*}
            (\nu \mu(f))(u, v) = ((\mu(f))(u))(v) = f(u, v) &\leadsto \nu \mu = \id_{\Bi(U, V; W)} \\
            (\mu \nu(h))(u)(v) = (\nu(h))(u, v) = h(u)(v) &\leadsto \mu \nu = \id_{\hom(U, \hom(V, W))}
        \end{align*}
        и $\mu$ --- изоморфизм.
    \end{proofpart}
\end{proof}

\begin{defn}
    Отображение $\mu \lambda$ называется \textit{изоморфизмом сопряженности}.
\end{defn}