\section{Тензорное произведение гомоморфизмов}

\begin{defn}
    $k$ --- коммутативное кольцо с единицей. Пусть $\alpha \colon U_1 \to U_2$ и $\beta \colon V_1 \to V_2$ --- гомоморфизмы $k$-модулей. Рассмотрим отображение $\alpha \times \beta \colon U_1 \times V_1 \to U_2 \times V_2$, т.ч. $(u, v) \mapsto (\alpha(u), \beta(v))$. Пусть $t_1 \colon U_1 \times V_1 \to U_1 \otimes_k V_1$ и $t_2 \colon U_2 \times V_2 \to U_2 \otimes_k V_2$ --- канонические билинейные отображения. Рассмотрим следующую диаграмму:
    \begin{diagram}
        U_1 \times V_1 & & \rTo^{t_1} & & U_1 \otimes_k V_1 \\
        & \rdTo_{t_2 \circ (\alpha \times \beta)} & & \ldDashto_{\alpha \otimes \beta} & \\
        & & U_2 \otimes_k V_2 & &
    \end{diagram}
    Ясно, что $t_2 \circ (\alpha \times \beta)$ билинейно. Тогда $\exists! \alpha \otimes \beta \colon U_1 \otimes_k V_1 \to U_2 \otimes_k V_2$, т.ч. $t_2 \circ (\alpha \times \beta) = (\alpha \otimes \beta) \circ t_1$. $\alpha \otimes \beta$ называется \textit{тензорным произведением гомоморфизмов}.
\end{defn}

\begin{thm}
    $\id_{U_1} \otimes \id_{V_1} = \id_{U_1 \otimes_k V_1}$.
\end{thm}

\begin{proof}
    Имеем
    \begin{align*}
        t_2 \circ (\id_{U_1} \times \id_{V_1}) &\colon (u, v) \mapsto (u, v) \mapsto u \otimes v \\
        \id_{U_1 \otimes_k V_1} \circ t_1 &\colon (u, v) \mapsto u \otimes v \mapsto u \otimes v
    \end{align*}
    откуда $\id_{U_1 \otimes_k V_1} = \id_{U_1} \otimes \id_{V_1}$.
\end{proof}

\begin{thm}
    Пусть даны $k$-гомоморфизмы:
    \begin{diagram}
        U_1 & \rTo^{\alpha_1} & U_2 & \rTo^{\alpha_2} & U_3 \\
        V_1 & \rTo^{\beta_1}  & V_2 & \rTo^{\beta_2}  & V_3
    \end{diagram}
    Тогда $(\alpha_2 \alpha_1) \otimes (\beta_2 \beta_1) = (\alpha_2 \otimes \beta_2) \circ (\alpha_1 \otimes \beta_1)$.
\end{thm}

\begin{proof}
    Построим следующую (заведомо коммутативную) диаграмму:
    \begin{diagram}
        & & U_1 \times V_1 & \rTo^{t_1} & U_1 \otimes V_1 & & \\
        & \ldTo(2,4)^{(\alpha_2 \alpha_1) \times (\beta_2 \beta_1)} & \dTo_{\alpha_1 \times \beta_1} & & \dTo_{\alpha_1 \otimes \beta_1} & & \\
        & & U_2 \times V_2 & \rTo^{t_2} & U_2 \otimes V_2 & & \\
        & \ldTo_{\alpha_2 \times \beta_2} & & & & \rdTo^{\alpha_2 \otimes \beta_2} \\
        U_3 \times V_3 & & & \rTo^{t_3} & & & U_3 \otimes V_3
    \end{diagram}
    Тогда $(\alpha_2 \otimes \beta_2) \circ (\alpha_1 \otimes \beta_1) \circ t_1 = t_3 \circ ((\alpha_2\alpha_1) \times (\beta_2 \beta_1))$, откуда $(\alpha_2 \alpha_1) \otimes (\beta_2 \beta_1) = (\alpha_2 \otimes \beta_2) \circ (\alpha_1 \otimes \beta_1)$.
\end{proof}

\begin{thm}
    Пусть $\alpha$ и $\beta$ --- изоморфизмы, тогда $\alpha \otimes \beta$ --- изоморфизм.
\end{thm}

\begin{proof}
    $(\alpha \otimes \beta) \circ (\alpha^{-1} \otimes \beta^{-1}) = (\alpha \alpha^{-1}) \otimes (\beta \beta^{-1}) = \id_{U_1} \otimes \id_{V_1} = \id_{U_1 \otimes V_1}$. Аналогично, $(\alpha^{-1} \otimes \beta^{-1}) \circ (\alpha \otimes \beta) = \id_{U_2 \otimes V_2}$.
\end{proof}

\begin{thm}
    $\alpha \otimes \beta = (\alpha \otimes \id_{V_1}) \circ (\id_{U_1} \otimes \beta) = (\id_{U_1} \otimes \beta) \circ (\alpha \otimes \id_{V_1})$.
\end{thm}

\begin{proof}
    Очевидно\footnote{На самом деле --- ни хера}.
\end{proof}

\begin{thm}
    Пусть $\alpha_1, \alpha_2 \in \lsub{K}{\hom(U_1, U_2)}, \beta \in \lsub{K}{\hom(V_1, V_2)}$. Тогда $(\alpha_1 + \alpha_2) \otimes \beta = \alpha_1 \otimes \beta + \alpha_2 \otimes \beta$.
\end{thm}

\begin{proof}
    $((\alpha_1 + \alpha_2) \otimes \beta) \circ t_1 = t_2 \circ ((\alpha_1 + \alpha_2) \times \beta) = t_2 \circ (\alpha_1 \times \beta) + t_2 \circ(\alpha_2 \times \beta) = (\alpha_1 \otimes \beta) \circ t_2 + (\alpha_2 \otimes \beta) \circ t_2 = (\alpha_1 \otimes \beta + \alpha_2 \otimes \beta) \circ t_2$, откуда $(\alpha_1 + \alpha_2) \otimes \beta = \alpha_1 \otimes \beta + \alpha_2 \otimes \beta$.
\end{proof}