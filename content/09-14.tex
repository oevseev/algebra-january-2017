\section{Определение унитарного пространства. Неравенства Коши и треугольника в УП}

\begin{defn}
    \textit{Унитарным пространством} называется пара $(\lsub{\mathbb{C}}{L}, F)$, где $\lsub{\mathbb{C}}{L}$ --- \textbf{конечномерное} $\mathbb{C}$-линейное пространство, а $F \colon L^2 \to \mathbb{C}$ --- положительно определенная \textbf{эрмитова} форма (\textit{скалярное произведение}).
    Аналогично ЕП вводятся понятия \textit{нормы} вектора и \textit{расстояния} между двумя векторами.
\end{defn}

\begin{thm*}
    $L$ --- УП. Пусть $u, v \in L$. Тогда:
    \begin{enumerate}
        \item $|(u, v)| \le ||u|| \cdot ||v||$
        \item $||u + v|| \le ||u|| + ||v||$
        \item $\rho(u, v) \le \rho(u, w) + \rho(w, v)\ \forall w \in L$
    \end{enumerate}
\end{thm*}

\begin{proof}
    \begin{proofpart}
        Пусть $t \in \mathbb{C}$. Считаем, что $v \neq \nil$. Тогда $0 \le (u + tv, u + tv) = \norm{u}^2 + t(v, u) + \bar{t}(u, v) + {|t|}^2\norm{v}^2$. Положим $t = \lambda(u, v),\ \lambda \in \mathbb{R}$. Тогда $0 \le \norm{u}^2 + 2\lambda{|(u, v)|}^2 + \lambda^2 {|(u, v)|}^2 \norm{v}^2$. В таком случае $D = 4{|(u, v)|}^4 - 4{|(u, v)|}^2\norm{u}^2\norm{v}^2 \le 0$, откуда $|(u, v)| \le \norm{u} \cdot \norm{v}$
    \end{proofpart}
    
    \begin{proofpart}
        Аналогично доказательству для ЕП.
    \end{proofpart}
    
    \begin{proofpart}
        Аналогично доказательству для ЕП.
    \end{proofpart}
\end{proof}