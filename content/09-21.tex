\section{Изометрические операторы, простейшие свойства}

\begin{defn}
    Пусть $L$ --- ЕП или УП, $a$ --- оператор $L$. $a$ называется \textit{изометрическим}, если $\forall u, v\ (a(u), a(v)) = (u, v)$. Изометрические операторы в УП называются \textit{унитарными}, а в ЕП --- \textit{ортогональными}.
\end{defn}

\begin{thm*}
    $L$ --- ЕП или УП, $a$ --- оператор. $a$ изометричен $\Leftrightarrow$ $\hat{a} a = \id_L$.
\end{thm*}

\begin{proof}
    \begin{proofpart}{($\Rightarrow$)}
        $\forall u, v \in L$ имеем $(u, v) = (a(u), a(v)) = (u, \hat{a}a(v))$, откуда $\hat{a}a(v) = v$ и $\hat{a} a = \id_L$.
    \end{proofpart}

    \begin{proofpart}{($\Leftarrow$)}
        Имеем $(a(u), a(v)) = (u, \hat{a}a(v)) = (u, v)$ и $a$ изометричен.
    \end{proofpart}
\end{proof}

\begin{cor}
    Любой изометрический оператор обратим и обратен своему сопряженному.
\end{cor}

\begin{proof}
    Следует из формулировки теоремы.
\end{proof}

\begin{cor}
    $a$ изометричен $\Leftrightarrow$ $\hat{a}$ изометричен.
\end{cor}

\begin{proof}
    Следует из того, что $\hat{\hat{a}} = a$.
\end{proof}

\begin{cor}
    Любой изометрический оператор нормален.
\end{cor}

\begin{proof}
    $\forall a\ \hat{a} a = \id_L = a \hat{a}$ и $a$ нормален.
\end{proof}

\begin{cor}
    $L$ --- УП, $a$ --- оператор $L$. Равносильны:
    \begin{enumerate}
        \item $a$ --- унитарный оператор.
        \item Для любого ОН-базиса $B$ матрица $[a]_B$ унитарна.
        \item Существует ОН-базис $B$, т.ч. матрица $[a]_B$ унитарна.
    \end{enumerate}
\end{cor}

\begin{cor}
    $L$ --- ЕП, $a$ --- оператор $L$. Равносильны:
    \begin{enumerate}
        \item $a$ --- ортогональный оператор.
        \item Для любого ОН-базиса $B$ матрица $[a]_B$ ортогональна.
        \item Существует ОН-базис $B$, т.ч. матрица $[a]_B$ ортогональна.
    \end{enumerate}
\end{cor}