\section{Приведение пары вещественных квадратичных форм к каноническому виду}

\begin{thm*}
    Пусть $Q_1, Q_2 \colon \lsub{\mathbb{R}}{L} \to \mathbb{R}$ --- квадратичные формы, причем $L$ конечномерно. Пусть дополнительно $Q_1$ положительно определена. Тогда существует базис $B$ в $\lsub{\mathbb{R}}{L}$, т.ч. $[Q_1]_B$ и $[Q_2]_B$ одновременно диагональны.
\end{thm*}

\begin{proof}
    Пусть $F_1, F_2 \colon L^2 \to \mathbb{R}$ --- соответствующие билинейные формы. В силу положительной определенности $F_1$ пространство $(\lsub{\mathbb{R}}{L}, F_1)$ будет являться евклидовым. Выберем ОН-базис $\tilde{B}$ в $\lsub{\mathbb{R}}{L}$. Положим $\tilde{F_i} \coloneqq [F_i]_{\tilde{B}}$.
    
    Ясно, что $\tilde{F_1} = E_n$. Рассмотрим оператор $a$ на $L$, т.ч. $[a]_{\tilde{B}} = \tilde{F_2}$. Так как $\tilde{F_2}$ симметрична (как матрица симметрической билинейной формы), то $a$ самосопряжен. В таком случае существует ОН-базис $B$ в $\lsub{\mathbb{R}}{L}$, т.ч. $M_2 \coloneqq [a]_B = \diag(\lambda_1, \dots, \lambda_n)$. Пусть $\tilde{B} \stackrel{C}{\leadsto} B$. $C$ ортогональна как матрица перехода между ОН-базисами.
    
    Имеем:
    \begin{align*}
        [Q_2]_B &= [F_2]_B = C^T \tilde{F_2} C = C^{-1} \tilde{F_2} C = C^{-1} [a]_{\tilde{B}} C = M_2\ \text{(диагональная матрица)} \\
        [Q_1]_B &= [F_1]_B = C^T \tilde{F_1} C = C^T C = E_n\ \text{(диагональная матрица)}
    \end{align*}
\end{proof}