\section{Линейные функционалы на УП}

\begin{defn}
    Пусть $\lsub{\mathbb{C}}{L}$ --- УП, $v \in L$. Отображение $\varphi_v \colon L \to \mathbb{C}$, т.ч. для $u \in L$ $\varphi_v(u) = (u, v)$ назовем \textit{функционалом скалярного умножения на $v$}. Ясно, что $\varphi_v \in L^*$.
\end{defn}

\begin{defn}
    $\lsub{\mathbb{C}}{L}$ --- $C$-линейное пространство. На $L$ введем прежнее сложение и новое умножение на скаляры $\mu$, т.ч. $\mu(\alpha, v) \coloneqq \bar{\alpha} v$. Ясно, что $(L, +, \mu) \eqqcolon \bar{L}$ --- $\mathbb{C}$-линейное пространство.
\end{defn}

\begin{thm*}
    Пусть $\lsub{\mathbb{C}}{L}$ --- УП. Рассмотрим отображение $f \colon \bar{L} \to L^*$, т.ч. $f(v) = \varphi_v$. Справедливы следующие утверждения:
    \begin{enumerate}
        \item $f$ --- изоморфизм.
        \item Если $B$ --- ОН-базис в $\lsub{\mathbb{C}}{\bar{L}}$, то $f(B)$ --- дуальный к нему базис в $L^*$.
    \end{enumerate}
\end{thm*}

\begin{proof}
    \begin{proofpart}
        $$f(u + v) = (\bullet, u + v) = (\bullet, u) + (\bullet, v) = f(u) + f(v)$$
        $$f(\mu(\alpha, v)) = (\bullet, \bar{\alpha}v) = \bar{\bar{\alpha}}(\bullet, v) = \alpha(\bullet, v) = \alpha f(v)$$
        Отсюда $f$ --- $\mathbb{C}$-линейное отображение. Предположим, что $f(v) = 0$, то есть $\forall u \in L\ (u, v) = 0$, тогда $(v, v) = 0 \leadsto v = \nil \leadsto \ker f = \nilset$ и $f$ --- мономорфизм. Но $\dim \lsub{\mathbb{C}}{\bar{L}} = \dim \lsub{\mathbb{C}}{L^*}$, и $f$ --- изоморфизм.
    \end{proofpart}
    \begin{proofpart}
        Пусть $B = \family{e_i}{i=1}{n}$. Тогда $f(B) = \family{\varphi_{e_j} = (\bullet, e_j)}{j=1}{n}$. Но $\varphi_{e_j}(e_i) = (e_i, e_j) = \delta_{ij}$, откуда $f(B)$ --- дуальный к $B$ базис $L^*$.
    \end{proofpart}
\end{proof}