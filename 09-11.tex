\section{Ортогональное проектирование на подпространство}

\begin{defn}
    $L$ --- ЕП, $V \le \lsub{\mathbb{R}}{L}$, $L = V \oplus V^\perp$. Проектор на $V$ параллельно $V^\perp$ называется \textit{оператором ортогонального проектирования на $V$}. В этом случае $v \in L$ единственным образом представляется в виде $v = v_{\text{pr}} + v^\perp$, где $v_{\text{pr}}$ называется \textit{ортогональной проекцией $v$ на $V$}, а $v^\perp$ --- \textit{ортогональной составляющей $v$}.
\end{defn}

\begin{defn}
    $L$ --- ЕП, $V \le \lsub{\mathbb{R}}{L}$, $v \in L$. Величина $\rho(v, V) \coloneqq \inf \left\{ \rho(v, x) \mid x \in V \right\}$ называется \textit{расстоянием между $v$ и $V$}.
\end{defn}

\begin{thm*}
    $L$ --- ЕП, $V \le \lsub{\mathbb{R}}{L}$, $v \in L$. Пусть $p$ --- оператор ортогонального проектирования на $V$, тогда $\rho(v, V) = \norm{v^\perp} = \norm{v - p(v)}$.
\end{thm*}

\begin{proof}
    Положим $v_{\text{pr}} \coloneqq p(v) \in V$. Очевидно, что $\rho(v, V) \le \norm{v^\perp}$ (т.к. $\rho(v, v_\text{pr}) = \norm{v - v_pr} = \norm{v^\perp}$). В то же время, полагая $u \in V$, имеем $\rho^2(v, u) = \norm{v - u}^2 = \norm{v^\perp + (v_\text{pr} - u)}^2 = \norm{v^\perp}^2 + \norm{v_\text{pr} - u}^2 \ge \norm{v^\perp}^2$ (в силу ортогональности векторов суммы), откуда $\rho(v, V) \ge \norm{v^\perp}$. Так, $\rho(v, V) = \norm{v^\perp}$.
\end{proof}