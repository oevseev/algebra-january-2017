\section{Свойства ортогонального дополнения в ЕП}

\begin{defn}
    $L$ --- ЕП, $V \le \lsub{\mathbb{R}}{L}$. Подпространство $V^\perp = \left\{x \in L \mid \forall v \in V\ (v, x) = 0 \right\} \le L$ называется \textit{ортогональным дополнением} к $V$.
\end{defn}

\begin{thm*}
    $L$ --- ЕП. Пусть $V, V_1, V_2 \le \lsub{\mathbb{R}}{L}$. Справедливы следующие утверждения:
    \begin{enumerate}
        \item $V_1 \subset V_2 \Rightarrow V_2^\perp \subset V_1^\perp$
        \item $(V_1 + V_2)^\perp = V_1^\perp \cap V_2^\perp$
        \item $L = V \oplus V^\perp$
        \item $(V^\perp)^\perp = V$
        \item $(V_1 \cap V_2)^\perp = V_1^\perp + V_2^\perp$
    \end{enumerate}
\end{thm*}

\begin{proof}
    \begin{proofpart}
        Рассмотрим произвольный $v_2' \in V_2^\perp$. Имеем $\forall v_2 \in V_2\ (v_2, v_2') = 0 \Rightarrow \forall v_1 \in V_1\ (v_1, v_2') = 0$. При этом $\forall v_1 \in V_1\ \forall v_1' \in V_1^\perp\ (v_1, v_1') = 0$. Так как $V_1^\perp$ максимально по включению, $V_2^\perp \subset V_1^\perp$.
    \end{proofpart}

    \begin{proofpart}
        $V_1, V_2 \subset V_1 + V_2 \stackrel{\text{(а)}}{\Rightarrow} (V_1 + V_2)^\perp \subset V_1^\perp \cap V_2^\perp$. В то же время, пусть $u \in V_1^\perp \cap V_2^\perp$ и $v = v_1 + v_2 \in V_1 + V_2$, тогда $(u, v) = (u, v_1) + (u, v_2) = 0 + 0 = 0 \leadsto u \in (V_1 + V_2)^\perp$ и обратное включение доказано.
    \end{proofpart}

    \begin{proofpart}
        Считаем, что $V \neq \{\nil\}$. Выберем в $V$ ОН-базис $\family{e_i}{i=1}{k}$ и дополним его до ОН-базиса $L$: $\family{e_i}{i=1}{n}$. Докажем, что $V^\perp = \gen{}{e_{k+1}, \dots, e_n}$. Пусть $v = \sum_{i=1}^n \alpha_i e_i \in V^\perp$. Тогда $\forall j \le k$ имеем $0 = (e_j, v) = \sum_{i=1}^n \alpha_i (e_i, e_j) = \sum_{i=1}^n \alpha_i \delta_{ij} = \alpha_j$, откуда $v \in \gen{}{e_{k+1}, \dots, e_n}$. Обратное включение очевидно. Так, $V = V \oplus V^\perp$ (данная сумма прямая в силу того, что если $x \in X \cap X^\perp$, то $(x, x) = 0$ и $x = \nil$).
    \end{proofpart}

    \begin{proofpart}
        $L = V \oplus V^\perp$. Выделяя несобственные подпространства в $V$ и $V^\perp$ и применяя пункт (в), получаем $L = V^\perp \oplus V^{\perp\perp}$. При этом $\dim \lsub{\mathbb{R}}{V} = \dim \lsub{\mathbb{R}}{V^{\perp\perp}}$ и очевидно, что $V \subset V^{\perp\perp}$, откуда $V = V^{\perp\perp}$.
    \end{proofpart}

    \begin{proofpart}
        $(V_1 \cap V_2)^\perp \stackrel{\text{(б)}}{=} (V_1^\perp + V_2^\perp)^{\perp\perp} \stackrel{\text{(г)}}{=} V_1^\perp + V_2^\perp$.
    \end{proofpart}
\end{proof}