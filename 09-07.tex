\section{Линейные функционалы на линейном пространстве. Дуальный базис}

\begin{defn}
    $K$ --- поле, $\lsub{K}{V}$ --- линейное пространство. $V^* = \hom(V, K)$ называется \textit{пространством функционалов (также дуальным, двойственным, сопряженным пространством)} на $V$.
\end{defn}

\begin{thm}
    $\dim \lsub{K}{V} < \infty \Rightarrow \dim \lsub{K}{V^*} = \dim \lsub{K}{V}$.
\end{thm}

\begin{proof}
    $\dim \lsub{K}{V} = \dim \lsub{K}{\hom(V, K)} = \dim \lsub{K}{V} \cdot \dim \lsub{K}{K} = \dim \lsub{K}{V}$.
\end{proof}

\begin{defn}
    Пусть $B = \family{u_i}{i=1}{n}$ --- базис $\lsub{K}{V}$. \textit{Дуальным} к $B$ называется базис $B^* = \family{\phi_j}{j=1}{n}$ пространства $V^*$ такой, что $\phi_j(u_i) = \delta_{ij}$.
\end{defn}

\begin{thm}
    $B^*$ --- действительно базис $\lsub{K}{V^*}$.
\end{thm}

\begin{proof}
    Пусть $\phi \in V^*$.
    
    \begin{align*}
        \phi(v) &= \phi\left(\sum_{i=1}^n \alpha_i u_i\right) = \sum_{i=1}^n \alpha_i \phi(u_i) = \sum_{i=1}^n \alpha_i \sum_{j=1}^n \phi(u_j) \phi_j(u_i) \\
        &= \sum_{j=1}^n \phi(u_j) \sum_{i=1}^n \alpha_i \phi_j(u_i) = \sum_{j=1}^n \phi(u_j) \phi_j \left(\sum_{i=1}^n \alpha_i u_i \right) = \sum_{j=1}^n \phi(u_j) \phi_j(v)
    \end{align*}
    
    Отсюда $\phi \in \gen{K}{B^*}$. Так как $\#B^* = \dim \lsub{K}{V^*}$, то $B^*$ --- базис $V^*$.
\end{proof}