\section{Дуальное отображение. Свойства операции перехода к дуальному отображению. Второе пространство функционалов, его связь с исходным пространством}

\begin{defn}
    Пусть $f \colon \lsub{K}{U} \to \lsub{K}{V}$ --- $K$-линейное отображение. Построим отображение $f^* \colon V^* \to U^*$ следующим образом:
    
    \begin{diagram}
        U &                       & \rTo^f &            & V \\
          & \rdDashto_{f^*(\phi)} &        & \ldTo_\phi &   \\
          &                       & K      &            &   
    \end{diagram}

    То есть, для $\phi \in V^*$ положим $f^*(\phi) \coloneqq \phi \circ f \in U^*$. Очевидно, что оно $K$-линейно. $f^*$ называется \textit{дуальным к $f$} отображением.
\end{defn}

\begin{rem}
    Дуальное отображение обладает следующими свойствами:
    \begin{enumerate}
        \item $(f_1 + f_2)^*(\phi) = \phi \circ (f_1 + f_2) = \phi \circ f_1 + \phi \circ f_2 = f_1^*(\phi) + f_2^*(\phi)$
        \item $(gf)^*(\phi) = \phi (gf) = (\phi g)f = f^*(\phi g) = f^* (g^*(\phi)) = (f^* \circ g^*)(\phi)$
    \end{enumerate}
\end{rem}

\begin{defn}
    $\lsub{K}{V}$ --- линейное пространство. Построим отображение $\delta_V \colon V \to V^{**}$ так, чтобы $\forall \phi \in V^*\ (\delta_V(v))(\phi) = \phi(v)$ ($\delta_V(v) = \bullet(v)$).
\end{defn}

\begin{thm*}
    $\lsub{K}{U}$, $\lsub{K}{V}$ --- линейные пространства.
    \begin{enumerate}
        \item $\delta_V$ --- изоморфизм.
        \item Пусть $f \colon U \to V$ --- $K$-линейное отображение, тогда $\delta_V \circ f = f^{**} \circ \delta_U$.
    \end{enumerate}
\end{thm*}

\begin{rem}
    $\delta$ можно рассматривать как функтор в терминах теории категорий (следующая диаграмма коммутирует):
    
    \begin{diagram}
        U                    & \rTo^f        & V                    \\
        \dTo^{\delta_U}_\sim &               & \dTo_{\delta_V}^\sim \\
        U^{**}               & \rTo^{f^{**}} & V^{**}
    \end{diagram}
\end{rem}

\begin{proof}
    \begin{proofpart}
        Докажем, что $\delta_V$ мономорфно. Пусть $v \in V \setminus \nilset$ и $\delta_V(v) = 0$. Тогда $\forall \phi \in V^*\ \phi(v) = 0$. Построим базис $\{v, \dots\}$ и двойственный к нему $\{\phi_1, \dots\}$. Отсюда по определению $\phi_1(v) = 1$. Полученное противоречие доказывает мономорфность $\delta_V$. Так как $\delta_V$ --- мономорфизм пространств одинаковой размерности, то $\delta_V$ --- изоморфизм.
    \end{proofpart}

    \begin{proofpart}
        \begin{align*}
            ((\delta_V \circ f)(u))(\phi) &= (\delta_V(f(u)))(\phi) = \phi(f(u)) = (\phi \circ f)(u) = (f^*(\phi))(u) = (\delta_U(u))(f^*(\phi)) \\
            &= (\delta_U(u) \circ f^*)(\phi) = (f^{**}(\delta_U(u)))(\phi) = (f^{**} \circ \delta_U(u))(\phi)
        \end{align*}
        
        Так, $\delta_V \circ f = f^{**} \circ \delta_U$.
    \end{proofpart}
\end{proof}