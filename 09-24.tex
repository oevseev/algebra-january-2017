\section{Вращения в $\mathbb{R}^3$, теорема Эйлера. Несобственно ортогональные операторы в $\mathbb{R}^2$}

\begin{defn}
    Пусть $L$ --- ЕП, $a$ --- \textbf{ортогональный} оператор. Выберем $B$ --- произвольный ОН-базис $\lsub{\mathbb{R}}{L}$, т.ч. $\det [a]_B = \pm 1$. В случае, если $\det [a]_B = 1$, $a$ назовем \textit{собственно ортогональным}, в противном случае --- \textit{несобственно ортогональным}.
\end{defn}

\begin{defn}
    Собственно ортогональные операторы на $\mathbb{R}^2$ и $\mathbb{R}^3$ со стандартным скалярным произведением называют вращениями.
\end{defn}

\begin{thm}
    Пусть $a \in \End \lsub{\mathbb{R}}{\mathbb{R}^3}$ --- вращение. Тогда существует ОН-базис $B$ в $R^3$, т.ч. $$[a]_B = \begin{pmatrix}
        \cos \phi & -\sin \phi & 0 \\
        \sin \phi & \cos \phi  & 0 \\
        0         & 0          & 1
    \end{pmatrix}$$
\end{thm}

\begin{proof}
    Следует из более общей теоремы о канонической форме матрицы ортогонального оператора.
\end{proof}

\begin{thm}{(Эйлера)}
    Композиция двух вращений в $\mathbb{R}^3$ вокруг пересекающихся осей является вращением вокруг некоторой третьей оси.
\end{thm}

\begin{proof}
    Следует из того, что вращения в $R^3$ образуют группу.
\end{proof}

\begin{thm}
    Любой несобственно ортогональный оператор в $\mathbb{R}^2$ является отражением отностиельно некоторой прямой.
\end{thm}

\begin{proof}
    Каноническая форма несобственно ортогонального оператора в $\mathbb{R}^2$ имеет вид $\begin{pmatrix}
        1 & 0 \\
        0 & -1
    \end{pmatrix}$, а отсюда очевидным образом следует утверждение теоремы.
\end{proof}