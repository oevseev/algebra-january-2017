\documentclass{scrartcl}

\usepackage[utf8]{inputenc}
\usepackage[T1, T2A]{fontenc}
\usepackage[russian]{babel}

% LaTeX preamble for algebra named after Generalov A.I.
% (c) 2016, 2017 Oleg Evseev. For in-house use only.

\usepackage{amsmath}
\usepackage{amsfonts}
\usepackage{amssymb}
\usepackage{amsthm}
\usepackage{enumitem}
\usepackage{graphicx}
\usepackage{leftidx}
\usepackage{mathtools}
\usepackage{tikz}
\usepackage{marvosym} % What the fuck
\usepackage{tabu}

% For use with Apple Watch
\usepackage{xcolor}
\usepackage{pagecolor}

% diagrams.sty can be found at http://www.paultaylor.eu/diagrams/
\usepackage[small,nohug,heads=vee]{diagrams}
\diagramstyle[labelstyle=\scriptstyle]

% Structural defines

\newtheorem{thm}{Теорема}
\newtheorem*{thm*}{Теорема}
\newtheorem{prop}[thm]{Предложение} 
\newtheorem*{prop*}{Предложение}
\newtheorem{cor}{Следствие}
\newtheorem*{cor*}{Следствие}

\theoremstyle{definition}
\newtheorem*{defn}{Определение}

\theoremstyle{remark}
\newtheorem*{rem}{Замечание}
\newtheorem*{exmpl}{Пример}

\newtheoremstyle{lemma}{}{}{}{}{\bfseries}{.}{.5em}{}
\theoremstyle{lemma}
\newtheorem{lem}{Лемма}
\newtheorem*{lem*}{Лемма}

\newtheoremstyle{part}{}{}{}{}{\bfseries}{)}{.5em}{}
\theoremstyle{part}
\newtheorem{proofpart}{}

\makeatletter
\@addtoreset{equation}{section}
\@addtoreset{thm}{subsection}
\@addtoreset{cor}{thm}
\@addtoreset{proofpart}{thm}
\@addtoreset{proofpart}{lem}
\@addtoreset{proofpart}{cor}
\makeatother

\renewcommand{\thesection}{\arabic{part}.\arabic{section}}
\renewcommand*{\theproofpart}{\asbuk{proofpart}}
\renewcommand*{\thelem}{\Asbuk{lem}}

\renewcommand{\theenumi}{(\asbuk{enumi})}
\renewcommand{\labelenumi}{\asbuk{enumi})}

\AddEnumerateCounter{\Asbuk}{\@Asbuk}{\CYRM}
\AddEnumerateCounter{\asbuk}{\@asbuk}{\cyrm}

% Some useful declarations

\DeclareMathOperator{\End}{End}
\DeclareMathOperator{\id}{id}
\DeclareMathOperator{\im}{Im}
\DeclareMathOperator{\diag}{diag}
\DeclareMathOperator{\rk}{rk}
\DeclareMathOperator{\spec}{sp}
\DeclareMathOperator{\Bi}{Bi}

\renewcommand{\hom}{\operatorname{Hom}}
\renewcommand{\ker}{\operatorname{Ker}}

\newcommand\lsub[2]{\leftidx{_{#1}}{#2}}
\newcommand\family[3]{\left\{#1\right\}_{#2}^{#3}}
\newcommand\gen[2]{\lsub{#1}{\left\langle #2 \right\rangle}}

\newcommand\divby{\mathrel{\vdots}}
\newcommand\ndivby{\mathrel{\not\vdots}}

\newcommand\nil{\mathbf{0}}
\newcommand\nilset{\{\nil\}}
\newcommand{\norm}[1]{\left\lVert#1\right\rVert}

\newcommand{\inup}{\mathbin{\rotatebox[origin=c]{90}{$\in$}}}

\newcommand*\circled[1]{\tikz[baseline=(char.base)]{
    \node[shape=circle,draw,inner sep=2pt] (char) {#1};}}
    
% Thanks to http://texblog.net/latex-archive/maths/amsmath-matrix/
\makeatletter
\renewcommand*\env@matrix[1][*\c@MaxMatrixCols c]{%
    \hskip -\arraycolsep
    \let\@ifnextchar\new@ifnextchar
    \array{#1}}
\makeatother

% Part names on separate pages, found here:
% https://tex.stackexchange.com/questions/64215/make-part-in-scrartcl
\renewcommand\partheadstartvskip{\clearpage\null\vfil}
\renewcommand\partheadmidvskip{\par\nobreak\vskip 20pt\thispagestyle{empty}}
\renewcommand\partheadendvskip{\vfil\clearpage}
\renewcommand\raggedpart{\centering}


% Use either, not both
\usepackage[top=2cm, bottom=2cm, left=2cm, right=2cm]{geometry} % Standard
% % Apple Watch
\pagecolor{black}
\color{white}
\usepackage[paperwidth=120mm, paperheight=150mm, top=5mm, bottom=5mm, left=5mm, right=5mm]{geometry}

%\sloppy
 % Apple Watch

\title{Ответы на вопросы экзамена по АТЧ}
\subtitle{(январь 2017 г., 3-й семестр ПМиИ)}
\author{А. Константинов, О. Евсеев, Г. Енгалыч}

\begin{document}
    \pagenumbering{gobble}
    \thispagestyle{empty}
    \maketitle
    
    \begin{center}
        \vspace{120pt}
        \includegraphics[width=32em]{cat.png}
    \end{center}    

    \setcounter{part}{8}
    
    % Глава IX. Евклидовы и унитарные пространства
    \part{Евклидовы и унитарные пространства}
    \section{Определение евклидова пространства. Неравенства Коши и треугольника в ЕП}

\begin{defn}
    \textit{Евклидово пространство} --- пара $(\lsub{\mathbb{R}}{L}, (-, -))$, где $\lsub{\mathbb{R}}{L}$ --- \textbf{конечномерное} пространство, $(-, -) \colon L^2 \to \mathbb{R}$ --- положительно определенная билинейная форма.
    
    Величина $||u|| \coloneqq \sqrt{(u, u)}$ называется \textit{длиной (нормой) вектора $u$}.
    
    Величина $\rho(u, v) = ||u - v||$ называется \textit{расстоянием} между $u$ и $v$.
\end{defn}

\begin{thm*}
    $L$ --- ЕП. Пусть $u, v \in L$. Тогда:
    \begin{enumerate}
        \item $|(u, v)| \le ||u|| \cdot ||v||$
        \item $||u + v|| \le ||u|| + ||v||$
        \item $\rho(u, v) \le \rho(u, w) + \rho(w, v)\ \forall w \in L$
    \end{enumerate}
\end{thm*}

\begin{proof}
    \begin{proofpart}
        Пусть $t \in \mathbb{R}$. Считаем, что $u \neq \nil \neq v$. Тогда $(u + tv, u + tv) = (u, u) + 2(u, tv) + (tv, tv) = {||u||}^2 + 2t(u, v) + t^2{||v||}^2 > 0\ \forall t$, откуда $D = 4(u, v)^2 - 4{||u||}^2{||v||}^2 \le 0$ и $|(u, v)| \le ||u|| \cdot ||v||$.
    \end{proofpart}

    \begin{proofpart}
        ${||u + v||}^2 = (u + v, u + v) = {||u||}^2 + 2(u, v) + {||v||}^2 \le {||u||}^2 + 2||u|| \cdot ||v|| + {||v||}^2 = (||u|| + ||v||)^2$
    \end{proofpart}

    \begin{proofpart}
        $\rho(u, v) = ||u - v|| = ||(u - w) + (w - v)|| \le ||u - w|| + ||w - v|| = \rho(u, w) + \rho(w, v)$
    \end{proofpart}
\end{proof} % Определение евклидова пространства. Неравенства Коши и треугольника в ЕП
    \section{Матрица Грама набора векторов. Связь с линейной независимостью векторов}

\begin{defn}
    Пусть $M = \family{v_i}{i=1}{m}$ --- семейство векторов ЕП $L$. Матрица $G_M = ((v_i, v_j))_{ij}$ (где $i, j \in 1..m$) называется \textit{матрицей Грама} семейства $M$.
\end{defn}

\begin{rem}
    Ясно, что $G_M$ симметрична. Если $M = \family{e_i}{i=1}{m}$ --- базис, то $G_M = (g_{ij})$ является матрицей скалярного произведения (то есть, для $u, v \in L$ скалярное произведение выражается в виде $(u, v) = U^TG_MV$ в силу того, что $(u, v) = \sum_{i, j = 1}^m g_{ij} u_i v_j$).
\end{rem}

\begin{thm*}
    $L$ --- ЕП, $M = \family{v_i}{i=1}{m}$ --- семейство векторов $L$. $M$ линейно независимо $\Leftrightarrow$ матрица Грама $G_M$ обратима.
\end{thm*}

\begin{proof}
    \begin{proofpart}{($\Rightarrow$)}
        Предположим, что $G_M$ необратима. Тогда уравнение $G_M X = \nil$ имеет ненулевое решение $X = (x_1, \ldots, x_m)^T \neq \nil$. Положим $u \coloneqq \sum_{i=1}^m x_i v_i \neq \nil$ (по линейной независимости) $\Rightarrow (u, u) = X^T G_M X = X^T \cdot \nil = \nil \Rightarrow u = \nil$. Полученное противоречие доказывает обратимость $G_M$.
    \end{proofpart}

    \begin{proofpart}{($\Leftarrow$)}
        Предположим, что $\sum_{i=1}^m x_i v_i = \nil$. Домножив скалярно обе части на вектор $v_j$, получим $\sum_{i=1}^m x_i g_{ij} = 0\ \forall j$. Отсюда $X^T G_M = \nil$. После домножения справа на $G_M^{-1}$ получаем $X^T = \nil \Rightarrow \forall i\ x_i = 0$. Отсюда $M$ линейно независимо.
    \end{proofpart}
\end{proof} % Матрица Грама набора векторов. Связь с линейной независимостью векторов
    \section{Теорема о существовании ОНБ в ЕП}

\begin{thm*}
    Пусть $L$ --- евклидово пространство. Тогда в нем существует ортонормированный базис $\family{v_i}{i=1}{n}$.
\end{thm*}

\begin{proof}
    Пусть $\family{u_i}{i=1}{n}$ --- \textbf{произвольный} базис $L$. Произведем индукцию по $n$.
    \smallskip
    
    \underline{$n = 1$}. $\{v_1 = \frac{u_1}{||u_1||}\}$ --- ОН-базис.
    
    \underline{$n-1 \rightarrow n$}. Пусть $U \coloneqq \gen{}{u_1, \ldots, u_{n-1}} \le L$ --- $(n-1)$-мерное подпространство $L$. По индукционному предположению в нем можно выбрать ОН-базис $\family{v_i}{i=1}{n-1}$.
    
    Пусть $v \coloneqq u_n - \sum_{i=1}^{n-1} (u_n, v_i) v_i$. Тогда $\forall k = 1..n-1$ имеем:
    $$(v, v_k) = \left(u_n - \sum_{i=1}^{n-1} (u_n, v_i)v_i, v_k\right) = (u_n, v_k) - \sum_{i=1}^{n-1} (u_n, v_i)(v_i, v_k) \stackrel{\text{ОН-ть}}{=} (u_n, v_k) - (u_n, v_k) \underbrace{(v_k, v_k)}_{=1} = 0$$
    Можем дополнить ОН-базис $U$ до ОН-базиса $V$, положив $v_n = \frac{v}{||v||}$.
\end{proof} % Теорема о существовании ОНБ в ЕП
    \section{Матрица перехода между ОНБ в ЕП}

\begin{thm*}
    $L$ --- ЕП. Пусть $B = \family{u_i}{i=1}{n}$ и $B' = \family{u'_i}{i=1}{n}$ --- ОНБ, причем $B \stackrel{C}{\leadsto} B'$. Тогда $C = (c_{ij})$ ортогональна.
\end{thm*}

\begin{proof}
    $$\delta_{ij} = (u'_i, u'_j) = \left(\sum_{k=1}^n c_{ki} u_k, \sum_{t=1}^n c_{tj} u_t\right) = \sum_{k,t=1}^n c_{ki} c_{tj} \underbrace{(u_k, u_t)}_{\delta_{kt}} = \sum_{s=1}^n \underbrace{c_{si}}_{C^T[i, s]} \cdot \underbrace{c_{sj}}_{C[s, j]} = C^T C[i, j]$$
    
    Так как $C^T C = E_n$, то $C$ --- ортогональная матрица.
\end{proof} % Матрица перехода между ОНБ в ЕП
    \section{Свойства ортогонального дополнения в ЕП}

\begin{defn}
    $L$ --- ЕП, $V \le \lsub{\mathbb{R}}{L}$. Подпространство $V^\perp = \left\{x \in L \mid \forall v \in V\ (v, x) = 0 \right\} \le L$ называется \textit{ортогональным дополнением} к $V$.
\end{defn}

\begin{thm*}
    $L$ --- ЕП. Пусть $V, V_1, V_2 \le \lsub{\mathbb{R}}{L}$. Справедливы следующие утверждения:
    \begin{enumerate}
        \item $V_1 \subset V_2 \Rightarrow V_2^\perp \subset V_1^\perp$
        \item $(V_1 + V_2)^\perp = V_1^\perp \cap V_2^\perp$
        \item $L = V \oplus V^\perp$
        \item $(V^\perp)^\perp = V$
        \item $(V_1 \cap V_2)^\perp = V_1^\perp + V_2^\perp$
    \end{enumerate}
\end{thm*}

\begin{proof}
    \begin{proofpart}
        Рассмотрим произвольный $v_2' \in V_2^\perp$. Имеем $\forall v_2 \in V_2\ (v_2, v_2') = 0 \Rightarrow \forall v_1 \in V_1\ (v_1, v_2') = 0$. При этом $\forall v_1 \in V_1\ \forall v_1' \in V_1^\perp\ (v_1, v_1') = 0$. Так как $V_1^\perp$ максимально по включению, $V_2^\perp \subset V_1^\perp$.
    \end{proofpart}

    \begin{proofpart}
        $V_1, V_2 \subset V_1 + V_2 \stackrel{\text{(а)}}{\Rightarrow} (V_1 + V_2)^\perp \subset V_1^\perp \cap V_2^\perp$. В то же время, пусть $u \in V_1^\perp \cap V_2^\perp$ и $v = v_1 + v_2 \in V_1 + V_2$, тогда $(u, v) = (u, v_1) + (u, v_2) = 0 + 0 = 0 \leadsto u \in (V_1 + V_2)^\perp$ и обратное включение доказано.
    \end{proofpart}

    \begin{proofpart}
        Считаем, что $V \neq \{\nil\}$. Выберем в $V$ ОН-базис $\family{e_i}{i=1}{k}$ и дополним его до базиса $L$: $\family{e_i}{i=1}{n}$. Докажем, что $V^\perp = \gen{}{v_{k+1}, \dots, v_n}$. Пусть $v = \sum_{i=1}^n \alpha_i e_i \in V^\perp$. Тогда $\forall j \le k$ имеем $0 = (e_j, v) = \sum_{i=1}^n \alpha_i (e_i, e_j) = \sum_{i=1}^n \alpha_i \delta_{ij} = \alpha_j$, откуда $v \in \gen{}{v_{k+1}, \dots, v_n}$. Обратное включение очевидно. Так, $V = V \oplus V^\perp$ (данная сумма прямая в силу того, что если $x \in X \cap X^\perp$, то $(x, x) = 0$ и $x = \nil$).
    \end{proofpart}

    \begin{proofpart}
        $L = V \oplus V^\perp$. Выделяя несобственные подпространства в $V$ и $V^\perp$ и применяя пункт (в), получаем $L = V^\perp \oplus V^{\perp\perp}$. При этом $\dim \lsub{\mathbb{R}}{V} = \dim \lsub{\mathbb{R}}{V^{\perp\perp}}$ и очевидно, что $V \subset V^{\perp\perp}$, откуда $V = V^{\perp\perp}$.
    \end{proofpart}

    \begin{proofpart}
        $(V_1 \cap V_2)^\perp \stackrel{\text{(б)}}{=} (V_1^\perp + V_2^\perp)^{\perp\perp} \stackrel{\text{(г)}}{=} V_1^\perp + V_2^\perp$.
    \end{proofpart}
\end{proof} % Свойства ортогонального дополнения в ЕП
    \section{Обобщенная теорема Пифагора}

\begin{thm*}
    $L$ --- ЕП. Пусть $\family{v_i}{i=1}{m}$ --- ортогональный набор векторов $L$. Тогда $\norm{\sum_{i=1}^m v_i}^2 = \sum_{i=1}^m \norm{v_i}^2$.
\end{thm*}

\begin{proof}
    $${\norm{\sum_{i=1}^m v_i}}^2 = \left(\sum_{i=1}^m v_i,\ \sum_{j=1}^m v_j\right) = \sum_{i,j=1}^m (v_i, v_j) = \sum_{i=1}^m (v_i, v_i) = \sum_{i=1}^m \norm{v_i}^2$$
\end{proof} % Обобщенная теорема Пифагора
    \section{Линейные функционалы на линейном пространстве. Дуальный базис}

\begin{defn}
    $K$ --- поле, $\lsub{K}{V}$ --- линейное пространство. $\lsub{K}{V^*} \coloneqq \lsub{K}{\hom(V, K)}$ называется \textit{пространством функционалов (также дуальным, двойственным, сопряженным пространством)} на $V$.
\end{defn}

\begin{thm}
    $\dim \lsub{K}{V} < \infty \Rightarrow \dim \lsub{K}{V^*} = \dim \lsub{K}{V}$.
\end{thm}

\begin{proof}
    $\dim \lsub{K}{V^*} = \dim \lsub{K}{\hom(V, K)} = \dim \lsub{K}{V} \cdot \dim \lsub{K}{K} = \dim \lsub{K}{V}$.
\end{proof}

\begin{defn}
    Пусть $B = \family{u_i}{i=1}{n}$ --- базис $\lsub{K}{V}$. \textit{Дуальным} к $B$ называется базис $B^* = \family{\varphi_j}{j=1}{n}$ пространства $V^*$ такой, что $\varphi_j(u_i) = \delta_{ij}$.
\end{defn}

\begin{thm}
    $B^*$ --- действительно базис $\lsub{K}{V^*}$.
\end{thm}

\begin{proof}
    Пусть $\varphi \in V^*$.
    
    \begin{align*}
        \varphi(v) &= \varphi\left(\sum_{i=1}^n \alpha_i u_i\right) = \sum_{i=1}^n \alpha_i \varphi(u_i) = \sum_{i=1}^n \alpha_i \sum_{j=1}^n \varphi(u_j) \varphi_j(u_i) \\
        &= \sum_{j=1}^n \varphi(u_j) \sum_{i=1}^n \alpha_i \varphi_j(u_i) = \sum_{j=1}^n \varphi(u_j) \varphi_j \left(\sum_{i=1}^n \alpha_i u_i \right) = \sum_{j=1}^n \varphi(u_j) \varphi_j(v)
    \end{align*}
    
    Отсюда $\varphi \in \gen{K}{B^*}$. Так как $\#B^* = \dim \lsub{K}{V^*}$, то $B^*$ --- базис $V^*$.
\end{proof} % Линейные функционалы на линейном пространстве. Дуальный базис
    \section{Дуальное отображение. Свойства операции перехода к дуальному отображению. Второе пространство функционалов, его связь с исходным пространством}

\begin{defn}
    Пусть $f \colon \lsub{K}{U} \to \lsub{K}{V}$ --- $K$-линейное отображение. Построим $f^* \colon V^* \to U^*$ следующим образом:
    
    \begin{diagram}
        U &                       & \rTo^f &            & V \\
          & \rdDashto_{f^*(\phi)} &        & \ldTo_\phi &   \\
          &                       & K      &            &   
    \end{diagram}

    То есть, для $\phi \in V^*$ положим $f^*(\phi) \coloneqq \phi \circ f \in U^*$. Очевидно, что оно $K$-линейно. $f^*$ называется \textit{дуальным к $f$} отображением.
\end{defn}

\begin{rem}
    Дуальное отображение обладает следующими свойствами:
    \begin{enumerate}
        \item $(f_1 + f_2)^*(\phi) = \phi \circ (f_1 + f_2) = \phi \circ f_1 + \phi \circ f_2 = f_1^*(\phi) + f_2^*(\phi)$
        \item $(gf)^*(\phi) = \phi (gf) = (\phi g)f = f^*(\phi g) = f^* (g^*(\phi)) = (f^* \circ g^*)(\phi)$
    \end{enumerate}
\end{rem}

\begin{defn}
    $\lsub{K}{V}$ --- линейное пространство. Построим отображение $\delta_V \colon V \to V^{**}$ так, чтобы $\forall \phi \in V^*\ (\delta_V(v))(\phi) = \phi(v)$ ($\delta_V(v) = \bullet(v)$).
\end{defn}

\begin{thm*}
    $\lsub{K}{U}$, $\lsub{K}{V}$ --- линейные пространства.
    \begin{enumerate}
        \item $\delta_V$ --- изоморфизм.
        \item Пусть $f \colon U \to V$ --- $K$-линейное отображение, тогда $\delta_V \circ f = f^{**} \circ \delta_U$.
    \end{enumerate}
\end{thm*}

\begin{rem}
    $\delta$ можно рассматривать как функтор в терминах теории категорий (следующая диаграмма коммутирует):
    
    \begin{diagram}
        U                    & \rTo^f        & V                    \\
        \dTo^{\delta_U}_\sim &               & \dTo_{\delta_V}^\sim \\
        U^{**}               & \rTo^{f^{**}} & V^{**}
    \end{diagram}
\end{rem}

\begin{proof}
    \begin{proofpart}
        Докажем, что $\delta_V$ мономорфно. Пусть $v \in V \setminus \nilset$ и $\delta_V(v) = 0$. Тогда $\forall \phi \in V^*\ \phi(v) = 0$. Построим базис $\{v, \dots\}$ и двойственный к нему $\{\phi_1, \dots\}$. Отсюда по определению $\phi_1(v) = 1$. Полученное противоречие доказывает мономорфность $\delta_V$. Так как $\delta_V$ --- мономорфизм пространств одинаковой размерности, то $\delta_V$ --- изоморфизм.
    \end{proofpart}

    \begin{proofpart}
        \begin{align*}
            ((\delta_V \circ f)(u))(\phi) &= (\delta_V(f(u)))(\phi) = \phi(f(u)) = (\phi \circ f)(u) = (f^*(\phi))(u) = (\delta_U(u))(f^*(\phi)) \\
            &= (\delta_U(u) \circ f^*)(\phi) = (f^{**}(\delta_U(u)))(\phi) = (f^{**} \circ \delta_U(u))(\phi)
        \end{align*}
        
        Так, $\delta_V \circ f = f^{**} \circ \delta_U$.
    \end{proofpart}
\end{proof} % Дуальное отображение. Свойства операции перехода к дуальному отображению. Второе пространство функционалов, его связь с исходным пространством
    \section{Линейные функционалы на ЕП}

\begin{defn}
    $L$ --- ЕП. Пусть $v \in L$. Отображение $\varphi_v \colon L \to \mathbb{R}$, т.ч. $\varphi_v(u) = (u, v)$ назовем \textit{функционалом скалярного умножения} на $v$.
\end{defn}

\begin{thm*}
    $L$ --- ЕП. Рассмотрим $f \colon L \to L^*$, т.ч. $f(v) = \varphi_v$. Справедливо следующее:
    \begin{enumerate}
        \item $f$ --- изоморфизм.
        \item Если $B$ --- ОН-базис $\lsub{\mathbb{R}}{V}$, то $f(B)$ --- дуальный к нему в $\lsub{\mathbb{R}}{L^*}$.
    \end{enumerate}
\end{thm*}

\begin{proof}
    \begin{proofpart}
        Отображение $f$ $K$-линейно за счет линейности скалярного произведения по первому аргументу. Докажем мономорфность $f$. Пусть $f(v) = 0$ для $v \in L$. Тогда $(\bullet, v) \equiv 0 \leadsto (v, v) = 0 \leadsto v = \nil$. Так как $\dim \lsub{\mathbb{R}}{L^*} = \dim \lsub{\mathbb{R}}{L}$, то $f$ --- мономорфизм $K$-линейных пространств одинаковой размерности, и, как следствие, является изоморфизмом.
    \end{proofpart}
    \begin{proofpart}
        Пусть $B = \family{e_i}{i=1}{n}$. Тогда $f(B) = \family{\varphi_j = (\bullet, e_j)}{j=1}{n}$. В частности, $\varphi_j(e_i) = (e_i, e_j) = \delta_{ij}$, откуда $f(B)$ --- дуальный к $B$ базис $L^*$.
    \end{proofpart}
\end{proof}

\begin{cor*}
    $L$ --- ЕП. Пусть $\varphi \colon \lsub{\mathbb{R}}{L} \to \mathbb{R} \in L^*$, тогда $\exists! v \in L \colon \varphi(u) = (u, v)\ \forall u \in L$.
\end{cor*} % Линейные функционалы на ЕП
    \section{Проекторы в линейных пространствах}

\begin{defn}
    $\lsub{K}{V}$ --- линейное пространство. Эндоморфизм $p \in \End \lsub{K}{V}$ называется \textit{проектором (оператором проектирования)}, если $p^2 = p$.
\end{defn}

\begin{exmpl}
    $$\begin{array}{ c c c c c }
        \lsub{K}{V} & = & U     & \oplus & U'    \\
        \inup       &   & \inup &        & \inup \\
        v           & = & u     & +      & u'
    \end{array}$$
    $p_U(v) \coloneqq u$ --- проектор, называемый \textit{проектором на $U$ параллельно $U'$}.
\end{exmpl}

\begin{thm*}
    Пусть $p \in \End \lsub{K}{V}$ --- проектор. Тогда:
    \begin{enumerate}
        \item $V = \ker p \oplus \im p$.
        \item $p$ совпадает с проектором на $\im p$ параллельно $\ker p$.
    \end{enumerate}
\end{thm*}

\begin{proof}
    \begin{proofpart}
        Пусть $v \in V$. Тогда $p(v) = p^2(v)$. Положим $y \coloneqq v - p(v)$. Ясно, что $p(y) = \nil$. Так:
        \begin{equation}\label{09-10:star}\tag{$*$}
            v = p(v) + y, \qquad p(v) \in \im p,\ y \in \ker p
        \end{equation}
        откуда $V = \im p + \ker p$. Пусть $x \in \ker p \cap \im p$, тогда $\exists z \in V \colon x = p(z)$ и $p(x) = \nil$. Отсюда $\nil = p(x) = p^2(z) = p(z) = x$. Следовательно, $\ker p \cap \im p = \nilset$ и соответствующая сумма прямая.
    \end{proofpart}
    \begin{proofpart}
        Явным образом следует из \eqref{09-10:star}.
    \end{proofpart}
\end{proof} % Проекторы в линейных пространствах
    \section{Ортогональное проектирование на подпространство}

\begin{defn}
    $L$ --- ЕП, $V \le \lsub{\mathbb{R}}{L}$, $L = V \oplus V^\perp$. Проектор на $V$ параллельно $V^\perp$ называется \textit{оператором ортогонального проектирования на $V$}. В этом случае $v \in L$ единственным образом представляется в виде $v = v_{\text{pr}} + v^\perp$, где $v_{\text{pr}}$ называется \textit{ортогональной проекцией $v$ на $V$}, а $v^\perp$ --- \textit{ортогональной составляющей $v$}.
\end{defn}

\begin{defn}
    $L$ --- ЕП, $V \le \lsub{\mathbb{R}}{L}$, $v \in L$. Величина $\rho(v, V) \coloneqq \inf \left\{ \rho(v, x) \mid x \in V \right\}$ называется \textit{расстоянием между $v$ и $V$}.
\end{defn}

\begin{thm*}
    $L$ --- ЕП, $V \le \lsub{\mathbb{R}}{L}$, $v \in L$. Пусть $p$ --- оператор ортогонального проектирования на $V$, тогда $\rho(v, V) = \norm{v^\perp} = \norm{v - p(v)}$.
\end{thm*}

\begin{proof}
    Положим $v_{\text{pr}} \coloneqq p(v) \in V$. Очевидно, что $\rho(v, V) \le \norm{v^\perp}$ (т.к. $\rho(v, v_\text{pr}) = \norm{v - v_\text{pr}} = \norm{v^\perp}$). В то же время, полагая $u \in V$, имеем $\rho^2(v, u) = \norm{v - u}^2 = \norm{v^\perp + (v_\text{pr} - u)}^2 = \norm{v^\perp}^2 + \norm{v_\text{pr} - u}^2 \ge \norm{v^\perp}^2$ (в силу ортогональности векторов суммы), откуда $\rho(v, V) \ge \norm{v^\perp}$. Так, $\rho(v, V) = \norm{v^\perp}$.
\end{proof} % Ортогональное проектирование на подпространство
    
    % "Петушиные билеты." (c) Г. Енгалыч
    %\input{content/09-12} % Приложения процесса ортогонализации Грама-Шмидта. Многочлены Лежандра и Чебышева
    %\input{content/09-13} % (*) Понятие предгильбертова пространства. Контрпример
    \setcounter{section}{13}
    
    \section{Определение унитарного пространства. Неравенства Коши и треугольника в УП}

\begin{defn}
    \textit{Унитарным пространством} называется пара $(\lsub{\mathbb{C}}{L}, F)$, где $\lsub{\mathbb{C}}{L}$ --- \textbf{конечномерное} $\mathbb{C}$-линейное пространство, а $F \colon L^2 \to \mathbb{C}$ --- положительно определенная \textbf{эрмитова} форма (\textit{скалярное произведение}).
    Аналогично ЕП вводятся понятия \textit{нормы} вектора и \textit{расстояния} между двумя векторами.
\end{defn}

\begin{thm*}
    $L$ --- УП. Пусть $u, v \in L$. Тогда:
    \begin{enumerate}
        \item $|(u, v)| \le ||u|| \cdot ||v||$
        \item $||u + v|| \le ||u|| + ||v||$
        \item $\rho(u, v) \le \rho(u, w) + \rho(w, v)\ \forall w \in L$
    \end{enumerate}
\end{thm*}

\begin{proof}
    \begin{proofpart}
        Пусть $t \in \mathbb{C}$. Считаем, что $v \neq \nil$. Тогда $0 \le (u + tv, u + tv) = \norm{u}^2 + t(v, u) + \bar{t}(u, v) + {|t|}^2\norm{v}^2$. Положим $t = \lambda(u, v),\ \lambda \in \mathbb{R}$. Тогда $0 \le \norm{u}^2 + 2\lambda{|(u, v)|}^2 + \lambda^2 {|(u, v)|}^2 \norm{v}^2$. В таком случае $D = 4{|(u, v)|}^4 - 4{|(u, v)|}^2\norm{u}^2\norm{v}^2 \le 0$, откуда $|(u, v)| \le \norm{u} \cdot \norm{v}$
    \end{proofpart}
    
    \begin{proofpart}
        Аналогично доказательству для ЕП.
    \end{proofpart}
    
    \begin{proofpart}
        Аналогично доказательству для ЕП.
    \end{proofpart}
\end{proof} % Определение унитарного пространства. Неравенства Коши и треугольника в УП
    \section{Существование ОНБ в УП. Матрица перехода между ОНБ в УП}

\begin{thm}
    Пусть $\lsub{\mathbb{C}}{L}$ --- УП, тогда существует ОН-базис $\family{v_i}{i=1}{n}$ в $\lsub{\mathbb{C}}{L}$.
\end{thm}

\begin{proof}
    Аналогично случаю ЕП (процесс Грама-Шмидта).
\end{proof}

\begin{defn}
    Матрица $C \in M_n(\mathbb{C})$ называется \textit{унитарной}, если $C^{-1} = \bar{C}^T$ ($\eqqcolon C^*$ --- \textit{эрмитово сопряженная матрица}).
\end{defn}

\begin{thm*}
    Пусть $\lsub{\mathbb{C}}{L}$ --- УП, $B$ и $B'$ --- ОНБ в $\lsub{\mathbb{C}}{L}$, причем $B \stackrel{C}{\leadsto} B'$. Тогда $C$ --- унитарная матрица.
\end{thm*}

\begin{proof}
    Пусть $B = \family{u_i}{i=1}{n}$ и $B' = \family{u'_i}{i=1}{n}$, $C = (c_{ij})$. Тогда:
    $$\delta_{ij} = (u'_i, u'_j) = \left(\sum_{k=1}^n c_{ki} u_i,\ \sum_{t=1}^n c_{tj} u_j \right) = \sum_{k,t=1}^n c_{ki} \overline{c_{tj}} \underbrace{(u_i, u_j)}_{\delta_{ij}} = \sum_{s=1}^n \underbrace{c_{si}}_{C[s, i]} \cdot \underbrace{\overline{c_{sj}}}_{\bar{C}^T[j, s]} = C^*C[j, i]$$
    Так, $C^*C = E_n$, откуда $C$ унитарна.
\end{proof} % Существование ОНБ в УП. Матрица перехода между ОНБ в УП
    \section{Линейные функционалы на УП}

\begin{defn}
    Пусть $\lsub{\mathbb{C}}{L}$ --- УП, $v \in L$. Отображение $\varphi_v \colon L \to \mathbb{C}$, т.ч. для $u \in L$ $\varphi_v(u) = (u, v)$ назовем \textit{функционалом скалярного умножения на $v$}. Ясно, что $\varphi_v \in L^*$.
\end{defn}

\begin{defn}
    $\lsub{\mathbb{C}}{L}$ --- $C$-линейное пространство. На $L$ введем прежнее сложение и новое умножение на скаляры $\mu$, т.ч. $\mu(\alpha, v) \coloneqq \bar{\alpha} v$. Ясно, что $(L, +, \mu) \eqqcolon \bar{L}$ --- $\mathbb{C}$-линейное пространство.
\end{defn}

\begin{thm*}
    Пусть $\lsub{\mathbb{C}}{L}$ --- УП. Рассмотрим отображение $f \colon \bar{L} \to L^*$, т.ч. $f(v) = \varphi_v$. Справедливы следующие утверждения:
    \begin{enumerate}
        \item $f$ --- изоморфизм.
        \item Если $B$ --- ОН-базис в $\lsub{\mathbb{C}}{L}$, то $f(B)$ --- дуальный к нему базис в $L^*$.
    \end{enumerate}
\end{thm*}

\begin{proof}
    \begin{proofpart}
        $$f(u + v) = (\bullet, u + v) = (\bullet, u) + (\bullet, v) = f(u) + f(v)$$
        $$f(\alpha v) = (\bullet, \bar{\alpha}v) = \bar{\bar{\alpha}}(\bullet, v) = \alpha(\bullet, v) = \alpha f(v)$$
        Отсюда $f$ --- $\mathbb{C}$-линейное отображение. Предположим, что $f(v) = 0$, то есть $\forall u \in L\ (u, v) = 0$, тогда $(v, v) = 0 \leadsto v = \nil \leadsto \ker f = \nilset$ и $f$ --- мономорфизм. Но $\dim \lsub{\mathbb{C}}{\bar{L}} = \dim \lsub{\mathbb{C}}{L^*}$, и $f$ --- изоморфизм.
    \end{proofpart}
    \begin{proofpart}
        Пусть $B = \family{e_i}{i=1}{n}$. Тогда $f(B) = \family{\varphi_{e_j} = (\bullet, e_j)}{j=1}{n}$. Но $\varphi_{e_j}(e_i) = (e_i, e_j) = \delta_{ij}$, откуда $f(B)$ --- дуальный к $B$ базис $L^*$.
    \end{proofpart}
\end{proof} % Линейные функционалы на УП
    \section{Сопряженные операторы в ЕП, их свойства}

\begin{defn}
    Пусть $\lsub{\mathbb{R}}{L}$ --- ЕП, $a \in \End \lsub{\mathbb{R}}{L}$. Положим $f \colon L \to L^*$ --- изоморфизм, т.ч. $v \mapsto \varphi_v = (\bullet, v)$; также положим $a^* \colon L^* \to L^*$ --- дуальное к $a$ отображение ($a^*(\varphi) = \varphi \circ a$). Оператор $\hat{a} \colon f^{-1} \circ a^* \circ f \colon L \to L$ назовем \textit{сопряженным} к $a$.
\end{defn}

\begin{rem}
    Определение наглядно описывается следующей диаграммой:
    \begin{diagram}
        L      & \rTo^{\hat{a}} & L             \\
        \dTo^f &                & \uTo_{f^{-1}} \\
        L^*    & \rTo^{a^*}     & L^*
    \end{diagram}
\end{rem}

\begin{thm}
    $L$ --- ЕП. Пусть $a, b \in \End \lsub{\mathbb{R}}{L}$. Тогда:
    \begin{enumerate}
        \item $\widehat{a + b} = \hat{a} + \hat{b}$.
        \item $\forall \alpha \in \mathbb{R}\ \widehat{\alpha a} = \alpha \hat{a}$.
        \item $\widehat{ab} = \hat{b} \hat{a}$.
        \item Если $a$ обратим, то $\hat{a}$ обратим и $\hat{a}^{-1} = \widehat{a^{-1}}$.
    \end{enumerate}
\end{thm}

\begin{proof}
    \begin{proofpart}
        $$\widehat{a + b} = f \circ (a + b)^* \circ f^{-1} = f \circ (a^* + b^*) \circ f^{-1} = f \circ a^* \circ f^{-1} + f \circ b^* \circ f^{-1} = \hat{a} + \hat{b}$$
    \end{proofpart}

    \begin{proofpart}
        $$\widehat{\alpha a} = f \circ (\alpha a)^* \circ f^{-1} = \alpha f \circ a^* \circ f^{-1} = \alpha \hat{a}$$
    \end{proofpart}

    \begin{proofpart}
        $$\widehat{ab} = f \circ (ab)^* \circ f^{-1} = f \circ (b^* a^*) \circ f^{-1} = (f \circ b^* \circ f^{-1}) \circ (f \circ a^* \circ f^{-1}) = \hat{b} \hat{a}$$
    \end{proofpart}

    \begin{proofpart}
        Ясно, что $(\id_V)^* = \bullet \circ \id_V = \bullet = \id_{V^*}$. Тогда $\widehat{\id_V} = \id_V$, откуда $\id_V = \widehat{a \circ a^{-1}} = \widehat{a^{-1}} \circ \hat{a}$.
    \end{proofpart}
\end{proof}

\begin{thm}
    Пусть $L$ --- ЕП, $a \in \End \lsub{\mathbb{R}}{L}$. Тогда $\forall u, v \in L\ (a(u), v) = (u, \hat{a}(v))$.
\end{thm}

\begin{proof}
    Имеем $f \circ \hat{a} = a^* \circ f$, откуда $\forall v \in L\ f(\hat{a}(v)) = a^*(f(v))$. Но $f(\hat{a}(v)) = (\bullet, \hat{a}(v))$, $a^*(f(v)) = (a(\bullet), v)$, следовательно $\forall u \in L\ (u, \hat{a}(v)) = (a(u), v)$.
\end{proof}

\begin{cor}
    Справедливы следующие утверждения:
    \begin{enumerate}
        \item $\hat{\hat{a}} = a$.
        \item Из обратимости $\hat{a}$ следует обратимость $a$.
    \end{enumerate}
\end{cor}

\begin{proof}
    \begin{proofpart}
        $(a(u), v) = (u, \hat{a}(v)) = (\hat{\hat{a}}(u), v) \leadsto \forall v \in L\ (a(u) - \hat{\hat{a}}(u), v) = 0 \leadsto a(u) = \hat{\hat{a}}(u)\ \forall u \in L$.
    \end{proofpart}

    \begin{proofpart}
        Из обратимости $\hat{a}$ следует обратимость $\hat{\hat{a}} = a$.
    \end{proofpart}
\end{proof}

\begin{cor}
    $L$ --- ЕП, $a \in \End \lsub{\mathbb{R}}{L}$. Пусть $V \le \lsub{\mathbb{R}}{L}$, тогда $V$ $a$-инвариантно в том и только в том случае, когда $V^\perp$ $\hat{a}$-инвариантно.
\end{cor}
\begin{proof}
    \begin{proofpart}{($\Rightarrow$)}
        Пусть $x \in V, y \in V^\perp$. Тогда $0 = (a(x), y) = (x, \hat{a}(y))$, откуда $\hat{a}(y) \in V^\perp$.
    \end{proofpart}

    \begin{proofpart}{($\Leftarrow$)}
        Следует из применения ($\Rightarrow$) к $\hat{a}$ и $V^\perp$, так как $\hat{\hat{a}} = a$ и $V^{\perp\perp} = V$.
    \end{proofpart}
\end{proof}

\begin{cor}
    Справедливы следующие утверждения:
    \begin{enumerate}
        \item $\ker \hat{a} = (\im a)^\perp$.
        \item $\im \hat{a} = (\ker a)^\perp$.
    \end{enumerate}
\end{cor}

\begin{proof}
    \begin{proofpart}
        $$x \in \ker \hat{a} \Leftrightarrow \forall u \in L\ (u, \hat{a}(x)) = 0 \Leftrightarrow x \in (\im a)^\perp$$
    \end{proofpart}

    \begin{proofpart}
        $$\ker a = \ker \hat{\hat{a}} = (\im \hat{a})^\perp \Rightarrow \im \hat{a} = (\ker a)^\perp$$
    \end{proofpart}
\end{proof}

\begin{cor}
    Справедливы следующие утверждения:
    \begin{enumerate}
        \item $\dim \lsub{\mathbb{R}}{\ker a} = \dim \lsub{\mathbb{R}}{\ker \hat{a}}$.
        \item $\dim \lsub{\mathbb{R}}{\im a} = \dim \lsub{\mathbb{R}}{\im \hat{a}}$.
    \end{enumerate}
\end{cor}

\begin{proof}
    Из предыдущего следствия имеем: $$L = \ker \hat{a} \oplus \im a = \ker a \oplus \im \hat{a}$$
    откуда $$\begin{cases}
        \dim \lsub{\mathbb{R}}{\ker \hat{a}} + \dim \lsub{\mathbb{R}}{\im a} = \dim \lsub{\mathbb{R}}{L} \\
        \dim \lsub{\mathbb{R}}{\ker a} + \dim \lsub{\mathbb{R}}{\im \hat{a}} = \dim \lsub{\mathbb{R}}{L} \\
        \dim \lsub{\mathbb{R}}{\ker a} + \dim \lsub{\mathbb{R}}{\im a} = \dim \lsub{\mathbb{R}}{L}
    \end{cases}$$
    и соответствующие равенства очевидны.
\end{proof}

\begin{thm}
    $L$ --- ЕП, $a \in \End \lsub{\mathbb{R}}{L}$. Пусть $B = \family{e_i}{i=1}{n}$ --- ОН-базис $\lsub{\mathbb{R}}{L}$ и пусть $[a]_B = A = (a_{ij})$. Тогда $[\hat{a}]_B = A^T$.
\end{thm}

\begin{proof}
    Пусть $[\hat{a}]_B = D = (d_{ij})$. Имеем $\forall j\ \hat{a}(e_j) = \sum_{i=1}^n d_{ij} e_i$ и $\forall j\ a(e_j) = \sum_{i=1}^n a_{ij} e_i$. Но $d_{ij} = (\hat{a}(e_j), e_i) = (e_j, a(e_i)) = a_{ji}$, откуда $D = A^T$.
\end{proof}

\begin{cor*}
    В предыдущих обозначениях:
    \begin{enumerate}
        \item $\chi_a = \chi_{\hat{a}}$.
        \item $\spec(\hat{a}) = \spec(a)$.
    \end{enumerate}
\end{cor*} % Сопряженные операторы в ЕП, их свойства
    \section{Сопряженные операторы в УП}

\begin{defn}
    $\lsub{\mathbb{C}}{L}$ --- УП, $a \in \End \lsub{\mathbb{C}}{L}$. \textit{Сопряженным к $a$} назовем оператор $\hat{a}$, т.ч. $\forall u, v \in L\ (a(u), v) = (u, \hat{a}(v))$.
\end{defn}

\begin{rem}
    Пусть $v \in L$. Рассмотрим отображение $\psi_v \colon L \to \mathbb{C}$, т.ч. $\psi_v(u) = (a(u), v)$. Ясно, что $\psi_v \in L^*$. Тогда $\exists! \tilde{v} \colon \psi_v = \varphi_{\tilde{v}} = (\bullet, \tilde{v})$. Положим $\hat{a}(v) \coloneqq \tilde{v}$. Таким образом мы предъявим явную конструкцию сопряженного оператора.
    
    Предположим, что существует оператор $\tilde{a}$, т.ч. $(a(u), v) = (u, \tilde{a}(v))\ \forall u, v \in L$. Тогда $\forall u, v \in L$ имеем $(u, \tilde{a}(v)) = (u, \hat{a}(v)) \leadsto (u, \tilde{a}(v) - \hat{a}(v)) = 0 \leadsto \tilde{a}(v) = \hat{a}(v)\ \forall v$. Таким образом, сопряженный оператор существует и определен однозначно.
\end{rem}

\begin{thm}
    $L$ --- УП. Пусть $a, b \in \End \lsub{\mathbb{C}}{L}$. Тогда:
    \begin{enumerate}
        \item $\widehat{a + b} = \hat{a} + \hat{b}$.
        \item $\forall \alpha \in \mathbb{R}\ \widehat{\alpha a} = \bar{\alpha} \hat{a}$.
        \item $\widehat{ab} = \hat{b} \hat{a}$.
        \item $\hat{\hat{a}} = a$.
        \item $a$ обратим $\Leftrightarrow$ $\hat{a}$ обратим.
    \end{enumerate}
\end{thm}

\begin{proof}
    \begin{proofpart}
        $$((a + b)(u), v) = (a(u), v) + (b(u), v) = (u, \hat{a}(v)) + (u, \hat{b}(v)) = (u, (\hat{a} + \hat{b})(v)) \Rightarrow \widehat{a + b} = \hat{a} + \hat{b}$$
    \end{proofpart}

    \begin{proofpart}
        $$((\alpha a)(u), v) = \alpha(a(u), v) = \alpha(u, \hat{a}(v)) = (u, \bar{\alpha}\hat{a}(v)) \Rightarrow \widehat{\alpha a} = \bar{\alpha} \hat{a}$$
    \end{proofpart}

    \begin{proofpart}
        $$((ab)(u), v) = (a(b(u)), v) = (b(u), \hat{a}(v)) =(u, \hat{b}(\hat{a}(v))) = (u, \hat{b} \hat{a}(v)) \Rightarrow \widehat{ab} = \hat{b} \hat{a}$$
    \end{proofpart}

    \begin{proofpart}
        $$(a(u), v) = (u, \hat{a}(v)) = \overline{(\hat{a}(v), u)} = \overline{(v, \hat{\hat{a}}(u))} = (\hat{\hat{a}}(u), v) \Rightarrow a = \hat{\hat{a}}$$
    \end{proofpart}

    \begin{proofpart}
        Пусть $a$ обратим. Тогда:
        $$\id_L = \widehat{\id_L} = \widehat{a \circ a^{-1}} = \widehat{a^{-1}} \circ \hat{a}$$
        откуда $\hat{a}$ обратим. Обратное следует из предыдущего пункта.
    \end{proofpart}
\end{proof}

\begin{cor}
    Полностью аналогично Следствию 2 из предыдущего вопроса.
\end{cor}

\begin{cor}
    Полностью аналогично Следствию 3 из предыдущего вопроса.
\end{cor}

\begin{cor}
    Полностью аналогично Следствию 4 из предыдущего вопроса.
\end{cor}

\begin{thm}
    $L$ --- УП, $a \in \End \lsub{\mathbb{C}}{L}$. Пусть $B = \family{e_i}{i=1}{n}$ --- ОН-базис $\lsub{\mathbb{C}}{L}$ и пусть $[a]_B = A = (a_{ij})$. Тогда $[\hat{a}]_B = A^*$.
\end{thm}

\begin{proof}
    Пусть $[\hat{a}]_B = B = (b_{ij})$. Тогда $a(e_i) = \sum_{j=1}^{n} a_{ji} e_j$, $\hat{a}(e_i) = \sum_{j=1} b_{ji} e_j$. В то же время $a_{ji} = (a(e_i), e_j) = (e_i, \hat{a}(e_j)) = \overline{(\hat{a}(e_j), e_i)} = \overline{b_{ij}}$, откуда $B = A^*$.
\end{proof}

\begin{cor*}
    В предыдущих обозначениях:
    \begin{enumerate}
        \item $\chi_{\hat{a}} = \overline{\chi_a}$.
        \item $\spec(a) = \family{\lambda_i}{i=1}{n} \Rightarrow \spec(\hat{a}) = \family{\bar{\lambda_i}}{i=1}{n}$.
    \end{enumerate}
\end{cor*}

\begin{proof}
    \begin{proofpart}
        $$\chi_{\hat{a}} = |A^* - \lambda E| = |\bar{A}^T - \lambda E| = |\overline{(A - \lambda E)}^T| = |\overline{A - \lambda E}| = \overline{\chi_a}$$
    \end{proofpart}

    \begin{proofpart}
        Следует из первого пункта.
    \end{proofpart}
\end{proof} % Сопряженные операторы в УП
    \section{Нормальные операторы в УП. Свойства собственных векторов нормального оператора}

\begin{defn}
    Пусть $L$ --- ЕП или УП, $a \in \End \lsub{K}{L}$. $a$ называется \textit{нормальным}, если он коммутирует с $\hat{a}$: $\hat{a} a = a \hat{a}$. В свою очередь матрица $A \in M_n(\mathbb{C})$ называется \textit{нормальной}, если $AA^* = A^*A$.
\end{defn}

\begin{thm*}
    $\lsub{\mathbb{C}}{L}$ --- УП. Пусть $a \in \End \lsub{\mathbb{C}}{L}$. Тогда $a$ нормален $\Leftrightarrow$ существует ОН-базис $B$ в $\lsub{\mathbb{C}}{L}$, т.ч. $[a]_B$ диагональна.
\end{thm*}

\begin{proof}
    \begin{proofpart}{($\Rightarrow$)}
        Пусть $\dim \lsub{\mathbb{C}}{L} = n$. Проведем доказательство при помощи ММИ.
        
        \underline{$n = 1$}. Тривиальный случай.
        
        \underline{$n \to n + 1$}. Пусть $\lambda \in \spec(a)$. Считаем, что $\mathcal{L}_a(\lambda) \lneqq L$ (иначе можно выбрать произвольный ОН-базис). Из $a$-инвариантности $\mathcal{L}_a(\lambda)$ следует $\hat{a}$-инвариантность $\mathcal{L}_a(\lambda)^\perp$. $\mathcal{L}_a(\lambda)$ также $\hat{a}$-инвариантно (так как $\mathcal{L}_a(\lambda) \ni a(\hat{a}(v)) = \hat{a}(a(v)) = \lambda \hat{a}(v)$, откуда $\hat{a}(v) \in \mathcal{L}_a(\lambda)$), и $\mathcal{L}_a(\lambda)^\perp$ $a$-инвариантно.
        
        Рассмотрим $a\vert_{\mathcal{L}_a(\lambda)^\perp}$ и $\hat{a}\vert_{\mathcal{L}_a(\lambda)^\perp}$. Они коммутируют. Тогда по ИП существует ОН-базис $B_0$ в $\mathcal{L}_a(\lambda)^\perp$, состоящий из собственных векторов. Пусть $B_1$ --- произвольный ОН-базис в $\mathcal{L}_a(\lambda)$. Тогда $B \coloneqq B_0 \cup B_1$ --- ОН-базис $L$, состоящий из собственных векторов, т.е. $[a]_B$ диагональна.
    \end{proofpart}

    \begin{proofpart}
        Пусть $B$ --- ОН-базис в $\lsub{\mathbb{C}}{L}$, т.ч. $[a]_B = \diag(\lambda_1, \dots, \lambda_n) \eqqcolon A$. Тогда $[\hat{a}]_B = A^* = \diag(\bar{\lambda_1}, \dots, \bar{\lambda_n})$. Ясно, что $A$ и $A^*$ коммутируют, тогда коммутируют и соответствующие операторы, откуда $a$ нормален.
    \end{proofpart}
\end{proof}

\begin{cor}
    $A \in M_n(\mathbb{C})$ нормальна $\Leftrightarrow$ существует унитарная $U \in M_n(\mathbb{C})$ такая, что $U^{-1} AU$ диагональна.
\end{cor}

\begin{proof}
    Следует из унитарности матрицы перехода между ОН-базисами.
\end{proof}

\begin{cor}
    Пусть $L$ --- УП, $a \in \End \lsub{\mathbb{C}}{L}$ --- нормальный оператор. Тогда $\forall \lambda \in \spec(a)\ \mathcal{L}_a(\lambda) = \mathcal{L}_{\hat{a}}(\bar{\lambda})$.
\end{cor}

\begin{proof}
    Пусть $B = \family{e_i}{i=1}{n}$ --- ОН-базис $L$, т.ч. $[a]_B = \diag(\lambda_1, \dots, \lambda_n)$. Тогда $[\hat{a}]_B = [a]_B^* = \diag(\bar{\lambda_1}, \dots, \bar{\lambda_n})$. В таком случае $\forall i\ a(e_i) = \lambda_i e_i$ и $\hat{a}(e_i) = \bar{\lambda_i} e_i$. Так как $e_i$, принадлежащие $\lambda_i \in \spec(a)$, составляют базисы $\mathcal{L}_a(\lambda)$ и $\mathcal{L}_{\hat{a}}(\bar{\lambda})$, то $\mathcal{L}_a(\lambda) = \mathcal{L}_{\hat{a}}(\bar{\lambda})$.
\end{proof}

\begin{cor}
    $L$ --- УП, $a$ --- нормальный оператор. Пусть $\lambda, \mu \in \spec(a)$ --- \textbf{различные} собственные числа. Тогда $\mathcal{L}_a(\lambda) \perp \mathcal{L}_a(\mu)$.
\end{cor}

\begin{proof}
    Пусть $u \in \mathcal{L}_a(\lambda)$, $v \in \mathcal{L}_a(\mu) = \mathcal{L}_{\hat{a}}(\bar{\mu})$. Тогда:
    $$\begin{array}{ccc}
        (a(u), v)      & = & (u, \hat{a}(v)) \\
        \Vert          &   & \Vert           \\
        (\lambda u, v) &   & (u, \bar{\mu}v) \\
        \Vert          &   & \Vert           \\
        \lambda(u, v)  &   & \mu(u, v)
    \end{array}$$
    Отсюда $(\mu - \lambda)(u, v) = 0$. Но $\mu \neq \lambda$, следовательно $(u, v) = 0$.
\end{proof} % Нормальные операторы в УП. Свойства собственных векторов нормального оператора
    \section{Нормальные операторы в ЕП (лемма об инвариантном двумерном подпространстве, каноническая форма матрицы нормального оператора, следствия)}

% Мне физически больно было это все техать.

\begin{lem*}
    Пусть $L$ --- ЕП, $a$ --- нормальный оператор $L$, не имеющий вещественных собственных чисел. Тогда существует двумерное $a$- и $\hat{a}$-инвариантное подпространство $V \le \lsub{\mathbb{R}}{L}$ и при этом существует ОН-базис $B_0$ в $\lsub{\mathbb{R}}{V}$, т.ч. $[a\vert_V]_{B_0} = \begin{pmatrix}
        \alpha & \beta \\
        -\beta & \alpha
    \end{pmatrix}$, где $\alpha, \beta \in \mathbb{R}$.
\end{lem*}

\begin{proof}
    Пусть $B = \family{u_k}{k=1}{n}$ --- ОН-базис $\lsub{\mathbb{R}}{L}$, $A \coloneqq [a]_B$ --- нормальная матрица. Введем вспомогательный оператор $d \colon \mathbb{C}^n \to \mathbb{C}^n$, т.ч. $x \mapsto Ax$. Так как $A$ нормальна, то $d$ --- нормальный оператор.
    \smallskip
    
    Пусть $\lambda = \alpha + \beta i \in \spec(d)$ (здесь $\beta \neq 0$) и $v = X + iY \in \mathcal{L}_d(\lambda) \setminus \nilset$ (где $X, Y \in \mathbb{R}^n$). Так как $A(X + iY) = d(v) = \lambda v = (a + \beta i)(X + iY)$, то по методу неопределенных коэффициентов имеем:
    \begin{equation}\label{09-20:1}
        \begin{cases}
            AX = \alpha X - \beta Y \\
            AY = \beta X + \alpha Y
        \end{cases}
    \end{equation}
    
    Так как $A \bar{v} = A(X - iY) = \bar{\lambda} \bar{v}$, то $\bar{v} \in \mathcal{L}_d(\bar{\lambda}) \perp \mathcal{L}_d(\lambda) \ni v$. Отсюда $0 = (v, \bar{v}) = (X + iY)^T(\overline{X - iY}) = (X + iY)^T (X +iY) = X^T X + iY^T X + iX^T Y - Y^T Y = X^T X - Y^T Y + 2i X^T Y$, откуда:
    \begin{equation}\label{09-20:2}
        \begin{cases}
            X^T X = Y^T Y \\
            X^T Y = 0
        \end{cases}
    \end{equation}
    
    Так как $v \in \mathcal{L}^{\hat{a}}(\bar{\lambda})$ и $[\hat{a}]_B = A^T$, то:
    \begin{equation}\label{09-20:3}
        \begin{cases}
            A^T X = \alpha X + \beta Y \\
            A^T Y = -\beta X + \alpha Y
        \end{cases}
    \end{equation}
    При этом $\norm{v} = (X + iY)^T (X - iY) = X^T X + Y^T Y \stackrel{\eqref{09-20:2}}{=} 2 X^T X \stackrel{\eqref{09-20:2}}{=} 2 Y^T Y$. Выбрав $v$ такой, что $\norm{v} = 2$, получим $X^T X = Y^T Y = 1$.
    \smallskip
    
    Определим $v_1$ и $v_2$ следующим образом
    \begin{align*}
        v_1 &\coloneqq \sum_{k=1}^n x_k u_k,\ X = (x_k)_{k=1}^n \\
        v_2 &\coloneqq \sum_{k=1}^n y_k u_k,\ Y = (y_k)_{k=1}^n
    \end{align*}
    Рассмотрим $\lsub{\mathbb{R}}{V} \coloneqq \gen{\mathbb{R}}{v_1, v_2} \le \lsub{\mathbb{R}}{L}$. По \eqref{09-20:1} имеем:
    $$\begin{cases}
        a(v_1) = \alpha v_1 - \beta v_2 \in V \\
        a(v_2) = \beta v_1 + \alpha v_2 \in V
    \end{cases}$$
    откуда $V$ $a$-инвариантно. По \eqref{09-20:3} имеем:
    $$\begin{cases}
        \hat{a}(v_1) = \alpha v_1 + \beta v_2 \in V \\
        \hat{a}(v_2) = -\beta v_1 + \alpha v_2 \in V
    \end{cases}$$
    откуда $V$ $\hat{a}$-инвариантно.
    
    Из \eqref{09-20:2} также следует, что $(v_1, v_2) = 0$. По выбору $v$ имеем $\norm{v_1} = \norm{v_2} = 1$. При этом $[a\vert_V] = \begin{pmatrix}
        \alpha & -\beta \\
        \beta & \alpha
    \end{pmatrix}$ и лемма доказана.
\end{proof}

\begin{thm*}
    Пусть $L$ --- ЕП, $a$ --- оператор $L$. Справедливы следующие утверждения:
    \begin{enumerate}
        \item $a$ нормален $\Leftrightarrow$ существует ОН-базис $B$ в $\lsub{\mathbb{R}}{L}$, т.ч.
        \begin{equation}\label{09-20:star}\tag{$*$}
            [a]_B = \begin{pmatrix}
                A_1 &        & \\
                    & \ddots & \\
                    &        & A_t
            \end{pmatrix},\
            \text{где $A_k$ --- либо $1 \times 1$-матрица, либо имеет вид}\
            \begin{pmatrix}
                \alpha_k & \beta_k \\
                -\beta_k & \alpha_k
            \end{pmatrix}
        \end{equation}
        \item Такой вид матрицы определен однозначно оператором $a$ с точностью до перестановки диагональных блоков.
    \end{enumerate}
\end{thm*}

\begin{proof}
    \begin{proofpart}
        \underline{($\Rightarrow$)}. Пусть $\dim \lsub{\mathbb{R}}{L} = n$. Проведем доказательство индукцией по $n$.
        \smallskip
        
        $\mathbf{n = 1}$. Тривиальный случай.
        
        $\mathbf{n = 2}$. Возможно два случая:
        \begin{enumerate}
            \item если $\spec(a) \neq \varnothing$, то рассмотрим $\mathcal{L}_a(\lambda)$. Если оно совпадает с $L$, то достаточно выбрать любой ОН-базис, чтобы получить диагональную форму. В противном случае $\mathcal{L}_a(\lambda)$ и $\mathcal{L}_a(\lambda)^\perp$ $a$- и $\hat{a}$-инвариантны, и к ним можно применить случай $n = 1$.
            \item если $\spec(a) = \varnothing$, то по лемме в $L$ можно выбрать ОН-базис $B_0$, т.ч. $[a]_B = \begin{pmatrix}
                \alpha & \beta \\
                -\beta & \alpha
            \end{pmatrix}$.
        \end{enumerate}
    
        $\mathbf{n - 1 \to n}$. Пусть утверждение верно для $\dim \lsub{\mathbb{R}}{L} < n$. Тогда возможно два случая:
        \begin{enumerate}
            \item если $\spec(a) \neq \varnothing$, то случай аналогичен похожему случаю для $n = 2$ с применением ИП.
            \item если $\spec(a) \neq \varnothing$, то по лемме выделим двумерное $a$- и $\hat{a}$-инвариантное подпространство $V \le \lsub{\mathbb{R}}{L}$. Тогда $V$ и $V^\perp$ $a$- и $\hat{a}$-инвариантны, и к ним можно применить ИП.
        \end{enumerate}
        \smallskip
        
        \underline{($\Leftarrow$)} Достаточно доказать, что каждый блок получившейся матрицы нормален. В случае $1 \times 1$-блока это очевидно, а для $A_k = \begin{pmatrix}
            \alpha_k & \beta_k \\
            -\beta_k & \alpha_k
        \end{pmatrix}$ нормальность легко проверяется умножением.
    \end{proofpart}

    \begin{proofpart}
        Т.к. искомая форма матрицы блочно-диагонально, то такое представление соответствует разложению соответствующего $\lsub{\mathbb{R}}{L}$ $\mathbb{R}[x]$-модуля в прямую сумму подмодулей.
        
        Для блока $A_k$ вида $(\lambda)$ минимальным многочленом является $x - \lambda$, поэтому соответствующий подмодуль имеет вид $\mathbb{R}[x] / \langle x - \lambda \rangle$ --- циклический примарный.
        
        Для блока $A_k$ вида $\begin{pmatrix}
            \alpha_k & \beta_k \\
            -\beta_k & \alpha_k
        \end{pmatrix}$ характеристический многочлен имеет вид $\chi_{A_k} = (\alpha_k - x)^2 + \beta_k^2$ и, очевидно, неприводим. Тогда минимальный многочлен (делящий $\chi_{A_k}$ по следствию теоремы Гамильтона-Кэли) совпадает с ним. Отсюда такому блоку соответствует подмодуль вида $\mathbb{R}[x] / \langle x^2 - 2 \alpha_k x + \alpha_k^2 + \beta_k^2 \rangle$ --- циклический примарный. Такое разложение единственно с точностью до порядка слагаемых по теореме Крулля-Шмидта. Тогда единственно и представление матрицы в искомом виде.
    \end{proofpart}
\end{proof}

\begin{cor}
    $A \in M_n(\mathbb{R})$ нормальна $\Leftrightarrow$ существует ортогональная $C \in M_n(\mathbb{R})$, т.ч. $C^T AC$ имеет вид \eqref{09-20:star}.
\end{cor}

\begin{proof}
    Очевидно.
\end{proof}

\begin{cor}
    $A \in M_n(\mathbb{R})$. $A$ симметрична $\Leftrightarrow$ существует ортогональная $C \in M_n(\mathbb{R})$, т.ч. $C^T AC$ диагональна.
\end{cor}

\begin{proof}
    \begin{proofpart}{($\Rightarrow$)}
        Можно считать, что $A$ имеет вид \eqref{09-20:star}. Так как она симметрична, то она диагональна.
    \end{proofpart}
    
    \begin{proofpart}{($\Leftarrow$)}
        $A^T = (CDC^T)^T = CD^TC^T = CDC^T = A$, откуда $A$ симметрична.
    \end{proofpart}
\end{proof}

\begin{cor}
    Пусть $A \in M_n(\mathbb{R})$ --- симметрическая матрица, тогда $\chi_A$ имеет только вещественные корни.
\end{cor}

\begin{proof}
    Пусть $A = CDC^{-1}$, где $D = \diag(\lambda_1, \dots, \lambda_n),\ \lambda_k \in \mathbb{R}$. Тогда $\chi_A = \chi_D = \prod_{k=1}^n (\lambda_k - x)$.
\end{proof} % Нормальные операторы в ЕП (лемма об инвариантном двумерном подпространстве, каноническая форма матрицы нормального оператора, следствия)
    \section{Изометрические операторы, простейшие свойства}

\begin{defn}
    Пусть $L$ --- ЕП или УП, $a$ --- оператор $L$. $a$ называется \textit{изометрическим}, если $\forall u, v\ (a(u), a(v)) = (u, v)$. Изометрические операторы в УП называются \textit{унитарными}, а в ЕП --- \textit{ортогональными}.
\end{defn}

\begin{thm*}
    $L$ --- ЕП или УП, $a$ --- оператор. $a$ изометричен $\Leftrightarrow$ $\hat{a} a = \id_L$.
\end{thm*}

\begin{proof}
    \begin{proofpart}{($\Rightarrow$)}
        $\forall u, v \in L$ имеем $(u, v) = (a(u), a(v)) = (u, \hat{a}a(v))$, откуда $\hat{a}a(v) = v$ и $\hat{a} a = \id_L$.
    \end{proofpart}

    \begin{proofpart}{($\Leftarrow$)}
        Имеем $(a(u), a(v)) = (u, \hat{a}a(v)) = (u, v)$ и $a$ изометричен.
    \end{proofpart}
\end{proof}

\begin{cor}
    Любой изометрический оператор обратим и обратен своему сопряженному.
\end{cor}

\begin{proof}
    Следует из формулировки теоремы.
\end{proof}

\begin{cor}
    $a$ изометричен $\Leftrightarrow$ $\hat{a}$ изометричен.
\end{cor}

\begin{proof}
    Следует из того, что $\hat{\hat{a}} = a$.
\end{proof}

\begin{cor}
    Любой изометрический оператор нормален.
\end{cor}

\begin{proof}
    $\forall a\ \hat{a} a = \id_L = a \hat{a}$ и $a$ нормален.
\end{proof}

\begin{cor}
    $L$ --- УП, $a$ --- оператор $L$. Равносильны:
    \begin{enumerate}
        \item $a$ --- унитарный оператор.
        \item Для любого ОН-базиса $B$ матрица $[a]_B$ унитарна.
        \item Существует ОН-базис $B$, т.ч. матрица $[a]_B$ унитарна.
    \end{enumerate}
\end{cor}

\begin{cor}
    $L$ --- ЕП, $a$ --- оператор $L$. Равносильны:
    \begin{enumerate}
        \item $a$ --- ортогональный оператор.
        \item Для любого ОН-базиса $B$ матрица $[a]_B$ ортогональна.
        \item Существует ОН-базис $B$, т.ч. матрица $[a]_B$ ортогональна.
    \end{enumerate}
\end{cor} % Изометрические операторы, простейшие свойства
    \section{Унитарный оператор: свойства собственных чисел, связь между унитарными и эрмитовыми матрицами. Унитарная группа}

\begin{thm*}
    $L$ --- УП, $a$ --- оператор. $a$ унитарен $\Leftrightarrow \exists$ ОН-базис $B$, т.ч. $[a]_B$ диагональна и $\forall \lambda \in \spec(a)\ |\lambda| = 1$.
\end{thm*}

\begin{proof}
    Из унитарности $a$ следует нормальность $a$ и диагональность матрицы $[a]_B = \diag(\lambda_1, \dots, \lambda_n) \eqqcolon A$. Рассмотрим $A^* = \diag(\bar{\lambda_1}, \dots, \bar{\lambda_n})$. $a$ унитарен в том и только в том случае, когда $A^*A = E$. Но $A^*A = E \Leftrightarrow \diag({|\lambda_1|}^2, \dots, {|\lambda_n|}^2) = E \Leftrightarrow \forall \lambda \in \spec(a)\ |\lambda| = 1$.
\end{proof}

\begin{cor}
    $A \in M_n(\mathbb{C})$ унитарна $\Leftrightarrow \exists$ унитарная $U \in M_n(\mathbb{C})$, т.ч. $U^{-1} AU$ диагональна и $\forall \lambda \in \spec(a) |\lambda| = 1$.
\end{cor}

\begin{proof}
    Следует из свойств унитарной матрицы и доказательства теоремы.
\end{proof}

\begin{cor}
    $A \in M_n(\mathbb{C})$ унитарна $\Leftrightarrow \exists$ эрмитова $\Phi \in M_n(\mathbb{C})$, т.ч. $A = e^{i\Phi}$.
\end{cor}

\begin{proof}
    \begin{proofpart}{($\Rightarrow$)}
        Пусть $U \in M_n(\mathbb{C})$ --- унитарная матрица, т.ч. $U^{-1} AU = D = \diag(\lambda_1, \dots, \lambda_n)$, где $|\lambda_j| = 1\ \forall j$. Пусть $\lambda_j = e^{i\phi_j}$.
        
        Положим $\tilde{\Phi} \coloneqq \diag(\phi_1, \dots, \phi_n)$. Имеем $D = e^{i\tilde{\Phi}} \leadsto A = UDU^{-1} = Ue^{i\tilde{\Phi}}U^{-1} = e^{iU\tilde{\Phi}U^{-1}}$. Положим $\Phi \coloneqq U\tilde{\Phi}U^{-1}$. Получим $D = e^{i\Phi}$. При этом $\Phi^* = (U\tilde{\Phi}U^{-1})^* = (U^{-1})* \tilde{\Phi}^* U^* = U \tilde{\Phi} U^{-1} = \Phi$, т.е. $\Phi$ эрмитова.
    \end{proofpart}

    \begin{proofpart}{($\Leftarrow$)}
        Имеем $A = e^{i\Phi}$, где $\Phi$ эрмитова. Тогда:
        $$A^* = (e^{i\Phi})^* = \left(\sum_{k=0}^\infty \frac{1}{k!} (i\Phi)^k\right)^* = \sum_{k=0}^\infty \frac{1}{k!} \left((i\Phi)^*\right)^k = \sum_{k=0}^\infty \frac{1}{k!} (-i\Phi)^k = e^{-i\Phi}$$
        Ясно, что $i\Phi$ и $-i\Phi$ коммутируют. Имеем $AA^* = e^{i\Phi} e^{-i\Phi} = E$, откуда $A$ унитарна.
    \end{proofpart}
\end{proof}

\begin{defn}
    $L$ --- УП, $n \coloneqq \dim \lsub{\mathbb{C}}{L}$. Множество всех унитарных операторов на $L$ образует группу. Аналогично, множество всех унитарных $n \times n$-матриц образует группу $U(n)$, называемую унитарной группой порядка $n$.
\end{defn} % Унитарный оператор: свойства собственных чисел, связь между унитарными и эрмитовыми матрицами. Унитарная группа
    \section{Ортогональные операторы: каноническая форма матрицы ортогонального оператора. Ортогональная группа, специальная ортогональная группа}

\begin{thm*}
    $L$ --- ЕП, $a$ --- оператор $L$. Справедливы следующие утверждения:
    \begin{enumerate}
        \item $a$ ортогонален $\Leftrightarrow \exists$ ОН-базис $B$, т.ч.
        \begin{equation}\label{09-23:venus}\tag{\Venus}
            [a]_B = \begin{pmatrix}
                E_s &      &     &        & \\
                    & -E_t &     &        & \\
                    &      & A_1 &        & \\
                    &      &     & \ddots & \\
                    &      &     &        & A_t
            \end{pmatrix},\
            \text{где}\ A_k = \begin{pmatrix}
                \cos \phi_k & -\sin \phi_k \\
                \sin \phi_k & \cos \phi_k
            \end{pmatrix},\ \phi_k \not\in \left\{\pi l \mid l \in \mathbb{Z} \right\}
        \end{equation}
        \item Матрица вида \eqref{09-23:venus} определена однозначно с точностью до перестановки блоков $A_k$.
    \end{enumerate}
\end{thm*}

\begin{proof}
    \begin{proofpart}
        \underline{($\Rightarrow$)} $a$ ортогонален $\leadsto a$ нормален $\leadsto \exists$ ОН-базис $B$, т.ч. матрица $[a]_B$ имеет канонический вид (\eqref{09-20:star} из соответствующей теоремы). Обозначим блоки такой матрицы как $B_1, \dots, B_t$. Так как $A^T A = E$, то $B_k^T B_k \in \{E_1, E_2\}$. Имеем $B_k = (\pm 1)$ в случае $1 \times 1$-блока и $\alpha_k^2 + \beta_k^2 = 1$ в случае второго блока. Во втором случае $\exists \phi_k \not\in \left\{\pi l \mid l \in z \right\}$, т.ч. $\begin{cases}
            \alpha_k = \cos \phi_k \\
            -\beta_k = \sin \phi_k
        \end{cases}$, и соответствующий блок $A_k$ имеет нужный вид.
        \smallskip
        
        \underline{$(\Leftarrow)$} Легко проверяется, что матрица вида \eqref{09-23:venus} ортогональна, а тогда ортогонален и соответствующий оператор.
    \end{proofpart}

    \begin{proofpart}
        Напрямую следует из единственности канонической формы матрицы нормального оператора.
    \end{proofpart}
\end{proof}

\begin{cor*}
    $A \in M_n(\mathbb{R})$ ортогональна $\Leftrightarrow \exists$ ортогональная $C \in M_n(\mathbb{R})$, т.ч. $C^T AC$ имеет вид \eqref{09-23:venus}.
\end{cor*}

\begin{proof}
    Очевидно.
\end{proof}

\begin{defn}
    Ортогональные операторы, действующие на ЕП $L$, образуют группу. Аналогично, ортогональные $n \times n$-матрицы образуют группу $\text{S}(n, \mathbb{R})$, называемую \textit{ортогональной}. $\forall M \in \text{S}(n, \mathbb{R})\ \det M = \pm 1$. Матрицы $M \in \text{S}(n, \mathbb{R})$, т.ч. $\det M = 1$, образуют группу $\text{SO}(n, \mathbb{R})$, называемую \textit{специальной ортогональной}.
\end{defn} % Ортогональные операторы: каноническая форма матрицы ортогонального оператора. Ортогональная группа, специальная ортогональная группа
    \section{Вращения в $\mathbb{R}^3$, теорема Эйлера. Несобственно ортогональные операторы в $\mathbb{R}^2$}

\begin{defn}
    Пусть $L$ --- ЕП, $a$ --- \textbf{ортогональный} оператор. Выберем $B$ --- произвольный ОН-базис $\lsub{\mathbb{R}}{L}$, т.ч. $\det [a]_B = \pm 1$. В случае, если $\det [a]_B = 1$, $a$ назовем \textit{собственно ортогональным}, в противном случае --- \textit{несобственно ортогональным}.
\end{defn}

\begin{defn}
    Собственно ортогональные операторы на $\mathbb{R}^2$ и $\mathbb{R}^3$ со стандартным скалярным произведением называют вращениями.
\end{defn}

\begin{thm}
    Пусть $a \in \End \lsub{\mathbb{R}}{\mathbb{R}^3}$ --- вращение. Тогда существует ОН-базис $B$ в $R^3$, т.ч. $$[a]_B = \begin{pmatrix}
        \cos \phi & -\sin \phi & 0 \\
        \sin \phi & \cos \phi  & 0 \\
        0         & 0          & 1
    \end{pmatrix}$$
\end{thm}

\begin{proof}
    Следует из более общей теоремы о канонической форме матрицы ортогонального оператора.
\end{proof}

\begin{thm}{(Эйлера)}
    Композиция двух вращений в $\mathbb{R}^3$ вокруг пересекающихся осей является вращением вокруг некоторой третьей оси.
\end{thm}

\begin{proof}
    Следует из того, что вращения в $R^3$ образуют группу.
\end{proof}

\begin{thm}
    Любой несобственно ортогональный оператор в $\mathbb{R}^2$ является отражением отностиельно некоторой прямой.
\end{thm}

\begin{proof}
    Каноническая форма несобственно ортогонального оператора в $\mathbb{R}^2$ имеет вид $\begin{pmatrix}
        1 & 0 \\
        0 & -1
    \end{pmatrix}$, а отсюда очевидным образом следует утверждение теоремы.
\end{proof} % Вращения в R^3, теорема Эйлера. Несобственно ортогональные операторы в R^2
    
    % С нумерацией будут костыли
    \section*{9.25 -- 9.27. Свойства самосопряженных операторов на ЕП и УП}

\setcounter{thm}{0}

\begin{thm}
    Пусть $L$ --- УП, $a$ --- самосопряженный оператор (т.е. т.ч. $a = \hat{a}$). Тогда $\forall u \in L\ (a(u), u) \in \mathbb{R}$.
\end{thm}

\begin{proof}
    $\overline{(a(u), u)} = (u, a(u)) = (a(u), u)$, откуда $(a(u), u) \in \mathbb{R}$.
\end{proof}

\begin{thm}
    $L$ --- ЕП, $a$ --- оператор $L$. $a$ --- самосопряженный $\Leftrightarrow \exists$ ОН-базис $B$ в $\lsub{\mathbb{R}}{L}$, т.ч. $[a]_B$ диагональна.
\end{thm}

\begin{proof}
    \begin{proofpart}{($\Rightarrow$)}
        $a$ самосопряжен $\leadsto a$ нормален $\leadsto \exists$ ОН-базис $B$ в $\lsub{\mathbb{R}}{L}$, т.ч. $A \coloneqq [a]_B$ имеет канонический вид \eqref{09-20:star} из соответствующей теоремы. Так как $a$ самосопряжен, то $A^T = A$; соответственно блоков вида $\begin{pmatrix}
            \alpha_k & \beta_k \\
            \beta_k  & \alpha_k
        \end{pmatrix}$ в $A$ нет, и $A$ диагональна.
    \end{proofpart}

    \begin{proofpart}{($\Leftarrow$)}
        Пусть $[a]_B = \diag(\lambda_1, \dots, \lambda_n)$. Тогда $[a]_B^T = [a]_B$, и $a$ самосопряжен.
    \end{proofpart}
\end{proof}

\begin{thm}
    $L$ --- УП, $a$ --- оператор. $a$ самосопряжен $\Leftrightarrow \exists$ ОН-базис в $\lsub{\mathbb{C}}{L}$, т.ч. $[a]_B$ диагональна, и при этом $\forall \lambda \in \spec(a)\ \lambda \in \mathbb{R}$.
\end{thm}

\begin{proof}
    \begin{proofpart}{($\Rightarrow$)}
        $a$ самосопряжен $\leadsto a$ нормален $\leadsto \exists$ ОН-базис $B$ в $\lsub{\mathbb{R}}{L}$, т.ч. $A \coloneqq [a]_B = \diag(\lambda_1, \dots, \lambda_n)$. При этом $A^* = A$, откуда $\forall k\ \lambda_k \in \mathbb{R}$ и все собственные числа матрицы (а, соответственно, и оператора) вещественны.
    \end{proofpart}
    
    \begin{proofpart}{($\Leftarrow$)}
        Пусть $A \coloneqq [a]_B = \diag(\lambda_1, \dots, \lambda_n)$ и $\lambda_k \in \mathbb{R}$. Тогда $A^* = A$, и $a$ самосопряжен.
    \end{proofpart}
\end{proof}

\begin{cor*}
    Эрмитова матрица имеет только вещественные собственные числа.
\end{cor*} % Свойства самосопряженных операторов на ЕП и УП (+9.26, 9.27)
    \setcounter{section}{27}
    
    \section{Положительно определенные самосопряженные операторы, простейшие свойства, примеры (операторы $\hat{a}a$ и $a\hat{a}$)}

\begin{defn}
    $L$ --- ЕП или УП, $a$ --- самосопряженный оператор. $a$ называется \textit{положительно определенным}, если:
    \begin{enumerate}
        \item $\forall u \in L\ (a(u), u) \ge 0$.
        \item $(a(u), u) = 0 \Leftrightarrow u = 0$.
    \end{enumerate}
\end{defn}

\begin{rem}
    Если $a$ положительно определен, то $a$ обратим:
    $$a(u) = \nil \leadsto (a(u), u) = 0 \leadsto u = \nil \leadsto \ker a = \nilset$$
    откуда $a$ мономорфизм и, как следствие, изоморфизм.
\end{rem}

\begin{thm}
    $L$ --- ЕП или УП, $a$ --- оператор. Пусть $a$ обратим, тогда $a\hat{a}$ и $\hat{a}a$ --- положительно определенные самосопряженные операторы.
\end{thm}

\begin{proof}
    Ясно, что $a \hat{a}$ самосопряжен. Имеем:
    \begin{enumerate}
        \item $(a\hat{a}(u), u) = (\hat{a}(u), \hat{a}(u)) \ge 0$.
        \item $(a\hat{a}(u), u) = 0 \leadsto \hat{a}(u) = \nil \stackrel{\cdot \widehat{a^{-1}}}{\leadsto} u = \nil$.
    \end{enumerate}
    Для $\hat{a} a$ утверждение доказывается аналогично.
\end{proof}

\begin{thm}
    $L$ --- ЕП или УП, $a$ --- самосопряженный оператор. $a$ положительно определен $\Leftrightarrow \forall \lambda \in \spec(a)\ \lambda > 0$.
\end{thm}

\begin{proof}
    \begin{proofpart}{($\Rightarrow$)}
        Пусть $a$ самосопряжен, тогда $\exists$ ОН-базис $B$ в $\lsub{K}{L}$, т.ч. $[a]_B = \diag(\lambda_1, \dots, \lambda_n)$ и $\lambda_k \in \mathbb{R}\ \forall k$. Имеем:
        $$\lambda_j = (a(e_j), e_j) > 0 \leadsto (\lambda_j e_j, e_j) = \lambda_j (e_j, e_j) = \lambda_j \cdot 1 > 0 \leadsto \lambda_j > 0\ \forall j$$
    \end{proofpart}

    \begin{proofpart}{($\Leftarrow$)}
        Пусть $B = \family{e_j}{j=1}{n}$ --- ОН-базис в $\lsub{K}{L}$, т.ч. $[a]_B = \diag(\lambda_1, \dots, \lambda_n)$ и $\forall j\ \lambda_j > 0$. Пусть $u \in L$, тогда $u$ представим в виде $u = \sum_{j=1}^n x_j e_j$, где $x_j \in K$. Имеем:
        \begin{align*}
            (a(u), u) &= \left(a\left(\sum_{j=1}^n x_j e_j\right),\ \sum_{k=1}^n x_k e_k\right) = \sum_{j,k=1}^n x_j \bar{x_k} (a(e_j), e_k) \\
            &= \sum_{j,k=1}^n x_j \bar{x_k} \lambda_j (e_j, e_k) = \sum_{s=1}^n x_s \bar{x_s} \lambda_s = \sum_{s=1}^n \norm{x_s}^2 \lambda_s \ge 0
        \end{align*}
        
        Предположим, что $(a(u), u) = 0$, тогда $\sum_{s=1}^n \norm{x_s}^2 \lambda_s = 0$. Но $\lambda_s > 0\ \forall s$, откуда $\norm{x_s}^2 = 0\ \forall s$ и $u = \nil$.
    \end{proofpart}
\end{proof} % Положительно определенные самосопряженные операторы, простейшие свойства, примеры (операторы $\hat{a}a$ и $a\hat{a}$)
    \section{Теорема об извлечении корня из положительно определенного оператора}

\begin{thm*}
    Пусть $L$ --- ЕП или УП, $a$ --- положительно определенный оператор. Пусть $m \in \mathbb{N} \setminus \{1\}$, тогда $\exists!$ положительно определенный оператор $c$, т.ч. $c^m = a$.
\end{thm*}

\begin{proof}
    \begin{proofpart}{(существование)}
        Имеем $a$ --- положительно определенный самосопряженный оператор. Тогда $\exists$ ОН-базис $B = \family{e_j}{j=1}{n}$, т.ч. $[a]_B = \diag(\lambda_1, \dots, \lambda_n)$, где $\lambda_j > 0\ \forall j$. Пусть $\mu_j = \sqrt[m]{\lambda_j} \in \mathbb{R}_+$. Построим оператор $c$, т.ч. $[c]_B = \diag(\mu_1, \dots, \mu_n)$, т.е. $c(e_j) = \mu_j e_j$. Тогда $[c^m]_B = [a]_B$ и, соответственно, $c^m = a$. Заметим, что $\forall \lambda \in \spec(a)\ \mathcal{L}_a(\lambda) = \mathcal{L}_c (\sqrt[m]{\lambda})$.
    \end{proofpart}

    \begin{proofpart}{(единственность)}
        Предположим, что $\tilde{c}$ --- еще один положительно определенный оператор, т.ч. $\tilde{c}^m = a$. Тогда $\exists$ ОН-базис $B'$ в $\lsub{K}{L}$, т.ч. $[\tilde{c}]_{B'} = \diag(\tilde{\mu_1}, \dots, \tilde{\mu_n})$, причем $\forall j\ \tilde{\mu_j} > 0$. Тогда $[a]_B' = \diag(\tilde{\mu_1}^m, \dots, \tilde{\mu_n}^m)$. По единственности ЖНФ можно считать, что $\lambda_j = \mu_j^m = \tilde{\mu_j}^m\ \forall j$. Аналогично предыдущему пункту имеем $\mathcal{L}_c(\sqrt[m]{\lambda}) = \mathcal{L}_a(\lambda) = \mathcal{L}_{\tilde{c}}(\sqrt[m]{\lambda})\ \forall \lambda \in \spec(a)$, тогда $c$ и $\tilde{c}$ совпадают на $\mathcal{L}_a(\lambda)\ \forall \lambda \in \spec(a)$. Но тогда они совпадают и на $\bigoplus_{\lambda \in \spec(a)} \mathcal{L}_a(\lambda) = L$, откуда $\tilde{c} \equiv c$.
    \end{proofpart}
\end{proof} % Теорема об извлечении корня из положительно определенного оператора
    \section{Теорема о полярном разложении оператора}

\begin{thm*}
    $L$ --- ЕП или УП, $a$ --- оператор. $a$ обратим $\Leftrightarrow$
    \begin{enumerate}
        \item $a$ можно представить в виде $a = r \circ u$, где $r$ --- положительно определенный самосопряженный оператор, $u$ --- изометрический оператор.
        \item Такое представление единственно.
    \end{enumerate}
\end{thm*}

\begin{proof}
    \begin{proofpart}
        Так как $a$ обратим, то $a \hat{a}$ --- положительно определенный самосопряженный оператор. По теореме об извлечении корня $\exists!$ положительно определенный самосопряженный оператор $r$, т.ч. $r^2 = a \hat{a}$. Т.к. $r$ положительно определен, то он обратим. Положим $u \coloneqq r^{-1} a$. Тогда $\hat{u} = \widehat{r^{-1} a} = \hat{a} \widehat{r^{-1}} = \hat{a} r^{-1}$, откуда $u\hat{u} = r^{-1} a\hat{a} r^{-1} = r^{-1} r^2 r^{-1} = \id_L$ и $u$ изометричен.
    \end{proofpart}

    \begin{proofpart}
        Предположим, что $a$ допускает еще одно полярное разложение в виде $a = r'u'$. Тогда:
        \begin{align*}
            \hat{a} &= \hat{u}\hat{r} = u^{-1} r \leadsto a\hat{a} = ru u^{-1} r = r^2 \\
            \hat{a} &= \hat{u'}\hat{r'} = (u')^{-1} r \leadsto a\hat{a} = r'u' (u')^{-1} r' = (r')^2
        \end{align*}
        Таким образом, и $r$, и $r'$ --- квадратные корни оператора $a\hat{a}$. Тогда они совпадают, откуда $ru = ru' \stackrel{r^{-1} \cdot}{\leadsto} u = u'$.
    \end{proofpart}
\end{proof} % Теорема о полярном разложении оператора
    \section{Характеризация операторов ортогонального проектирования}

\begin{thm*}
    $L$ --- ЕП или УП. $a$ --- оператор. Равносильны:
    \begin{enumerate}
        \item $a$ --- оператор ортогонального проектирования.
        \item $a$ --- самосопряженный проектор.
        \item $a$ самосопряжен и $\forall \lambda \in \spec(a)\ \lambda \in \{0, 1\}$.
    \end{enumerate}
\end{thm*}

\begin{proof}\ % еее костыльная верстка
    \medskip
    
    \textbf{(а) $\Rightarrow$ (в).} 
    Представим $L$ в виде $\lsub{K}{L} = \im a \oplus (\im a)^\perp$. Выберем в $\im a$ и $(\im a)^\perp$ ОН-базисы $B_0$ и $B_1$ соответственно. Положим $B \coloneqq B_0 \cup B_1$ --- это будет ОН-базис $\lsub{K}{L}$. Тогда $A \coloneqq [a]_B = \diag(\underbrace{1, \dots, 1}_{k\ \text{раз}}, 0, \dots, 0)$, где $k = \dim \lsub{K}{\im a}$, откуда $\spec(a) = \spec A \subset \{0, 1\}$.
    \medskip
    
    \textbf{(в) $\Rightarrow$ (б).}
    Так как $a$ самосопряжен, то $\exists$ ОН-базис $B$ в $\lsub{K}{L}$, т.ч. $A \coloneqq [a]_B = \diag(\lambda_1, \dots, \lambda_n)$, при этом $\lambda_j \in \{0, 1\}$. Тогда $A^2 = A$, откуда $a^2 = a$ и $a$ --- проектор.
    \medskip
    
    \textbf{(б) $\Rightarrow$ (а).} 
    Представим $L$ в виде $\lsub{K}{L} = \im a \oplus \ker a$. Тогда по самосопряженности $a$ имеем $\ker a = \ker \hat{a} = (\im a)^\perp$ и $L = \im a \oplus (\im a)^\perp$, откуда $a$ --- оператор ортогонального проектирования.
    \medskip
    
\end{proof} % Характеризация операторов ортогонального проектирования
    \section{Спектральное разложение самосопряженного оператора. Спектральное разложение симметрической вещественной формы}

\begin{thm*}
    $L$ --- ЕП или УП, $a$ --- самосопряженный оператор. Для $\lambda \in \spec(a)$ введем обозначения $V_\lambda \coloneqq \mathcal{L}_a(\lambda)$, $p_\lambda$ --- оператор ортогонального проектирования на $V_\lambda$. Справедливы следующие утверждения:
    \begin{enumerate}
        \item $a = \sum_{\lambda \in \spec(a)} \lambda p_\lambda$, причем $\lambda$ считаются без учета кратности.
        \item $\forall \lambda, \mu \in \spec(a)\ p_\lambda p_\mu = \begin{cases}
            p_\lambda, & \mu = \lambda \\
            0,         & \mu \neq \lambda
        \end{cases}$
    \end{enumerate}
\end{thm*}

\begin{proof}
    \begin{proofpart}
        Так как $\lsub{K}{L} = \bigoplus_{\lambda \in \spec(a)} V_\lambda$, то $\forall u \in L$ имеем $u = \sum_{\lambda \in \spec(a)} u_\lambda$, где $u_\lambda \in V_\lambda$. Так как $a$ нормален, то $V_\lambda \perp V_\mu\ \forall \lambda \neq \mu \in \spec(a)$. При этом ясно, что $p_\lambda(u) = u_\lambda$. Так,
        $$a(u) = a\left(\sum_{\lambda \in \spec(a)} u_\lambda\right) = \sum_{\lambda \in \spec(a)} a(u_\lambda) = \sum_{\lambda \in \spec(a)} \lambda u_\lambda = \sum_{\lambda \in \spec(a)} \lambda p_\lambda(u) = \left(\sum_{\lambda \in \spec(a)} \lambda p_\lambda\right)(u)$$
    \end{proofpart}

    \begin{proofpart}
        Очевидным образом следует из того, что $p_\lambda(u) = u_\lambda$.
    \end{proofpart}
\end{proof}

\begin{rem}
    Можно переформулировать теорему следующим образом: пусть $L$ --- ЕП или УП, $a$ --- самосопряженный оператор, $B = \family{e_j}{j=1}{n}$ --- ОН-базис, состоящий из собственных векторо. Положим $V_j = \gen{K}{e_j}$, $p_j$ --- оператор проектирования на $V_j$. Пусть $\lambda_j$ --- собственное число, которому принадлежит $e_j$. Тогда:
    \begin{enumerate}
        \item $a = \sum{j=1}^n \lambda_j p_j$.
        \item $p_k p_j = \begin{cases}
            p_k, & k = j \\
            0,   & k \neq j
        \end{cases}$
    \end{enumerate}
\end{rem}

\begin{cor*}
    Пусть $A \in M_n(\mathbb{R})$ и $A^T = A$. Пусть $\family{u_k}{k=1}{n}$ --- ОН-базис $\mathbb{R}^n$, состоящий из собственных векторов $A$ (т.е. $Au_k = \lambda_k u_k$). Тогда $A = \sum_{k=1}^n \lambda_k u_k u_k^T$.
\end{cor*}

\begin{proof}
    Достаточно воспользоваться переформулированным вариантом теоремы, заметив, что $(u_ku_k^T)x = u_k(u_k^Tx) = (u_k, x)u_k$ --- проекция $x$ на $u_k$.
\end{proof} % Спектральное разложение самосопряженного оператора. Спектральное разложение симметрической вещественной формы
    \section{Приведение пары вещественных квадратичных форм к каноническому виду}

\begin{thm*}
    Пусть $Q_1, Q_2 \colon \lsub{\mathbb{R}}{L} \to \mathbb{R}$ --- квадратичные формы, причем $L$ конечномерно. Пусть дополнительно $Q_1$ положительно определена. Тогда существует базис $B$ в $\lsub{\mathbb{R}}{L}$, т.ч. $[Q_1]_B$ и $[Q_2]_B$ одновременно диагональны.
\end{thm*}

\begin{proof}
    Пусть $F_1, F_2 \colon L^2 \to \mathbb{R}$ --- соответствующие билинейные формы. В силу положительной определенности $F_1$ пространство $(\lsub{\mathbb{R}}{L}, F_1)$ будет являться евклидовым. Выберем ОН-базис $\tilde{B}$ в $\lsub{\mathbb{R}}{L}$. Положим $\tilde{F_i} \coloneqq [F_i]_{\tilde{B}}$.
    
    Ясно, что $\tilde{F_1} = E_n$. Рассмотрим оператор $a$ на $L$, т.ч. $[a]_{\tilde{B}} = \tilde{F_2}$. Так как $\tilde{F_2}$ симметрична (как матрица симметрической билинейной формы), то $a$ самосопряжен. В таком случае существует ОН-базис $B$ в $\lsub{\mathbb{R}}{L}$, т.ч. $M_2 \coloneqq [a]_B = \diag(\lambda_1, \dots, \lambda_n)$. Пусть $\tilde{B} \stackrel{C}{\leadsto} B$. $C$ ортогональна как матрица перехода между ОН-базисами.
    
    Имеем:
    \begin{align*}
        [Q_2]_B &= [F_2]_B = C^T \tilde{F_2} C = C^{-1} \tilde{F_2} C = C^{-1} [a]_{\tilde{B}} C = M_2\ \text{(диагональная матрица)} \\
        [Q_1]_B &= [F_1]_B = C^T \tilde{F_1} C = C^T C = E_n\ \text{(диагональная матрица)}
    \end{align*}
\end{proof} % Приведение пары вещественных квадратичных форм к каноническому виду
    
    % TODO: Ахтунг, костыль! Его нужно исправить (хотя, впрочем-то, без разницы)
    \setcounter{section}{0}
        
    % Глава X. Полилинейная алгебра
    \part{Полилинейная алгебра}
    \section{Понятие тензорнго произведения модулей (определение, конструкция, следствия, примеры)}

\begin{defn}
    $k$ --- коммутативное кольцо с единицей, $\lsub{k}{U}$ и $\lsub{k}{V}$ --- модули. \textit{Тензорным произведением} $U$ и $V$ называется пара $(T, t)$, где $\lsub{k}{T}$ --- модуль, а $t \colon U \times V \to T$ --- билинейное отображение, обладающее \textit{универсальным свойством тензорного произведения}:
    \begin{diagram}
        U \times V &         & \rTo^t &                   & T \\
                   & \rdTo_f &        & \ldDashto_\varphi &   \\
                   &         & W      &                   &
    \end{diagram}
    то есть $\forall f \colon U \times V \to W$ --- билинейного отображения, где $\lsub{k}{W}$ --- некоторый модуль, $\exists!$ $k$-гомоморфизм $\varphi \colon T \to W$, т.ч. $\varphi t = f$.
    
    Тензорное произведение единственно с точностью до изоморфизма. В связи с этим используется стандартное обозначение $T \eqqcolon U \otimes_k V$ (или, если кольцо ясно из контекста, $U \otimes V$).
\end{defn}

\begin{thm*}
    $k$ --- коммутативное кольцо с единицей. Пусть $U$ и $V$ --- $k$-модули, тогда их тензорное произведение существует.
\end{thm*}

\begin{proof}
    Пусть $X \coloneqq U \times V$. Построим свободный модуль $F \coloneqq F\langle X \rangle$. Положим:
    \begin{align*}
        Y_1 &\coloneqq \left\{(\alpha_1u_1 + \alpha_2u_2, v) - \alpha_1(u_1, v) - \alpha_2(u_2, v) \mid \alpha_1, \alpha_2 \in k, u_1, u_2 \in U, v \in V \right\} \\
        Y_2 &\coloneqq \left\{(u, \alpha_1v_1 + \alpha_2v_2) - \alpha_1(u, v_1) - \alpha_2(u, v_2) \mid \alpha_1, \alpha_2 \in k, u \in U, v_1, v_2 \in V \right\} \\
        Y & \coloneqq Y_1\ \mathring{\cup}\ Y_2
    \end{align*}

    Пусть $G \coloneqq \gen{K}{Y} \le \lsub{k}{F}$. Рассмотрим $\lsub{k}{T} \coloneqq F/G$. Введем отображение $t \colon U \times V \to T$, т.ч. $(u, v) \mapsto (u, v) + G$ --- билинейное вследствие определения $G$.

    Докажем, что пара $(T, t)$ удовлетворяет универсальному свойству тензорного произведения. Пусть $\rho \colon F \to T$ --- отображение факторизации, $i \colon U \times V \to F$ --- вложение. Ясно, что $t = \rho i$. Пусть $\lsub{k}{W}$ --- произвольный модуль и $f \colon U \times V \to W$ --- билинейное отображение. Рассмотрим следующую диаграмму:
    \begin{diagram}
        U \times V & & \rTo^t & & T \\
        & \rdTo(2, 4)_f \rdInto^i & & \ruOnto^\rho \ldDashto(2, 4)_\varphi & \\
        & & F & & \\
        & & \dDashto_{\tilde{f}} & & \\
        & & W & &
    \end{diagram}
    
    По определению $F$ $\exists!$ $k$-гомоморфизм $\tilde{f} \colon F \to W$, т.ч. $f = \tilde{f} i$. Покажем, что $G \subset \ker \tilde{f}$. Для этого достаточно доказать, что $Y \subset \ker \tilde{f}$. Рассмотрим это на примере $Y_1$ (для $Y_2$ действия аналогичны):
    \begin{align*}
        f(y_1) &= \tilde{f}((\alpha_1 u_1 + \alpha_2 u_2, v) - \alpha_1(u_1, v) - \alpha_2(u_2, v)) \\
        &= \tilde{f}((\alpha_1 u_1 + \alpha_2 u_2, v)) - \alpha_1 \tilde{f}((u_1, v)) - \alpha_2 \tilde{f}((u_2, v)) = \\
        &= \tilde{f}(i(\alpha_1 u_1 + \alpha_2 u_2, v)) - \alpha_1 \tilde{f}(i(u_1, v)) - \alpha_2 \tilde{f}(i(u_2, v)) = \\
        &= f(\alpha_1 u_1 + \alpha_2 u_2, v) - \alpha_1 f(u_1, v) - \alpha_2 f(u_2, v) = f(0, v) = 0
    \end{align*}
    
    По теореме о продолжении гомоморфизма на фактормодуль $\exists!$ $k$-гомоморфизм $\varphi \colon T \to W$, т.ч. $\tilde{f} = \varphi \rho$. Тогда $\varphi t = \varphi \rho i = \tilde{f} i = f$. Предположим, что $\tilde{\varphi} \colon T \to W$ --- еще один $k$-гомоморфизм, т.ч. $\tilde{\varphi} t = f$. Тогда $f = (\tilde{\varphi} \rho) i$. При этом $\tilde{f} i = f$ и $\tilde{f}$ --- единственное отображение, обладающее таким свойством. В таком случае $\tilde{\varphi} \rho = \tilde{f}$, но $\phi$ --- единственное отображение, обладающее соответствующим свойствим, откуда $\varphi = \tilde{\varphi}$.
\end{proof}

\begin{defn}
    $t \colon U \times V \to T$ называют \textit{каноническим билинейным отображением}. Широко используется обозначение $t(u, v) \eqqcolon u \otimes v$.
\end{defn}

\begin{cor}
    Множество $\{u \otimes v \mid u \in U, v \in V\}$ порождает $U \otimes_k V$ как $k$-модуль.
\end{cor}

\begin{proof}
    Следует из того, что $U \times V$ порождает $F\langle U \times V \rangle$.
\end{proof}

\begin{cor}
    $U \otimes_k V \simeq V \otimes_k U$ как $k$-модули.
\end{cor}

\begin{proof}
    Рассмотрим следующую диаграмму:
    \begin{diagram}
        U \times V & \rTo^t & U \otimes_k V \\
        & \rdTo_{t'} & \dTo^\sigma \uTo_\tau \\
        & & V \otimes_k U
    \end{diagram}
    
    Положим $t \colon (u, v) \mapsto u \otimes v$ и $t' \colon (u, v) \mapsto v \otimes u$ --- билинейные отображения. Тогда $\exists!$ $k$-гомоморфизмы $\sigma$ и $\tau$, т.ч. $\sigma t = t'$ и $\tau t' = t$, откуда $\sigma(u \otimes v) = v \otimes u$ и $\tau(v \otimes u) = u \otimes v$ соответственно. В таком случае $\sigma \tau(u \otimes v) = u \otimes v$ и при этом $\{u \otimes v \mid u \in u, v \in V\}$ порождает $U \otimes V$, откуда $\sigma \tau = \id_{U \otimes V}$. Аналогично, $\tau \sigma = \id_{V \otimes U}$ и $\sigma$ --- искомый изоморфизм.
\end{proof}

\begin{exmpl}\
    \begin{enumerate}
        \item Пусть $k$ --- коммутативное кольцо с единицей, $U$ --- $k$-модуль. Тогда $k \otimes_k U \simeq U$.
        \begin{proof}
            Рассмотрим $f \colon k \times U \to U$, т.ч. $f(\alpha, u) = \alpha u$. Рассмотрим дополнительно следующую диаграмму:
            \begin{diagram}
                k \times U & & \rTo^t & & k \otimes U \\
                & \rdTo_f & & \ldDashto_\sigma & \\
                & & U & &
            \end{diagram}
            Согласно универсальному свойству, $\exists! \sigma \colon k \otimes U \to U$, т.ч. $f = \sigma t$.
            
            Рассмотрим $\tau \colon U \to k \otimes U$, т.ч. $u \mapsto 1 \otimes u$. Тогда $\tau \sigma(\alpha \otimes u) = \tau \sigma t(\alpha, u) = \tau f(\alpha, u) = \tau(\alpha u) = \alpha(1 \otimes u) = \alpha \otimes u$. Аналогично, $\sigma \tau(u) = \sigma(1 \otimes u) = \sigma t(1, u) = f(1, u) = u$, и $\sigma$ --- изоморфизм.
        \end{proof}
    
        \item Пусть $k = \mathbb{Z}$, $U = \mathbb{Z} / m\mathbb{Z}$, $V = \mathbb{Z} / n\mathbb{Z}$ и $(m, n) = 1$. Тогда $U \otimes_{\mathbb{Z}} V = \{0\}$.
        
        \begin{proof}
            $(m, n) = 1 \leadsto \exists r, s \in \mathbb{Z} \colon 1 = mr + ns \leadsto \forall u \in U, v \in V\ u \otimes v = 1 \cdot (u \otimes v) = r(mu) \otimes v + s \cdot u \otimes (nv) = r \cdot 0 \otimes v + s \cdot u \otimes 0 = 0$.
        \end{proof}
    \end{enumerate}
\end{exmpl} % Понятие тензорнго произведения модулей (определение, конструкция, следствия, примеры)
    \section{Тензорное произведение гомоморфизмов}

\begin{defn}
    $k$ --- коммутативное кольцо с единицей. Пусть $\alpha \colon U_1 \to U_2$ и $\beta \colon V_1 \to V_2$ --- гомоморфизмы $k$-модулей. Рассмотрим отображение $\alpha \times \beta \colon U_1 \times V_1 \to U_2 \times V_2$, т.ч. $(u, v) \mapsto (\alpha(u), \beta(v))$. Пусть $t_1 \colon U_1 \times V_1 \to U_1 \otimes_k V_1$ и $t_2 \colon U_2 \times V_2 \to U_2 \otimes_k V_2$ --- канонические билинейные отображения. Рассмотрим следующую диаграмму:
    \begin{diagram}
        U_1 \times V_1 & & \rTo^{t_1} & & U_1 \otimes_k V_1 \\
        & \rdTo_{t_2 \circ (\alpha \times \beta)} & & \ldDashto_{\alpha \otimes \beta} & \\
        & & U_2 \otimes_k V_2 & &
    \end{diagram}
    Ясно, что $t_2 \circ (\alpha \times \beta)$ билинейно. Тогда $\exists! \alpha \otimes \beta \colon U_1 \otimes_k V_1 \to U_2 \otimes_k V_2$, т.ч. $t_2 \circ (\alpha \times \beta) = (\alpha \otimes \beta) \circ t_1$. $\alpha \otimes \beta$ называется \textit{тензорным произведением гомоморфизмов}.
\end{defn}

\begin{thm}
    $\id_{U_1} \otimes \id_{V_1} = \id_{U_1 \otimes_k V_1}$.
\end{thm}

\begin{proof}
    Имеем
    \begin{align*}
        t_2 \circ (\id_{U_1} \times \id_{V_1}) &\colon (u, v) \mapsto (u, v) \mapsto u \otimes v \\
        \id_{U_1 \otimes_k V_1} \circ t_1 &\colon (u, v) \mapsto u \otimes v \mapsto u \otimes v
    \end{align*}
    откуда $\id_{U_1 \otimes_k V_1} = \id_{U_1} \otimes \id_{V_1}$.
\end{proof}

\begin{thm}
    Пусть даны $k$-гомоморфизмы:
    \begin{diagram}
        U_1 & \rTo^{\alpha_1} & U_2 & \rTo^{\alpha_2} & U_3 \\
        V_1 & \rTo^{\beta_1}  & V_2 & \rTo^{\beta_2}  & V_3
    \end{diagram}
    Тогда $(\alpha_2 \alpha_1) \otimes (\beta_2 \beta_1) = (\alpha_2 \otimes \beta_2) \circ (\alpha_1 \otimes \beta_1)$.
\end{thm}

\begin{proof}
    Построим следующую (заведомо коммутативную) диаграмму:
    \begin{diagram}
        & & U_1 \times V_1 & \rTo^{t_1} & U_1 \otimes V_1 & & \\
        & \ldTo(2,4)^{(\alpha_2 \alpha_1) \times (\beta_2 \beta_1)} & \dTo_{\alpha_1 \times \beta_1} & & \dTo_{\alpha_1 \otimes \beta_1} & & \\
        & & U_2 \times V_2 & \rTo^{t_2} & U_2 \otimes V_2 & & \\
        & \ldTo_{\alpha_2 \times \beta_2} & & & & \rdTo^{\alpha_2 \otimes \beta_2} \\
        U_3 \times V_3 & & & \rTo^{t_3} & & & U_3 \otimes V_3
    \end{diagram}
    Тогда $(\alpha_2 \otimes \beta_2) \circ (\alpha_1 \otimes \beta_1) \circ t_1 = t_3 \circ ((\alpha_2\alpha_1) \times (\beta_2 \beta_1))$, откуда $(\alpha_2 \alpha_1) \otimes (\beta_2 \beta_1) = (\alpha_2 \otimes \beta_2) \circ (\alpha_1 \otimes \beta_1)$.
\end{proof}

\begin{thm}
    Пусть $\alpha$ и $\beta$ --- изоморфизмы, тогда $\alpha \otimes \beta$ --- изоморфизм.
\end{thm}

\begin{proof}
    $(\alpha \otimes \beta) \circ (\alpha^{-1} \otimes \beta^{-1}) = (\alpha \alpha^{-1}) \otimes (\beta \beta^{-1}) = \id_{U_1} \otimes \id_{V_1} = \id_{U_1 \otimes V_1}$. Аналогично, $(\alpha^{-1} \otimes \beta^{-1}) \circ (\alpha \otimes \beta) = \id_{U_2 \otimes V_2}$.
\end{proof}

\begin{thm}
    $\alpha \otimes \beta = (\alpha \otimes \id_{V_1}) \circ (\id_{U_1} \otimes \beta) = (\id_{U_1} \otimes \beta) \circ (\alpha \otimes \id_{V_1})$.
\end{thm}

\begin{proof}
    Очевидно\footnote{На самом деле --- ни хера}.
\end{proof}

\begin{thm}
    Пусть $\alpha_1, \alpha_2 \in \lsub{K}{\hom(U_1, U_2)}, \beta \in \lsub{K}{\hom(V_1, V_2)}$. Тогда $(\alpha_1 + \alpha_2) \otimes \beta = \alpha_1 \otimes \beta + \alpha_2 \otimes \beta$.
\end{thm}

\begin{proof}
    $((\alpha_1 + \alpha_2) \otimes \beta) \circ t_1 = t_2 \circ ((\alpha_1 + \alpha_2) \times \beta) = t_2 \circ (\alpha_1 \times \beta) + t_2 \circ(\alpha_2 \times \beta) = (\alpha_1 \otimes \beta) \circ t_2 + (\alpha_2 \otimes \beta) \circ t_2 = (\alpha_1 \otimes \beta + \alpha_2 \otimes \beta) \circ t_2$, откуда $(\alpha_1 + \alpha_2) \otimes \beta = \alpha_1 \otimes \beta + \alpha_2 \otimes \beta$.
\end{proof} % Тензорное произведение гомоморфизмов
    \section{Аддитивность функтора тензорного произведения}

\begin{lem*}
    $R$ --- а.к. с единицей. Рассмотрим диаграмму из $R$-модулей и $R$-гомоморфизмов:
    \begin{equation}\tag{$\natural$}\label{10-03:becarre}
        \begin{diagram}
            U_1 & \pile{\rTo^{i_1} \\ \lTo_{p_1}} & V & \pile{\lTo^{i_2} \\ \rTo_{p_2}} & U_2
        \end{diagram}
    \end{equation}
    
    Предположим, что \begin{enumerate}
        \item $p_1 i_1 = \id_{U_1}$,
        \item $p_2 i_2 = \id_{U_2}$,
        \item $i_1 p_1 + i_2 p_2 = \id_V$.
    \end{enumerate}
    Тогда $\lsub{R}{V} \simeq U_1 \oplus U_2$.
\end{lem*}

\begin{proof}
    Построим пару отображений $\sigma \colon U_1 \oplus U_2 \to V$ и $\tau \colon V \to U_1 \oplus U_2$, т.ч. $\sigma(u_1, u_2) \coloneqq i_1(u_1) + i_2(u_2)$ и $\tau(v) \coloneqq (p_1(v), p_2(v))$. Тогда $\sigma \tau(v) = (i_1 p_1 + i_2 p_2)(v) = v$, а $\tau \sigma(u_1, u_2) = \tau(i_1(u_1) + i_2(u_2)) = \tau i_1(u_1) + \tau i_2(u_2) = (p_1 i_1(u_1), 0) + (0, p_2 i_2(u_2)) = (u_1, 0) + (0, u_2) = (u_1, u_2)$, откуда $\sigma$ --- искомый изоморфизм.
\end{proof}

\begin{defn}
    Диаграмму вида \eqref{10-03:becarre}, для которой выполняются вышеуказанные соотношения, называют \textit{диаграммой прямой суммы}.
\end{defn}

\begin{thm*}
    Пусть $U_1, U_2, V$ --- $k$-модули, где $k$ --- коммутативное кольцо с единицей. Тогда $(U_1 \oplus U_2) \otimes_k V \simeq (U_1 \otimes_k V) \oplus (U_2 \otimes_k V)$.
\end{thm*}

\begin{proof}
    Для $U_1 \oplus U_2$ рассмотрим диаграмму прямой суммы:
    \begin{diagram}
        U_1 & \pile{\rTo^{i_1} \\ \lTo_{p_1}} & U_1 \oplus U_2 & \pile{\lTo^{i_2} \\ \rTo_{p_2}} & U_2
    \end{diagram}
    Здесь: $$\begin{tabu}{cc}
        i_1(u) \coloneqq (u, 0) & p_1(u_1, u_2) \coloneqq u_1 \\
        i_2(u) \coloneqq (0, u) & p_2(u_1, u_2) \coloneqq u_2
    \end{tabu}$$
    Рассмотрим дополнительно: $$\begin{tabu}{cc}
        \tilde{i_1} \coloneqq i_1 \otimes \id_V & \tilde{p_1} \coloneqq p_1 \otimes \id_V \\
        \tilde{i_2} \coloneqq i_2 \otimes \id_V & \tilde{p_2} \coloneqq p_2 \otimes \id_V
    \end{tabu}$$
    и соответствующую диаграмму прямой суммы:
    \begin{diagram}
        U_1 \otimes V & \pile{\rTo^{\tilde{i_1}} \\ \lTo_{\tilde{p_1}}} & (U_1 \oplus U_2) \otimes V & \pile{\lTo^{\tilde{i_2}} \\ \rTo_{\tilde{p_2}}} & U_2 \otimes V
    \end{diagram}
    
    Тогда $\tilde{p_1}\tilde{i_1} = (p_1 \otimes \id_V) \circ (i_1 \otimes \id_V) = p_1 i_1 \otimes \id_V = \id_{U_1} \otimes \id_V = \id_{U_1 \otimes V}$; аналогично $\tilde{p_2}\tilde{i_2} = \id_{U_2 \otimes V}$. Имеем $\tilde{i_1}\tilde{p_1} + \tilde{i_2}\tilde{p_2} = (i_1 \otimes \id_V) \circ (p_1 \otimes \id_V) + (i_2 \otimes \id_V) \circ (p_2 \otimes \id_V) = (i_1 p_1) \otimes \id_V + (i_2 p_2) \otimes \id_V = (i_1 p_1 + i_2 p_2) \otimes \id_V = \id_{U_1 \oplus U_2} \otimes \id_V = \id_{(U_1 \oplus U_2) \otimes V}$. Таким образом, последняя рассмотренная диаграмма действительно является диаграммой прямой суммы, и указанный изоморфизм доказан.
\end{proof} % Аддитивность функтора тензорного произведения
    \section{Тензорное произведение свободных модулей}

\begin{thm*}
    Пусть $k$ --- коммутативное кольцо с единицей, $\lsub{k}{U}, \lsub{k}{V}$ --- свободные модули рангов $m \in \mathbb{N}$ и $n \in \mathbb{N}$ соответственно. Тогда $U \otimes_k V$ --- свободный $k$-модуль ранга $mn$.
\end{thm*}

\begin{proof}
    Ясно, что $\lsub{k}{U} \simeq k^{\oplus m}$, $\lsub{k}{V} \simeq k^{\oplus n}$. Тогда $U \otimes V \simeq k^{\oplus m} \otimes k^{\oplus n} \simeq (k \otimes k^{\oplus n})^{\oplus m} \simeq (k^{\oplus n})^{\oplus m} \simeq k^{\oplus mn}$.
\end{proof} % Тензорное произведение свободных модулей
    \section{Теорема об изоморфизме сопряженности}

\begin{defn}
    Пусть $k$ --- коммутативное кольцо с единицей, $U, V, W$ --- $k$ модули. Множество всех билинейных отображений $U \times V \to W$ будем обозначать $\Bi(U, V; W)$. На этом множестве введем структуру $k$-модуля через поточечное сложение и умножение на скаляр.
\end{defn}

\begin{thm*}
    Пусть $k$ --- коммутативное кольцо с единицей, $U, V, W$ --- $k$-модули. Тогда существуют изоморфизмы $k$-модулей:
    $$\lsub{k}{\hom (U \otimes_k V, W)} \stackrel{\lambda}{\to} \lsub{k}{\Bi(U, V, W)} \stackrel{\mu}{\to} \lsub{k}{\hom(U, \lsub{k}{\hom(V, W)})}$$
\end{thm*}

\begin{proof}
    \begin{proofpart}
        Построим $\lambda$. Пусть $g \in \hom(U \otimes V, W)$, $t \colon U \times V \to U \otimes V$ --- каноническое билинейное отображение. Рассмотрим следующую диаграмму:
        \begin{diagram}
            U \times V & & \rTo^t & & U \otimes V \\
            & \rdDashto_{g \circ t} & & \ldTo_{g} & \\
            & & W & &
        \end{diagram}
        Так как $t$ билинейно, а $g$ --- гомоморфизм, то $g \circ t$ билинейно. Положим $\lambda(g) \coloneqq g \circ t$. $\lambda$ обратимо по универсальному свойству тензорного произведения. Ясно, что $\lambda$ является $k$-гомоморфизмом, следовательно $\lambda$ --- изоморфизм.
    \end{proofpart}

    \begin{proofpart}
        Построим $\mu$. Пусть $f \in \Bi(U, W; W)$. Положим $(\mu(f)(u))(v) \coloneqq f(u, v)$. $\mu(f)(u)$ и $\mu(f)$ являются $k$-гомоморфизмами вследствие билинейности $f$. Также ясно, что $\mu$ --- $k$-гомоморфизм. Рассмотрим $\nu \colon \hom(U, \hom(V, W)) \to \Bi(U, V; W)$, т.ч. $(\nu(h))(u, v) \coloneqq (h(u))(v)$. Ясно, что $\nu(h)$ билинейно, а $\nu$ является $k$-гомоморфизмом. Тогда:
        \begin{align*}
            (\nu \mu(f))(u, v) = ((\mu(f))(u))(v) = f(u, v) &\leadsto \nu \mu = \id_{\Bi(U, V; W)} \\
            (\mu \nu(h))(u)(v) = (\nu(h))(u, v) = h(u)(v) &\leadsto \mu \nu = \id_{\hom(U, \hom(V, W))}
        \end{align*}
        и $\mu$ --- изоморфизм.
    \end{proofpart}
\end{proof}

\begin{defn}
    Отображение $\mu \lambda$ называется \textit{изоморфизмом сопряженности}.
\end{defn} % Теорема об изоморфизме сопряженности
    \section{Базис тензорного произведения линейных пространств}

\begin{thm*}
    $K$ --- поле, $U, V$ --- конечномерные линейные пространства. Пусть $B_1 = \family{u_i}{i=1}{m}$ и $B_2 = \family{v_i}{i=1}{n}$ --- базисы $U$ и $V$ соответственно. Тогда $B_1 \otimes B_2 \coloneqq \family{u_i \otimes v_j}{i=1..m,\ j=1..n}{}$ --- базис $U \otimes_K V$.
\end{thm*}

\begin{proof}
    Известно, что $U \otimes V = \gen{K}{\{u \otimes v \mid u \in U, v \in V\}}$. Имеем $u \otimes v = \left(\sum_{i=1}^{m} \alpha_i u_i\right) \otimes \left(\sum_{j=1}^{n} \beta_j v_j \right) = \sum_{i,j=1}^{m,n} \alpha_i \beta_j (u_i \otimes v_j)$. Таким образом, $U \otimes V = \gen{K}{B_1 \otimes B_2}$. При этом известно, что $\dim \lsub{K}{(U \otimes_K V)} = mn = \#(B_1 \otimes B_2)$. Следовательно, $B_1 \otimes B_2$ --- базис.
\end{proof} % Базис тензорного произведения линейных пространств
    \section{Кронекерово произведение матриц, связь с тензорным произведением линейных операторов. Кронекерово произведение матриц Адамара}

\begin{defn}
    Пусть $A = (a_{ij}) \in M_{m,n}(K)$, $C = (c_{ij}) \in M_{k,l}(K)$. \textit{Кронекерово произведение} $A$ и $C$ --- это матрица $A \otimes C \in M_{mn,kl}(K)$, допускающая разбиение на $k \times l$-блоки, т.ч. $A \otimes C = \family{B_{ij}}{i,j=1}{m,n}$, где $B_{ij} = a_{ij}C$. 
\end{defn}

\begin{rem}
    Матрицу $A \otimes C$ можно явно представить в виде:
    $$(A \otimes C)[i, j] = A\left[ \left\lfloor \frac{i-1}{k} \right\rfloor + 1,\ \left\lfloor \frac{j-1}{l} \right\rfloor + 1 \right] \cdot C\left[ i - k \cdot \left\lfloor \frac{i-1}{k} \right\rfloor,\ j - l \cdot \left\lfloor \frac{j-1}{l} \right\rfloor \right]$$
    или
    $$(A \otimes C)[i, j] = A[(i - 1) \divop k + 1, (j - 1) \divop l + 1] \cdot C[(i - 1) \modop k + 1, (j - 1) \modop l + 1]$$
    где $a \divop b \coloneqq \left\lfloor \frac{a}{b} \right\rfloor$, $a \modop b \coloneqq a - b \cdot (a \divop b)$.
\end{rem}

\begin{thm}
    Пусть $U$ и $V$ --- $K$-линейные пространства, $a$ и $c$ --- операторы на $U$ и $V$ соответственно. Пусть дополнительно $B = \family{u_i}{i=1}{m}$ и $B' = \family{v_j}{j=1}{n}$ --- базисы $\lsub{K}{U}$ и $\lsub{K}{V}$ соответственно. Положим $A = (a_{ij}) \coloneqq [a]_B$, $B = (b_{ij}) \coloneqq [c]_B$. Упорядочим $B \otimes B'$ по строкам: $((1, 1), \dots, (1, n), (2, 1), \dots, \dots, (m, n))$. Тогда $[a \otimes c]_{B \times B'} = A \otimes C$.
\end{thm}

\begin{proof}
    Обозначим $B \otimes B' = \family{w_t}{t=1}{mn} = \family{u_i \otimes v_j}{i,j=1}{m,n}$. Заметим, что $i$ и $j$ можно выразить через $t$:
    \begin{equation}\label{10-07:1}
        i = \left\lfloor \frac{t-1}{n} \right\rfloor + 1, \quad
        j = t - n \left\lfloor \frac{t-1}{n} \right\rfloor
    \end{equation}
    Имеем $(a \otimes c)(w_t) = (a \otimes c)(u_i \otimes v_j) = a(u_i) \otimes c(v_j) = \left(\sum_{k=1}^m a_{ki} u_k\right) \otimes \left(\sum_{l=1}^n c_{lj} v_l\right) = \sum_{k,l=1}^{m,n} a_{ki}c_{lj} (u_k \otimes v_l)$. Но $u_k \otimes v_l = w_s$, где $s$ удовлетворяет соотношениям:
    \begin{equation}\label{10-07:2}
        k = \left\lfloor \frac{s-1}{n} \right\rfloor + 1, \quad
        l = s - n \left\lfloor \frac{s-1}{n} \right\rfloor
    \end{equation}
    Так, $([a \otimes c]_{B \otimes B'})[s, t] = a_{ki}c_{lj} \stackrel{\text{\eqref{10-07:1} и \eqref{10-07:2}}}{=} (A \otimes C)[s, t]$.
\end{proof}

\begin{defn}
    $C \in M_n(\mathbb{Z})$ называется \textit{матрицей Адамара}, если она состоит из $\pm 1$ и $C^T C = nE_n$.
\end{defn}

\begin{exmpl}
    \[
    C = \begin{pmatrix}
        1 &  1 \\
        1 & -1
    \end{pmatrix}, \qquad
    C = \begin{pmatrix}
        1 &  1 &  1 &  1 \\
        1 & -1 &  1 & -1 \\
        1 &  1 & -1 & -1 \\
        1 & -1 & -1 &  1
    \end{pmatrix}
    \]
\end{exmpl}

\begin{thm}
    Пусть $A$ и $B$ --- матрицы Адамара размеров $m$ и $n$ соответственно, тогда $A \otimes B$ --- матрица Адамара размера $mn$.
\end{thm}

\begin{proof}
    Ясно, что $A \otimes B$ состоит из $\pm 1$. Она имеет вид:
    $$A \otimes B = \begin{pmatrix}
        a_{11} B & \cdots & a_{1m} B \\
        \vdots   & \ddots & \vdots   \\
        a_{m1} B & \cdots & a_{mm} B
    \end{pmatrix}$$
    При этом $((A \otimes B)^T (A \otimes B))\overbrace{[i, j]}^{\text{блок}} = \sum_{k=1}^m a_{ki} a_{kj} B^TB = (M^TM)[i, j] \cdot N^TN = m \delta_{ij} N^TN$, т.е. $(A \otimes B)^T(A \otimes B)$ имеет блочно-диагональную форму. При этому $(N^TN)[k, l] = n\delta_{kl}$, то есть каждый блок диагонален. Ясно, что каждый диагональный элемент имеет значение $mn$, и $A \otimes B$ --- матрица Адамара.
\end{proof} % Кронекерово произведение матриц, связь с тензорным произведением линейных операторов. Кронекерово произведение матриц Адамара
    \section{Дуальные пространства и тензорное произведение}

\begin{thm*}
    Пусть $U$ и $V$ --- конечномерные $K$-пространства. Тогда:
    \begin{enumerate}
        \item Существует изоморфизм $\sigma \colon U^* \otimes V^* \to (U \otimes V)^*$, т.ч. $\sigma(\varphi \otimes \psi)(u \otimes v) = \varphi(u) \psi(v)$.
        \item Существует изоморфизм $\tau \colon U^* \otimes V \to \lsub{K}{\hom(U, V)}$, т.ч. $\tau(\varphi \otimes v)(u) = \varphi(u) \cdot v$.
    \end{enumerate}
\end{thm*}

\begin{proof}
    \begin{proofpart}
        Пусть $\varphi \in U^*$, $\psi \in V^*$. Рассмотрим изображение $\tilde{\alpha}_{\phi,\psi} \colon U \times V \to K$, т.ч. $(u, v) \mapsto \varphi(u)\psi(v)$. Рассмотрим следующую диаграмму:
        \begin{diagram}
            U \times V & & \rTo^\otimes & & U \otimes V \\
            & \rdTo_{\tilde{\alpha}_{\varphi,\psi}} & & \ldDashto_{\alpha_{\varphi, \psi}} \\
            & & K & &
        \end{diagram}
        По универсальному свойству тензорного произведения $\exists! \alpha_{\varphi, \psi} \in (U \otimes V)^*$, т.ч. $u \otimes v \mapsto \tilde{\alpha}_{\varphi, \psi}(u, v) = \varphi(u) \psi(v)$.
        
        Введем отображение $\sigma_0 \colon U^* \times V^* \to (U \otimes V)^*$, т.ч. $(\varphi, \psi) \mapsto \alpha_{\varphi, \psi}$. Рассмотрим следующую диаграмму:
        \begin{diagram}
            U^* \times V^* & & \rTo^\otimes & & U^* \otimes V^* \\
            & \rdTo_{\sigma_0} & & \ldDashto_{\sigma} \\
            & & (U \otimes V)^* & &
        \end{diagram}
        Ясно, что $\exists! \sigma \colon U^* \otimes V^* \to (U \otimes V)^*$, т.ч. $\varphi \otimes \psi \mapsto \sigma_0(\varphi, \psi) = \alpha_{\varphi, \psi}$. Следовательно, $\sigma(\varphi \otimes \psi)(u \otimes v) = \alpha_{\varphi, \psi}(u \otimes v) = \tilde{\alpha}_{\varphi, \psi}(u, v) = \varphi(u)\psi(v)$.
        
        Покажем, что $\sigma$ --- эпиморфизм. Пусть $B_1 = \family{u_i}{i=1}{m}$ и $B_2 = \family{v_j}{j=1}{n}$ --- базисы $\lsub{K}{U}$ и $\lsub{K}{V}$, а $B_1^* = \family{\varphi_i}{i=1}{m}$ и $B_2^* = \family{\psi_j}{j=1}{n}$ --- базисы $\lsub{K}{U^*}$ и $\lsub{K}{V^*}$. Пусть $B_1 \otimes B_2$ --- базис $U \otimes V$. 
        
        Рассмотрим $B_1^* \otimes B_2^*$. Имеем $\sigma(\varphi_k \otimes \psi_l)(u_i \otimes v_j) = \varphi_k(u_i) \psi_l(v_j) = \delta_{ki} \delta_{lj} = \delta_{(i, j);(k, l)}$. Из этого следует, что $\sigma(B_1^* \otimes B_2^*)$ --- дуальный к $B_1 \otimes B_2$ базис. Тогда $\im \sigma \supset \gen{K}{\sigma(B_1^* \otimes B_2^*)} = (U \otimes V)^*$, т.е. $\sigma$ эпиморфно. Так как $\dim \lsub{K}{(U^* \otimes V^*)} = \dim \lsub{K}{(U \otimes V)^*}$, то $\sigma$ --- изоморфизм.
    \end{proofpart}

    \begin{proofpart}
        Пусть $\varphi \in U^*$, $v \in V$. Рассмотрим отображение $\tilde{\tau} \colon U^* \times V \to \lsub{K}{\hom(U, V)}$, т.ч. $\tilde{\tau}(\varphi, v)(u) = \varphi(u) v$. Рассмотрим следующую диаграмму:
        \begin{diagram}
            U^* \times V & & \rTo^\otimes & & U^* \otimes V \\
            & \rdTo_{\tilde{\tau}} & & \ldDashto_{\tau} & \\
            & & \hom(U, V) & &
        \end{diagram}
        Видно, что $\exists! \tau \colon U^* \otimes V \to \hom(U, V)$, т.ч. $\tau(\varphi \otimes v) = \tilde{\tau}(\varphi, v)$. Покажем, что $\tau$ --- эпиморфизм. Пусть $f \in \hom(U, V)$. Воспользуемся базисом из предыдущего пункта. Пусть $[f]_{B,B'} = (\alpha_{ij})$. Тогда:
        $$f(u_j) = \sum_{i} a_{ij} v_i = \sum_{i} a_{ij} \sum_{k} \varphi_k(u_j) v_i = \sum_{i,k} a_{ij} \tau(\varphi_k \otimes v_i)(u_i) = \tau\left(\sum_{i,k} a_{ij}(\varphi_k \otimes v_i)\right)(u_i)$$
        Таким образом, $\tau$ --- эпиморфизм, а следовательно, и изоморфизм.
    \end{proofpart}
\end{proof} % Дуальные пространства и тензорное произведение
    \section{Согласованные замены базисов в линейном пространстве и пространстве функционалов. Контраградиентная матрица}

\begin{thm*}
    Пусть $B_1 = \family{u_i}{i=1}{n}$ и $B_2 = \family{v_j}{j=1}{n}$ --- базисы $K$-пространства $L$, $B^*_1 = \family{\varphi^i}{i=1}{n}$ и $B^*_2 = \family{\psi^j}{j=1}{n}$ --- соответствующие дуальные базисы $L$. Пусть $B_1 \stackrel{C}{\leadsto} B_2$. Тогда $B_1^* \stackrel{C^{-1T}}{\leadsto} B_2^*$.
\end{thm*}

\begin{proof}
    Докажем, что $B_2^* \stackrel{C^T}{\leadsto} B_1^*$. Пусть $B_2^* \stackrel{D = (d_{ij})}{\leadsto} B_1^*$. Тогда $\varphi^i(v_j) = \varphi^i\left(\sum_k c_{kj} u_k\right) = \sum_k c_{kj} \varphi^i(u_k) = c_{ij}$. С другой стороны, $\varphi^i(v_j) = \sum_l d_{li} \psi^l(v_j) = d_{ji}$, откуда $d_{ij} = c_{ji}$ и $D = C^T$.
\end{proof}

\begin{defn}
    Для $C \in M_n(K)^*$ матрица $\hat{C} \coloneqq C^{-1T}$ называется \textit{контраградиентной} к $C$.
\end{defn} % Согласованные замены базисов в линейном пространстве и пространстве функционалов. Контраградиентная матрица
    \section{Координаты тензора, классическое определение тензора}

\begin{defn}
    $L$ --- конечномерное $K$-пространство, $(p, q) \in \mathbb{Z}_+^2 \setminus \{(0, 0)\}$. Положим $T_p^q \coloneqq T_p^q(L) \coloneqq (L^*)^{\otimes p} \otimes L^{\otimes q}$. Элементы пространства $T_p^q$ называются \textit{$p$ раз ковариантными, $q$ раз контравариантными тензорами}. Число $p+q$ называется \textit{валентностью} $T \in T_p^q$. 
\end{defn}

\begin{rem}
    Справедливы следующие утверждения:
    \begin{enumerate}
        \item $\dim \lsub{K}{T_p^q(L)} = (\dim \lsub{K}{L})^{p+q}$.
        \item Существует цепочка изоморфизмов:
        $$T_p^q(L) \stackrel{f}{\longrightarrow} (L^*)^{\otimes p} \otimes (L^{**})^{\otimes q} \stackrel{\sigma}{\longrightarrow} (L^{\otimes p} \otimes (L^*)^{\otimes q})^* \stackrel{\lambda}{\longrightarrow} \lsub{K}{\Poly(L^p, (L^*)^q; K)}$$
        такая, что:
        \begin{enumerate}[label=\arabic*.]
            \item $f(\varphi_1 \otimes \ldots \otimes \varphi_p \otimes u_1 \otimes \ldots \otimes u_q) = \varphi_1 \otimes \ldots \otimes \varphi_p \otimes \delta_{u_1} \otimes \ldots \otimes \delta_{u_q}$, где $\delta_{u_i}(\varepsilon) = \varepsilon(u_1)$.
            \item $\sigma(\varphi_1 \otimes \ldots \otimes \varphi_p \otimes \psi_1 \otimes \ldots \otimes \psi_q)(u_1 \otimes \ldots \otimes u_p \otimes g_1 \otimes \ldots \otimes g_q) = \varphi_1(u_1) \cdot \ldots \cdot \varphi_p(u_p) \cdot \psi_1(g_1) \cdot \ldots \cdot \psi_q(g_q)$.
            \item $\lambda(\varepsilon)(u_1, \ldots, u_p, \varphi_1, \ldots, \varphi_q) = \varepsilon(u_1 \otimes \ldots \otimes u_p \otimes \varphi_1 \otimes \ldots \otimes \varphi_q)$.
        \end{enumerate}
    
        Положим $\theta = \lambda \sigma f$. Тогда справедлива формула:
        \begin{align*}
            &\phantom{{}={}} \theta(\varphi_1 \otimes \ldots \otimes \varphi_p \otimes u_1 \otimes \ldots \otimes u_q)(\tilde{u_1}, \ldots, \tilde{u_p}, \tilde{\varphi_1}, \ldots, \tilde{\varphi_q}) \\
            &= \lambda \circ \sigma(\varphi_1 \otimes \ldots \otimes \varphi_p \otimes \delta_{u_1} \otimes \ldots \otimes \delta_{u_j})(\tilde{u_1}, \ldots, \tilde{u_p}, \tilde{\varphi_1}, \ldots, \tilde{\varphi_q}) \\
            &= \sigma(\varphi_1 \otimes \ldots \otimes \varphi_p \otimes \delta_{u_1} \otimes \ldots \otimes \delta_{u_q})(\tilde{u_1} \otimes \ldots \otimes \tilde{u_p} \otimes \tilde{\varphi_1} \ldots \otimes \tilde{\varphi_q}) \\
            &= \varphi_1(\tilde{u_1}) \cdot \ldots \cdot \varphi_p(\tilde{u_p}) \cdot \delta_{u_1}(\tilde{\varphi_1}) \cdot \ldots \cdot \delta_{u_q}(\tilde{\varphi_q}) \\
            &= \varphi_1(\tilde{u_1}) \cdot \ldots \cdot \varphi_p(\tilde{u_p}) \cdot \tilde{\varphi_1}(u_1) \cdot \ldots \cdot \tilde{\varphi_q}(u_q)
        \end{align*}
    \end{enumerate}
\end{rem}

\begin{defn}
    Пусть $L$ --- $K$-линейное пространство, $T \in T_p^q(L)$. Пусть $B = \family{u_i}{i=1}{n}$ --- базис $\lsub{K}{L}$, $B^* = \family{\varphi^i}{i=1}{n}$ --- дуальный к нему. Положим $\tilde{B} \coloneqq (B^*)^{\otimes p} \otimes B^{\otimes q}$ --- базис $\lsub{K}{T_p^q(L)}$. Рассмотрим разложение $T$ по базисным тензорным произведениям из $\tilde{B}$:
    $$T = T_{i_1, \dots, i_p}^{j_1, \dots, j_p} \cdot \varphi^{i_1} \otimes \ldots \otimes \varphi^{i_p} \otimes u_{j_1} \otimes \ldots \otimes u_{j_q}$$
    Набор коэффициентов $\{T_{i_1, \dots, i_p}^{j_1, \dots, j_q}\}$ называется \textit{координатами (компонентами)} тензора $T$ относительно базиса $B$.
\end{defn}

\begin{thm*}
    Пусть $L$ --- $K$-пространство, $B_1 = \family{u_1}{i=1}{n}$ и $B_2 = \family{v_i}{i=1}{n}$ --- базисы $\lsub{K}{L}$. Пусть $B \stackrel{C}{\leadsto} B'$, $C = (c_j^i)$, $\hat{C} = (\hat{c}_i^j)$. Предположим $\{T_{i_1, \dots, i_p}^{j_1, \dots, j_q}\}$ и $P_{r_1, \dots, r_p}^{s_1, \dots, s_q}$ --- компоненты $T \in T_p^q(L)$ относительно $B_1$ и $B_2$ соответственно. Тогда:
    $$P_{r_1, \dots, r_p}^{s_1, \dots, s_q} = T_{i_1, \dots, i_p}^{j_1, \dots, j_q} \cdot c_{r_1}^{i_1} \cdot \ldots \cdot c_{r_p}^{i_p} \cdot \hat{c}_{j_1}^{s_1} \cdot \ldots \cdot \hat{c}_{j_q}^{s_q}$$
\end{thm*}

\begin{proof}
    Имеем $\hat{c}_k^j c_j^i = \sum_j \hat{C}[k, j] C[i, j] = \sum_j C^{-1}[j, k] C[i, j] = (C^{-1}C)[k, i] = \delta_k^i$. В то же время, $v_j = c_j^i u_i$ по определению $C$. Имеем:
    \begin{equation}\label{10-10:1}
        \hat{c}_k^j v_j = \hat{c}_k^j v_j = \hat{c}_k^j c_j^i u_i = \delta_k^i u_i = u_k
    \end{equation}
    
    Пусть базисы $B_1^* = \family{\varphi^j}{j=1}{n}$ и $B_2^* = \family{\psi^j}{j=1}{n}$ --- дуальные к $B_1$ и $B_2$ соответственно. По теореме о согласованной замене базисов $\lsub{K}{L}$ и $\lsub{K}{L^*}$ имеем $\psi^j = \hat{c}_i^j \varphi^{i}$. В то же время, $c_j^k \hat{c}_i^j = \sum_j C[k, j] \hat{C}[i, j] = \sum_j C[k, j] C^{-1}[j, i] = (CC^{-1})[k,i] = \delta_i^k$. Тогда:
    \begin{equation}\label{10-10:2}
        c_j^k \psi^j = c_j^k \hat{c}_i^j \varphi^i = \delta_i^k \varphi^i = \varphi^k
    \end{equation}
    
    Рассмотрим разложение $T$ по базису $B_1$:
    \begin{equation}\label{10-10:3}
        T = T_{i_1, \dots, i_p}^{j_1, \dots, j_p} \cdot \varphi^{i_1} \otimes \ldots \otimes \varphi^{i_p} \otimes u_{j_1} \otimes \ldots \otimes u_{j_q}
    \end{equation}
    Подставим формулы \eqref{10-10:1} и \eqref{10-10:2} в \eqref{10-10:3}. Тогда
    \begin{equation}\label{10-10:4}
       T = T_{i_1, \dots, i_p}^{j_1, \dots, j_p} \cdot (c_{r_1}^{i_1} \psi^{r_1}) \otimes \ldots \otimes (c_{r_p}^{i_p} \psi^{r_p}) \otimes (\hat{c}_{j_1}^{s_1} v_{s_1}) \otimes \ldots \otimes (\hat{c}_{j_q}^{s_q} v_{s_q})
    \end{equation}
    откуда искомое равенство ясно по полилинейности.
\end{proof} % Координаты тензора, классическое определение тензора
\end{document}
