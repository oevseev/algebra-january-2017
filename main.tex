\documentclass{scrartcl}

\usepackage[utf8]{inputenc}
\usepackage[T1, T2A]{fontenc}
\usepackage[russian]{babel}

% LaTeX preamble for algebra named after Generalov A.I.
% (c) 2016, 2017 Oleg Evseev. For in-house use only.

\usepackage{amsmath}
\usepackage{amsfonts}
\usepackage{amssymb}
\usepackage{amsthm}
\usepackage{enumitem}
\usepackage{graphicx}
\usepackage{leftidx}
\usepackage{mathtools}
\usepackage{tikz}
\usepackage{marvosym} % What the fuck
\usepackage{tabu}

% For use with Apple Watch
\usepackage{xcolor}
\usepackage{pagecolor}

% diagrams.sty can be found at http://www.paultaylor.eu/diagrams/
\usepackage[small,nohug,heads=vee]{diagrams}
\diagramstyle[labelstyle=\scriptstyle]

% Structural defines

\newtheorem{thm}{Теорема}
\newtheorem*{thm*}{Теорема}
\newtheorem{prop}[thm]{Предложение} 
\newtheorem*{prop*}{Предложение}
\newtheorem{cor}{Следствие}
\newtheorem*{cor*}{Следствие}

\theoremstyle{definition}
\newtheorem*{defn}{Определение}

\theoremstyle{remark}
\newtheorem*{rem}{Замечание}
\newtheorem*{exmpl}{Пример}

\newtheoremstyle{lemma}{}{}{}{}{\bfseries}{.}{.5em}{}
\theoremstyle{lemma}
\newtheorem{lem}{Лемма}
\newtheorem*{lem*}{Лемма}

\newtheoremstyle{part}{}{}{}{}{\bfseries}{)}{.5em}{}
\theoremstyle{part}
\newtheorem{proofpart}{}

\makeatletter
\@addtoreset{equation}{section}
\@addtoreset{thm}{subsection}
\@addtoreset{cor}{thm}
\@addtoreset{proofpart}{thm}
\@addtoreset{proofpart}{lem}
\@addtoreset{proofpart}{cor}
\makeatother

\renewcommand{\thesection}{\arabic{part}.\arabic{section}}
\renewcommand*{\theproofpart}{\asbuk{proofpart}}
\renewcommand*{\thelem}{\Asbuk{lem}}

\renewcommand{\theenumi}{(\asbuk{enumi})}
\renewcommand{\labelenumi}{\asbuk{enumi})}

\AddEnumerateCounter{\Asbuk}{\@Asbuk}{\CYRM}
\AddEnumerateCounter{\asbuk}{\@asbuk}{\cyrm}

% Some useful declarations

\DeclareMathOperator{\End}{End}
\DeclareMathOperator{\id}{id}
\DeclareMathOperator{\im}{Im}
\DeclareMathOperator{\diag}{diag}
\DeclareMathOperator{\rk}{rk}
\DeclareMathOperator{\spec}{sp}
\DeclareMathOperator{\Bi}{Bi}

\renewcommand{\hom}{\operatorname{Hom}}
\renewcommand{\ker}{\operatorname{Ker}}

\newcommand\lsub[2]{\leftidx{_{#1}}{#2}}
\newcommand\family[3]{\left\{#1\right\}_{#2}^{#3}}
\newcommand\gen[2]{\lsub{#1}{\left\langle #2 \right\rangle}}

\newcommand\divby{\mathrel{\vdots}}
\newcommand\ndivby{\mathrel{\not\vdots}}

\newcommand\nil{\mathbf{0}}
\newcommand\nilset{\{\nil\}}
\newcommand{\norm}[1]{\left\lVert#1\right\rVert}

\newcommand{\inup}{\mathbin{\rotatebox[origin=c]{90}{$\in$}}}

\newcommand*\circled[1]{\tikz[baseline=(char.base)]{
    \node[shape=circle,draw,inner sep=2pt] (char) {#1};}}
    
% Thanks to http://texblog.net/latex-archive/maths/amsmath-matrix/
\makeatletter
\renewcommand*\env@matrix[1][*\c@MaxMatrixCols c]{%
    \hskip -\arraycolsep
    \let\@ifnextchar\new@ifnextchar
    \array{#1}}
\makeatother

% Part names on separate pages, found here:
% https://tex.stackexchange.com/questions/64215/make-part-in-scrartcl
\renewcommand\partheadstartvskip{\clearpage\null\vfil}
\renewcommand\partheadmidvskip{\par\nobreak\vskip 20pt\thispagestyle{empty}}
\renewcommand\partheadendvskip{\vfil\clearpage}
\renewcommand\raggedpart{\centering}


% Use either, not both
\usepackage[top=2cm, bottom=2cm, left=2cm, right=2cm]{geometry} % Standard
% % Apple Watch
\pagecolor{black}
\color{white}
\usepackage[paperwidth=120mm, paperheight=150mm, top=5mm, bottom=5mm, left=5mm, right=5mm]{geometry}

%\sloppy
 % Apple Watch

\title{Ответы на вопросы экзамена по АТЧ}
\subtitle{(январь 2017 г., 3-й семестр ПМиИ)}
\author{А. Константинов, О. Евсеев, Г. Енгалыч}

\begin{document}
    \pagenumbering{gobble}
    \thispagestyle{empty}
    \maketitle
    
    \begin{center}
        \vspace{120pt}
        \includegraphics[width=32em]{cat.png}
    \end{center}    

    \setcounter{part}{8}
    
    % Глава IX. Евклидовы и унитарные пространства
    \part{Евклидовы и унитарные пространства}
    \section{Определение евклидова пространства. Неравенства Коши и треугольника в ЕП}

\begin{defn}
    \textit{Евклидово пространство} --- пара $(\lsub{\mathbb{R}}{L}, (-, -))$, где $\lsub{\mathbb{R}}{L}$ --- \textbf{конечномерное} пространство, $(-, -) \colon L^2 \to \mathbb{R}$ --- положительно определенная билинейная форма.
    
    Величина $||u|| \coloneqq \sqrt{(u, u)}$ называется \textit{длиной (нормой) вектора $u$}.
    
    Величина $\rho(u, v) = ||u - v||$ называется \textit{расстоянием} между $u$ и $v$.
\end{defn}

\begin{thm*}
    $L$ --- ЕП. Пусть $u, v \in L$. Тогда:
    \begin{enumerate}
        \item $|(u, v)| \le ||u|| \cdot ||v||$
        \item $||u + v|| \le ||u|| + ||v||$
        \item $\rho(u, v) \le \rho(u, w) + \rho(w, v)\ \forall w \in L$
    \end{enumerate}
\end{thm*}

\begin{proof}
    \begin{proofpart}
        Пусть $t \in \mathbb{R}$. Считаем, что $u \neq \nil \neq v$. Тогда $(u + tv, u + tv) = (u, u) + 2(u, tv) + (tv, tv) = {||u||}^2 + 2t(u, v) + t^2{||v||}^2 > 0\ \forall t$, откуда $D = 4(u, v)^2 - 4{||u||}^2{||v||}^2 \le 0$ и $|(u, v)| \le ||u|| \cdot ||v||$.
    \end{proofpart}

    \begin{proofpart}
        ${||u + v||}^2 = (u + v, u + v) = {||u||}^2 + 2(u, v) + {||v||}^2 \le {||u||}^2 + 2||u|| \cdot ||v|| + {||v||}^2 = (||u|| + ||v||)^2$
    \end{proofpart}

    \begin{proofpart}
        $\rho(u, v) = ||u - v|| = ||(u - w) + (w - v)|| \le ||u - w|| + ||w - v|| = \rho(u, w) + \rho(w, v)$
    \end{proofpart}
\end{proof} % Определение евклидова пространства. Неравенства Коши и треугольника в ЕП
    \section{Матрица Грама набора векторов. Связь с линейной независимостью векторов}

\begin{defn}
    Пусть $M = \family{v_i}{i=1}{m}$ --- семейство векторов ЕП $L$. Матрица $G_M = ((v_i, v_j))_{ij}$ (где $i, j \in 1..m$) называется \textit{матрицей Грама} семейства $M$.
\end{defn}

\begin{rem}
    Ясно, что $G_M$ симметрична. Если $M = \family{e_i}{i=1}{m}$ --- базис, то $G_M = (g_{ij})$ является матрицей скалярного произведения (то есть, для $u, v \in L$ скалярное произведение выражается в виде $(u, v) = U^TG_MV$ в силу того, что $(u, v) = \sum_{i, j = 1}^m g_{ij} u_i v_j$).
\end{rem}

\begin{thm*}
    $L$ --- ЕП, $M = \family{v_i}{i=1}{m}$ --- семейство векторов $L$. $M$ линейно независимо $\Leftrightarrow$ матрица Грама $G_M$ обратима.
\end{thm*}

\begin{proof}
    \begin{proofpart}{($\Rightarrow$)}
        Предположим, что $G_M$ необратима. Тогда уравнение $G_M X = \nil$ имеет ненулевое решение $X = (x_1, \ldots, x_m)^T \neq \nil$. Положим $u \coloneqq \sum_{i=1}^m x_i v_i \neq \nil$ (по линейной независимости) $\Rightarrow (u, u) = X^T G_M X = X^T \cdot \nil = \nil \Rightarrow u = \nil$. Полученное противоречие доказывает обратимость $G_M$.
    \end{proofpart}

    \begin{proofpart}{($\Leftarrow$)}
        Предположим, что $\sum_{i=1}^m x_i v_i = \nil$. Домножив скалярно обе части на вектор $v_j$, получим $\sum_{i=1}^m x_i g_{ij} = 0\ \forall j$. Отсюда $X^T G_M = \nil$. После домножения справа на $G_M^{-1}$ получаем $X^T = \nil \Rightarrow \forall i\ x_i = 0$. Отсюда $M$ линейно независимо.
    \end{proofpart}
\end{proof} % Матрица Грама набора векторов. Связь с линейной независимостью векторов
    \section{Теорема о существовании ОНБ в ЕП}

\begin{thm*}
    Пусть $L$ --- евклидово пространство. Тогда в нем существует ортонормированный базис $\family{v_i}{i=1}{n}$.
\end{thm*}

\begin{proof}
    Пусть $\family{u_i}{i=1}{n}$ --- \textbf{произвольный} базис $L$. Произведем индукцию по $n$.
    \smallskip
    
    \underline{$n = 1$}. $\{v_1 = \frac{u_1}{||u_1||}\}$ --- ОН-базис.
    
    \underline{$n-1 \rightarrow n$}. Пусть $U \coloneqq \gen{}{u_1, \ldots, u_{n-1}} \le L$ --- $(n-1)$-мерное подпространство $L$. По индукционному предположению в нем можно выбрать ОН-базис $\family{v_i}{i=1}{n-1}$.
    
    Пусть $v \coloneqq u_n - \sum_{i=1}^{n-1} (u_n, v_i) v_i$. Тогда $\forall k = 1..n-1$ имеем:
    $$(v, v_k) = \left(u_n - \sum_{i=1}^{n-1} (u_n, v_i)v_i, v_k\right) = (u_n, v_k) - \sum_{i=1}^{n-1} (u_n, v_i)(v_i, v_k) \stackrel{\text{ОН-ть}}{=} (u_n, v_k) - (u_n, v_k) \underbrace{(v_k, v_k)}_{=1} = 0$$
    Можем дополнить ОН-базис $U$ до ОН-базиса $V$, положив $v_n = \frac{v}{||v||}$.
\end{proof} % Теорема о существовании ОНБ в ЕП
    \section{Матрица перехода между ОНБ в ЕП}

\begin{thm*}
    $L$ --- ЕП. Пусть $B = \family{u_i}{i=1}{n}$ и $B' = \family{u'_i}{i=1}{n}$ --- ОНБ, причем $B \stackrel{C}{\leadsto} B'$. Тогда $C = (c_{ij})$ ортогональна.
\end{thm*}

\begin{proof}
    $$\delta_{ij} = (u'_i, u'_j) = \left(\sum_{k=1}^n c_{ki} u_k, \sum_{t=1}^n c_{tj} u_t\right) = \sum_{k,t=1}^n c_{ki} c_{tj} \underbrace{(u_k, u_t)}_{\delta_{kt}} = \sum_{s=1}^n \underbrace{c_{si}}_{C^T[i, s]} \cdot \underbrace{c_{sj}}_{C[s, j]} = C^T C[i, j]$$
    
    Так как $C^T C = E_n$, то $C$ --- ортогональная матрица.
\end{proof} % Матрица перехода между ОНБ в ЕП
    \section{Свойства ортогонального дополнения в ЕП}

\begin{defn}
    $L$ --- ЕП, $V \le \lsub{\mathbb{R}}{L}$. Подпространство $V^\perp = \left\{x \in L \mid \forall v \in V\ (v, x) = 0 \right\} \le L$ называется \textit{ортогональным дополнением} к $V$.
\end{defn}

\begin{thm*}
    $L$ --- ЕП. Пусть $V, V_1, V_2 \le \lsub{\mathbb{R}}{L}$. Справедливы следующие утверждения:
    \begin{enumerate}
        \item $V_1 \subset V_2 \Rightarrow V_2^\perp \subset V_1^\perp$
        \item $(V_1 + V_2)^\perp = V_1^\perp \cap V_2^\perp$
        \item $L = V \oplus V^\perp$
        \item $(V^\perp)^\perp = V$
        \item $(V_1 \cap V_2)^\perp = V_1^\perp + V_2^\perp$
    \end{enumerate}
\end{thm*}

\begin{proof}
    \begin{proofpart}
        Рассмотрим произвольный $v_2' \in V_2^\perp$. Имеем $\forall v_2 \in V_2\ (v_2, v_2') = 0 \Rightarrow \forall v_1 \in V_1\ (v_1, v_2') = 0$. При этом $\forall v_1 \in V_1\ \forall v_1' \in V_1^\perp\ (v_1, v_1') = 0$. Так как $V_1^\perp$ максимально по включению, $V_2^\perp \subset V_1^\perp$.
    \end{proofpart}

    \begin{proofpart}
        $V_1, V_2 \subset V_1 + V_2 \stackrel{\text{(а)}}{\Rightarrow} (V_1 + V_2)^\perp \subset V_1^\perp \cap V_2^\perp$. В то же время, пусть $u \in V_1^\perp \cap V_2^\perp$ и $v = v_1 + v_2 \in V_1 + V_2$, тогда $(u, v) = (u, v_1) + (u, v_2) = 0 + 0 = 0 \leadsto u \in (V_1 + V_2)^\perp$ и обратное включение доказано.
    \end{proofpart}

    \begin{proofpart}
        Считаем, что $V \neq \{\nil\}$. Выберем в $V$ ОН-базис $\family{e_i}{i=1}{k}$ и дополним его до базиса $L$: $\family{e_i}{i=1}{n}$. Докажем, что $V^\perp = \gen{}{v_{k+1}, \dots, v_n}$. Пусть $v = \sum_{i=1}^n \alpha_i e_i \in V^\perp$. Тогда $\forall j \le k$ имеем $0 = (e_j, v) = \sum_{i=1}^n \alpha_i (e_i, e_j) = \sum_{i=1}^n \alpha_i \delta_{ij} = \alpha_j$, откуда $v \in \gen{}{v_{k+1}, \dots, v_n}$. Обратное включение очевидно. Так, $V = V \oplus V^\perp$ (данная сумма прямая в силу того, что если $x \in X \cap X^\perp$, то $(x, x) = 0$ и $x = \nil$).
    \end{proofpart}

    \begin{proofpart}
        $L = V \oplus V^\perp$. Выделяя несобственные подпространства в $V$ и $V^\perp$ и применяя пункт (в), получаем $L = V^\perp \oplus V^{\perp\perp}$. При этом $\dim \lsub{\mathbb{R}}{V} = \dim \lsub{\mathbb{R}}{V^{\perp\perp}}$ и очевидно, что $V \subset V^{\perp\perp}$, откуда $V = V^{\perp\perp}$.
    \end{proofpart}

    \begin{proofpart}
        $(V_1 \cap V_2)^\perp \stackrel{\text{(б)}}{=} (V_1^\perp + V_2^\perp)^{\perp\perp} \stackrel{\text{(г)}}{=} V_1^\perp + V_2^\perp$.
    \end{proofpart}
\end{proof} % Свойства ортогонального дополнения в ЕП
    \section{Обобщенная теорема Пифагора}

\begin{thm*}
    $L$ --- ЕП. Пусть $\family{v_i}{i=1}{m}$ --- ортогональный набор векторов $L$. Тогда $\norm{\sum_{i=1}^m v_i}^2 = \sum_{i=1}^m \norm{v_i}^2$.
\end{thm*}

\begin{proof}
    $${\norm{\sum_{i=1}^m v_i}}^2 = \left(\sum_{i=1}^m v_i,\ \sum_{j=1}^m v_j\right) = \sum_{i,j=1}^m (v_i, v_j) = \sum_{i=1}^m (v_i, v_i) = \sum_{i=1}^m \norm{v_i}^2$$
\end{proof} % Обобщенная теорема Пифагора
    \section{Линейные функционалы на линейном пространстве. Дуальный базис}

\begin{defn}
    $K$ --- поле, $\lsub{K}{V}$ --- линейное пространство. $V^* = \hom(V, K)$ называется \textit{пространством функционалов (также дуальным, двойственным, сопряженным пространством)} на $V$.
\end{defn}

\begin{thm}
    $\dim \lsub{K}{V} < \infty \Rightarrow \dim \lsub{K}{V^*} = \dim \lsub{K}{V}$.
\end{thm}

\begin{proof}
    $\dim \lsub{K}{V} = \dim \lsub{K}{\hom(V, K)} = \dim \lsub{K}{V} \cdot \dim \lsub{K}{K} = \dim \lsub{K}{V}$.
\end{proof}

\begin{defn}
    Пусть $B = \family{u_i}{i=1}{n}$ --- базис $\lsub{K}{V}$. \textit{Дуальным} к $B$ называется базис $B^* = \family{\phi_j}{j=1}{n}$ пространства $V^*$ такой, что $\phi_j(u_i) = \delta_{ij}$.
\end{defn}

\begin{thm}
    $B^*$ --- действительно базис $\lsub{K}{V^*}$.
\end{thm}

\begin{proof}
    Пусть $\phi \in V^*$.
    
    \begin{align*}
        \phi(v) &= \phi\left(\sum_{i=1}^n \alpha_i u_i\right) = \sum_{i=1}^n \alpha_i \phi(u_i) = \sum_{i=1}^n \alpha_i \sum_{j=1}^n \phi(u_j) \phi_j(u_i) \\
        &= \sum_{j=1}^n \phi(u_j) \sum_{i=1}^n \alpha_i \phi_j(u_i) = \sum_{j=1}^n \phi(u_j) \phi_j \left(\sum_{i=1}^n \alpha_i u_i \right) = \sum_{j=1}^n \phi(u_j) \phi_j(v)
    \end{align*}
    
    Отсюда $\phi \in \gen{K}{B^*}$. Так как $\#B^* = \dim \lsub{K}{V^*}$, то $B^*$ --- базис $V^*$.
\end{proof} % Линейные функционалы на линейном пространстве. Дуальный базис
    \section{Дуальное отображение. Свойства операции перехода к дуальному отображению. Второе пространство функционалов, его связь с исходным пространством}

\begin{defn}
    Пусть $f \colon \lsub{K}{U} \to \lsub{K}{V}$ --- $K$-линейное отображение. Построим $f^* \colon V^* \to U^*$ следующим образом:
    
    \begin{diagram}
        U &                       & \rTo^f &            & V \\
          & \rdDashto_{f^*(\phi)} &        & \ldTo_\phi &   \\
          &                       & K      &            &   
    \end{diagram}

    То есть, для $\phi \in V^*$ положим $f^*(\phi) \coloneqq \phi \circ f \in U^*$. Очевидно, что оно $K$-линейно. $f^*$ называется \textit{дуальным к $f$} отображением.
\end{defn}

\begin{rem}
    Дуальное отображение обладает следующими свойствами:
    \begin{enumerate}
        \item $(f_1 + f_2)^*(\phi) = \phi \circ (f_1 + f_2) = \phi \circ f_1 + \phi \circ f_2 = f_1^*(\phi) + f_2^*(\phi)$
        \item $(gf)^*(\phi) = \phi (gf) = (\phi g)f = f^*(\phi g) = f^* (g^*(\phi)) = (f^* \circ g^*)(\phi)$
    \end{enumerate}
\end{rem}

\begin{defn}
    $\lsub{K}{V}$ --- линейное пространство. Построим отображение $\delta_V \colon V \to V^{**}$ так, чтобы $\forall \phi \in V^*\ (\delta_V(v))(\phi) = \phi(v)$ ($\delta_V(v) = \bullet(v)$).
\end{defn}

\begin{thm*}
    $\lsub{K}{U}$, $\lsub{K}{V}$ --- линейные пространства.
    \begin{enumerate}
        \item $\delta_V$ --- изоморфизм.
        \item Пусть $f \colon U \to V$ --- $K$-линейное отображение, тогда $\delta_V \circ f = f^{**} \circ \delta_U$.
    \end{enumerate}
\end{thm*}

\begin{rem}
    $\delta$ можно рассматривать как функтор в терминах теории категорий (следующая диаграмма коммутирует):
    
    \begin{diagram}
        U                    & \rTo^f        & V                    \\
        \dTo^{\delta_U}_\sim &               & \dTo_{\delta_V}^\sim \\
        U^{**}               & \rTo^{f^{**}} & V^{**}
    \end{diagram}
\end{rem}

\begin{proof}
    \begin{proofpart}
        Докажем, что $\delta_V$ мономорфно. Пусть $v \in V \setminus \nilset$ и $\delta_V(v) = 0$. Тогда $\forall \phi \in V^*\ \phi(v) = 0$. Построим базис $\{v, \dots\}$ и двойственный к нему $\{\phi_1, \dots\}$. Отсюда по определению $\phi_1(v) = 1$. Полученное противоречие доказывает мономорфность $\delta_V$. Так как $\delta_V$ --- мономорфизм пространств одинаковой размерности, то $\delta_V$ --- изоморфизм.
    \end{proofpart}

    \begin{proofpart}
        \begin{align*}
            ((\delta_V \circ f)(u))(\phi) &= (\delta_V(f(u)))(\phi) = \phi(f(u)) = (\phi \circ f)(u) = (f^*(\phi))(u) = (\delta_U(u))(f^*(\phi)) \\
            &= (\delta_U(u) \circ f^*)(\phi) = (f^{**}(\delta_U(u)))(\phi) = (f^{**} \circ \delta_U(u))(\phi)
        \end{align*}
        
        Так, $\delta_V \circ f = f^{**} \circ \delta_U$.
    \end{proofpart}
\end{proof} % Дуальное отображение. Свойства операции перехода к дуальному отображению. Второе пространство функционалов, его связь с исходным пространством
    \section{Линейные функционалы на ЕП}

\begin{defn}
    $L$ --- ЕП. Пусть $v \in L$. Отображение $\varphi_v \colon L \to \mathbb{R}$, т.ч. $\varphi_v(u) = (u, v)$ назовем \textit{функционалом скалярного умножения} на $v$.
\end{defn}

\begin{thm*}
    $L$ --- ЕП. Рассмотрим $f \colon L \to L^*$, т.ч. $f(v) = \varphi_v$. Справедливо следующее:
    \begin{enumerate}
        \item $f$ --- изоморфизм.
        \item Если $B$ --- ОН-базис $\lsub{\mathbb{R}}{V}$, то $f(B)$ --- дуальный к нему в $\lsub{\mathbb{R}}{L^*}$.
    \end{enumerate}
\end{thm*}

\begin{proof}
    \begin{proofpart}
        Отображение $f$ $K$-линейно за счет линейности скалярного произведения по первому аргументу. Докажем мономорфность $f$. Пусть $f(v) = 0$ для $v \in L$. Тогда $(\bullet, v) \equiv 0 \leadsto (v, v) = 0 \leadsto v = \nil$. Так как $\dim \lsub{\mathbb{R}}{L^*} = \dim \lsub{\mathbb{R}}{L}$, то $f$ --- мономорфизм $K$-линейных пространств одинаковой размерности, и, как следствие, является изоморфизмом.
    \end{proofpart}
    \begin{proofpart}
        Пусть $B = \family{e_i}{i=1}{n}$. Тогда $f(B) = \family{\varphi_j = (\bullet, e_j)}{j=1}{n}$. В частности, $\varphi_j(e_i) = (e_i, e_j) = \delta_{ij}$, откуда $f(B)$ --- дуальный к $B$ базис $L^*$.
    \end{proofpart}
\end{proof}

\begin{cor*}
    $L$ --- ЕП. Пусть $\varphi \colon \lsub{\mathbb{R}}{L} \to \mathbb{R} \in L^*$, тогда $\exists! v \in L \colon \varphi(u) = (u, v)\ \forall u \in L$.
\end{cor*} % Линейные функционалы на ЕП
    \section{Проекторы в линейных пространствах}

\begin{defn}
    $\lsub{K}{V}$ --- линейное пространство. Эндоморфизм $p \in \End \lsub{K}{V}$ называется \textit{проектором (оператором проектирования)}, если $p^2 = p$.
\end{defn}

\begin{exmpl}
    $$\begin{array}{ c c c c c }
        \lsub{K}{V} & = & U     & \oplus & U'    \\
        \inup       &   & \inup &        & \inup \\
        v           & = & u     & +      & u'
    \end{array}$$
    $p_U(v) \coloneqq u$ --- проектор, называемый \textit{проектором на $U$ параллельно $U'$}.
\end{exmpl}

\begin{thm*}
    Пусть $p \in \End \lsub{K}{V}$ --- проектор. Тогда:
    \begin{enumerate}
        \item $V = \ker p \oplus \im p$.
        \item $p$ совпадает с проектором на $\im p$ параллельно $\ker p$.
    \end{enumerate}
\end{thm*}

\begin{proof}
    \begin{proofpart}
        Пусть $v \in V$. Тогда $p(v) = p^2(v)$. Положим $y \coloneqq v - p(v)$. Ясно, что $p(y) = \nil$. Так:
        \begin{equation}\label{09-10:star}\tag{$*$}
            v = p(v) + y, \qquad p(v) \in \im p,\ y \in \ker p
        \end{equation}
        откуда $V = \im p + \ker p$. Пусть $x \in \ker p \cap \im p$, тогда $\exists z \in V \colon x = p(z)$ и $p(x) = \nil$. Отсюда $\nil = p(x) = p^2(z) = p(z) = x$. Следовательно, $\ker p \cap \im p = \nilset$ и соответствующая сумма прямая.
    \end{proofpart}
    \begin{proofpart}
        Явным образом следует из \eqref{09-10:star}.
    \end{proofpart}
\end{proof} % Проекторы в линейных пространствах
    \section{Ортогональное проектирование на подпространство}

\begin{defn}
    $L$ --- ЕП, $V \le \lsub{\mathbb{R}}{L}$, $L = V \oplus V^\perp$. Проектор на $V$ параллельно $V^\perp$ называется \textit{оператором ортогонального проектирования на $V$}. В этом случае $v \in L$ единственным образом представляется в виде $v = v_{\text{pr}} + v^\perp$, где $v_{\text{pr}}$ называется \textit{ортогональной проекцией $v$ на $V$}, а $v^\perp$ --- \textit{ортогональной составляющей $v$}.
\end{defn}

\begin{defn}
    $L$ --- ЕП, $V \le \lsub{\mathbb{R}}{L}$, $v \in L$. Величина $\rho(v, V) \coloneqq \inf \left\{ \rho(v, x) \mid x \in V \right\}$ называется \textit{расстоянием между $v$ и $V$}.
\end{defn}

\begin{thm*}
    $L$ --- ЕП, $V \le \lsub{\mathbb{R}}{L}$, $v \in L$. Пусть $p$ --- оператор ортогонального проектирования на $V$, тогда $\rho(v, V) = \norm{v^\perp} = \norm{v - p(v)}$.
\end{thm*}

\begin{proof}
    Положим $v_{\text{pr}} \coloneqq p(v) \in V$. Очевидно, что $\rho(v, V) \le \norm{v^\perp}$ (т.к. $\rho(v, v_\text{pr}) = \norm{v - v_\text{pr}} = \norm{v^\perp}$). В то же время, полагая $u \in V$, имеем $\rho^2(v, u) = \norm{v - u}^2 = \norm{v^\perp + (v_\text{pr} - u)}^2 = \norm{v^\perp}^2 + \norm{v_\text{pr} - u}^2 \ge \norm{v^\perp}^2$ (в силу ортогональности векторов суммы), откуда $\rho(v, V) \ge \norm{v^\perp}$. Так, $\rho(v, V) = \norm{v^\perp}$.
\end{proof} % Ортогональное проектирование на подпространство
    
    % Глава X. Полилинейная алгебра
    \part{Полилинейная алгебра}
    %\section{Понятие тензорнго произведения модулей (определение, конструкция, следствия, примеры)}

\begin{defn}
    $k$ --- коммутативное кольцо с единицей, $\lsub{k}{U}$ и $\lsub{k}{V}$ --- модули. \textit{Тензорным произведением} $U$ и $V$ называется пара $(T, t)$, где $\lsub{k}{T}$ --- модуль, а $t \colon U \times V \to T$ --- билинейное отображение, обладающее \textit{универсальным свойством тензорного произведения}:
    \begin{diagram}
        U \times V &         & \rTo^t &                   & T \\
                   & \rdTo_f &        & \ldDashto_\varphi &   \\
                   &         & W      &                   &
    \end{diagram}
    то есть $\forall f \colon U \times V \to W$ --- билинейного отображения, где $\lsub{k}{W}$ --- некоторый модуль, $\exists!$ $k$-гомоморфизм $\varphi \colon T \to W$, т.ч. $\varphi t = f$.
    
    Тензорное произведение единственно с точностью до изоморфизма. В связи с этим используется стандартное обозначение $T \eqqcolon U \otimes_k V$ (или, если кольцо ясно из контекста, $U \otimes V$).
\end{defn}

\begin{thm*}
    $k$ --- коммутативное кольцо с единицей. Пусть $U$ и $V$ --- $k$-модули, тогда их тензорное произведение существует.
\end{thm*}

\begin{proof}
    Пусть $X \coloneqq U \times V$. Построим свободный модуль $F \coloneqq F\langle X \rangle$. Положим:
    \begin{align*}
        Y_1 &\coloneqq \left\{(\alpha_1u_1 + \alpha_2u_2, v) - \alpha_1(u_1, v) - \alpha_2(u_2, v) \mid \alpha_1, \alpha_2 \in k, u_1, u_2 \in U, v \in V \right\} \\
        Y_2 &\coloneqq \left\{(u, \alpha_1v_1 + \alpha_2v_2) - \alpha_1(u, v_1) - \alpha_2(u, v_2) \mid \alpha_1, \alpha_2 \in k, u \in U, v_1, v_2 \in V \right\} \\
        Y & \coloneqq Y_1\ \mathring{\cup}\ Y_2
    \end{align*}

    Пусть $G \coloneqq \gen{K}{Y} \le \lsub{k}{F}$. Рассмотрим $\lsub{k}{T} \coloneqq F/G$. Введем отображение $t \colon U \times V \to T$, т.ч. $(u, v) \mapsto (u, v) + G$ --- билинейное вследствие определения $G$.

    Докажем, что пара $(T, t)$ удовлетворяет универсальному свойству тензорного произведения. Пусть $\rho \colon F \to T$ --- отображение факторизации, $i \colon U \times V \to F$ --- вложение. Ясно, что $t = \rho i$. Пусть $\lsub{k}{W}$ --- произвольный модуль и $f \colon U \times V \to W$ --- билинейное отображение. Рассмотрим следующую диаграмму:
    \begin{diagram}
        U \times V & & \rTo^t & & T \\
        & \rdTo(2, 4)_f \rdInto^i & & \ruOnto^\rho \ldDashto(2, 4)_\varphi & \\
        & & F & & \\
        & & \dDashto_{\tilde{f}} & & \\
        & & W & &
    \end{diagram}
    
    По определению $F$ $\exists!$ $k$-гомоморфизм $\tilde{f} \colon F \to W$, т.ч. $f = \tilde{f} i$. Покажем, что $G \subset \ker \tilde{f}$. Для этого достаточно доказать, что $Y \subset \ker \tilde{f}$. Рассмотрим это на примере $Y_1$ (для $Y_2$ действия аналогичны):
    \begin{align*}
        f(y_1) &= \tilde{f}((\alpha_1 u_1 + \alpha_2 u_2, v) - \alpha_1(u_1, v) - \alpha_2(u_2, v)) \\
        &= \tilde{f}((\alpha_1 u_1 + \alpha_2 u_2, v)) - \alpha_1 \tilde{f}((u_1, v)) - \alpha_2 \tilde{f}((u_2, v)) = \\
        &= \tilde{f}(i(\alpha_1 u_1 + \alpha_2 u_2, v)) - \alpha_1 \tilde{f}(i(u_1, v)) - \alpha_2 \tilde{f}(i(u_2, v)) = \\
        &= f(\alpha_1 u_1 + \alpha_2 u_2, v) - \alpha_1 f(u_1, v) - \alpha_2 f(u_2, v) = f(0, v) = 0
    \end{align*}
    
    По теореме о продолжении гомоморфизма на фактормодуль $\exists!$ $k$-гомоморфизм $\varphi \colon T \to W$, т.ч. $\tilde{f} = \varphi \rho$. Тогда $\varphi t = \varphi \rho i = \tilde{f} i = f$. Предположим, что $\tilde{\varphi} \colon T \to W$ --- еще один $k$-гомоморфизм, т.ч. $\tilde{\varphi} t = f$. Тогда $f = (\tilde{\varphi} \rho) i$. При этом $\tilde{f} i = f$ и $\tilde{f}$ --- единственное отображение, обладающее таким свойством. В таком случае $\tilde{\varphi} \rho = \tilde{f}$, но $\phi$ --- единственное отображение, обладающее соответствующим свойствим, откуда $\varphi = \tilde{\varphi}$.
\end{proof}

\begin{defn}
    $t \colon U \times V \to T$ называют \textit{каноническим билинейным отображением}. Широко используется обозначение $t(u, v) \eqqcolon u \otimes v$.
\end{defn}

\begin{cor}
    Множество $\{u \otimes v \mid u \in U, v \in V\}$ порождает $U \otimes_k V$ как $k$-модуль.
\end{cor}

\begin{proof}
    Следует из того, что $U \times V$ порождает $F\langle U \times V \rangle$.
\end{proof}

\begin{cor}
    $U \otimes_k V \simeq V \otimes_k U$ как $k$-модули.
\end{cor}

\begin{proof}
    Рассмотрим следующую диаграмму:
    \begin{diagram}
        U \times V & \rTo^t & U \otimes_k V \\
        & \rdTo_{t'} & \dTo^\sigma \uTo_\tau \\
        & & V \otimes_k U
    \end{diagram}
    
    Положим $t \colon (u, v) \mapsto u \otimes v$ и $t' \colon (u, v) \mapsto v \otimes u$ --- билинейные отображения. Тогда $\exists!$ $k$-гомоморфизмы $\sigma$ и $\tau$, т.ч. $\sigma t = t'$ и $\tau t' = t$, откуда $\sigma(u \otimes v) = v \otimes u$ и $\tau(v \otimes u) = u \otimes v$ соответственно. В таком случае $\sigma \tau(u \otimes v) = u \otimes v$ и при этом $\{u \otimes v \mid u \in u, v \in V\}$ порождает $U \otimes V$, откуда $\sigma \tau = \id_{U \otimes V}$. Аналогично, $\tau \sigma = \id_{V \otimes U}$ и $\sigma$ --- искомый изоморфизм.
\end{proof}

\begin{exmpl}\
    \begin{enumerate}
        \item Пусть $k$ --- коммутативное кольцо с единицей, $U$ --- $k$-модуль. Тогда $k \otimes_k U \simeq U$.
        \begin{proof}
            Рассмотрим $f \colon k \times U \to U$, т.ч. $f(\alpha, u) = \alpha u$. Рассмотрим дополнительно следующую диаграмму:
            \begin{diagram}
                k \times U & & \rTo^t & & k \otimes U \\
                & \rdTo_f & & \ldDashto_\sigma & \\
                & & U & &
            \end{diagram}
            Согласно универсальному свойству, $\exists! \sigma \colon k \otimes U \to U$, т.ч. $f = \sigma t$.
            
            Рассмотрим $\tau \colon U \to k \otimes U$, т.ч. $u \mapsto 1 \otimes u$. Тогда $\tau \sigma(\alpha \otimes u) = \tau \sigma t(\alpha, u) = \tau f(\alpha, u) = \tau(\alpha u) = \alpha(1 \otimes u) = \alpha \otimes u$. Аналогично, $\sigma \tau(u) = \sigma(1 \otimes u) = \sigma t(1, u) = f(1, u) = u$, и $\sigma$ --- изоморфизм.
        \end{proof}
    
        \item Пусть $k = \mathbb{Z}$, $U = \mathbb{Z} / m\mathbb{Z}$, $V = \mathbb{Z} / n\mathbb{Z}$ и $(m, n) = 1$. Тогда $U \otimes_{\mathbb{Z}} V = \{0\}$.
        
        \begin{proof}
            $(m, n) = 1 \leadsto \exists r, s \in \mathbb{Z} \colon 1 = mr + ns \leadsto \forall u \in U, v \in V\ u \otimes v = 1 \cdot (u \otimes v) = r(mu) \otimes v + s \cdot u \otimes (nv) = r \cdot 0 \otimes v + s \cdot u \otimes 0 = 0$.
        \end{proof}
    \end{enumerate}
\end{exmpl}
\end{document}
